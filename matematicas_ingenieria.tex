\documentclass[12pt,letterpaper]{report}

% ==================== PAQUETES ====================
\usepackage[spanish,es-tabla]{babel}
\usepackage[utf8]{inputenc}
\usepackage[T1]{fontenc}
\usepackage{amsmath,amssymb,amsthm}
\usepackage{graphicx}
\usepackage[left=2.5cm,right=2.5cm,top=3cm,bottom=3cm]{geometry}
\usepackage{setspace}
\usepackage{fancyhdr}
\usepackage{titlesec}
\usepackage{tocloft}
\usepackage[hidelinks]{hyperref}
\usepackage{float}
\usepackage{caption}
\usepackage{subcaption}
\usepackage{enumitem}
\usepackage{csquotes}

% ==================== CONFIGURACIONES ====================
% Interlineado
\onehalfspacing

% Encabezados y pies de página
\pagestyle{fancy}
\fancyhf{}
\fancyhead[L]{\leftmark}
\fancyhead[R]{\thepage}
\renewcommand{\headrulewidth}{0.5pt}

% Formato de capítulos y secciones
\titleformat{\chapter}[display]
  {\normalfont\huge\bfseries}{\chaptertitlename\ \thechapter}{20pt}{\Huge}
\titlespacing*{\chapter}{0pt}{0pt}{40pt}

% Configuración de índices
\renewcommand{\contentsname}{Índice}
\renewcommand{\listfigurename}{Índice de Figuras}
\renewcommand{\listtablename}{Índice de Tablas}
\renewcommand{\bibname}{Referencias}

% Numeración de ecuaciones por capítulo
\numberwithin{equation}{chapter}
\numberwithin{figure}{chapter}
\numberwithin{table}{chapter}

% ==================== INFORMACIÓN DEL DOCUMENTO ====================
\title{Investigación de Matemáticas para la Ingeniería I}
\author{
    Isael Alejandro Reyes Vargas \\
    Jafet Bahena González \\
    Danna Paola Sánchez Martínez \\
    Angel Gabriel Aguilar García \\
    Loreley Carrillo Juárez
}
\date{\today}

% ==================== INICIO DEL DOCUMENTO ====================
\begin{document}

% ==================== PORTADA ====================
\begin{titlepage}
    \centering
    \vspace*{1cm}
    
    {\Large\textbf{UNIVERSIDAD TECNOLÓGICA EMILIANO ZAPATA}}\\[0.5cm]
    {\large División Académica de Tecnologías de la Información y Diseño}\\[0.5cm]
    {\large Ingeniería en Desarrollo y Gestión de Software}\\[2cm]
    
    \rule{\linewidth}{0.5mm}\\[0.4cm]
    {\huge\bfseries Integradora de Matemáticas\\[0.2cm]para la Ingeniería I}\\[0.2cm]
    \rule{\linewidth}{0.5mm}\\[2cm]
    
    {\Large\textbf{Proyecto Integrador}}\\[1.5cm]
    
    \begin{minipage}{0.8\textwidth}
        \begin{flushleft}
            \textbf{Materia:}\\
            Matemáticas para la Ingeniería I\\[0.5cm]
            
            \textbf{Profesor:}\\
            M.C. Jorge Yusef Colin Castillo\\[0.5cm]
            
            \textbf{Grupo:} 7 C\\[0.5cm]
            
            \textbf{Integrantes del Equipo:}\\
            \begin{itemize}[leftmargin=1cm]
                \item Isael Alejandro Reyes Vargas
                \item Jafet Bahena González
                \item Danna Paola Sánchez Martínez
                \item Angel Gabriel Aguilar García
                \item Loreley Carrillo Juárez
            \end{itemize}
        \end{flushleft}
    \end{minipage}
    
    \vfill
    
    {\large Emiliano Zapata, Morelos, México}\\
    {\large \today}
\end{titlepage}

% ==================== ÍNDICES ====================
\tableofcontents
\cleardoublepage 

\listoffigures
\cleardoublepage 

\listoftables
\cleardoublepage 

% ==================== UNIDAD I ====================
\chapter{Unidad I: Funciones y Límites}

\section{Funciones}

La definición formal moderna de una función proviene de la teoría de conjuntos y está basada en la relación entre dos conjuntos. Sin embargo, la noción de función no siempre fue tan precisa. Históricamente, los matemáticos comenzaron a tratar con funciones sin formalizar completamente su definición.

La función se puede conceptualizar como una relación entre dos conjuntos, en la cual a cada elemento del primer conjunto (dominio) se le asigna un único elemento del segundo conjunto (contradominio).

En términos matemáticos, una función se puede definir como:
\begin{equation}
f: A \rightarrow B
\label{eq:definicion_funcion}
\end{equation}

Donde:
\begin{itemize}
    \item $A$ es el dominio de la función, que es el conjunto de todos los valores posibles de entrada.
    \item $B$ es el codominio, que es el conjunto de posibles valores de salida.
    \item Para cada $x \in A$, existe un único valor $f(x) \in B$.
\end{itemize}

\subsection{Historia}

El concepto de función comenzó a desarrollarse en el siglo XVII con los trabajos de matemáticos como René Descartes e Isaac Newton, pero su formalización no llegó hasta el siglo XIX. En la antigua Grecia, los matemáticos utilizaban nociones de relación entre cantidades, pero no se hablaba de funciones como una entidad matemática en sí misma.

\textbf{Leibniz (1680s):} Introdujo el concepto de una función en su trabajo sobre cálculo, lo que llevó a la adopción de la notación moderna que ahora usamos.

\textbf{Euler (1740s):} Definió funciones de manera más general y comenzó a utilizar la notación $f(x)$ para denotar el valor de la función $f$ para un valor dado de $x$.

\textbf{Cauchy y Weierstrass (Siglo XIX):} Formalizaron el concepto de función, especialmente con la definición de funciones continuas y diferenciables.

\subsection{Propiedades Fundamentales de las Funciones}

\subsubsection{Unicidad}

Una de las propiedades clave de las funciones es que a cada valor del dominio debe corresponder un único valor en el contradominio. Esta propiedad asegura que no haya ambigüedades en la asignación de valores.

\textit{Ejemplo:} En la función cuadrática $f(x) = x^2$, para $x = 2$, la imagen es 4, y no puede ser otro valor.

\subsubsection{Existencia y Unicidad de la Función}

Cuando decimos que una función está bien definida, esto significa que para cada $x$ en el dominio, existe un único valor $f(x)$ asociado. Este valor es único para ese $x$.

\subsubsection{Imagen y Pre-imagen}

\begin{itemize}
    \item La \textbf{imagen} de la función es el conjunto de todos los valores que puede tomar la función $f(x)$ a medida que $x$ recorre todo el dominio.
    \item La \textbf{pre-imagen} de un valor $y$ es el conjunto de todos los $x$ tal que $f(x) = y$.
\end{itemize}

\subsection{Aplicaciones de las Funciones}

Las funciones tienen una amplia variedad de aplicaciones en diferentes campos. Algunas de las más comunes son:

\begin{itemize}
    \item \textbf{Cálculo Diferencial e Integral:} Las funciones son esenciales para entender el cambio y el área bajo las curvas.
    \item \textbf{Ingeniería:} Las funciones se utilizan para modelar fenómenos físicos, como el comportamiento de circuitos eléctricos o el movimiento de partículas.
    \item \textbf{Economía:} Se utilizan para modelar la oferta y demanda, el costo de producción, y las tasas de crecimiento.
    \item \textbf{Biología:} Las funciones modelan el crecimiento poblacional, la dispersión de enfermedades y la dinámica de ecosistemas.
\end{itemize}

\section{Dominio}

El dominio de una función se refiere al conjunto de todos los valores posibles que pueden ser entradas válidas para la función. Dicho de otra manera, es el conjunto de valores que podemos sustituir en la variable independiente sin que se produzca una contradicción o un valor indefinido.

\subsection{Definición Formal del Dominio}

Si tenemos una función $f: A \rightarrow B$, entonces el dominio de $f$ es el conjunto $A$, que contiene todos los valores de $x$ para los cuales la función está bien definida. Esto significa que, para cada $x$ en el dominio, la expresión matemática de la función no debe involucrar ninguna operación inválida, como la división entre cero o la raíz cuadrada de un número negativo (en el caso de los números reales).

\subsection{Ejemplos de Restricciones en el Dominio}

Algunas restricciones comunes que pueden existir en el dominio de una función incluyen:

\subsubsection{División por cero}

Una función no puede tener como entrada un valor que cause que su expresión tenga un denominador igual a cero.

\textit{Ejemplo:}
\begin{equation}
f(x) = \frac{1}{x - 1}
\label{eq:division_cero}
\end{equation}

El dominio de esta función es $\{x \in \mathbb{R} \mid x \neq 1\}$, ya que en $x = 1$, el denominador se vuelve cero.

\subsubsection{Raíces de números negativos}

Las funciones que involucran raíces cuadradas o de orden par no pueden tener números negativos en su dominio si estamos trabajando con números reales.

\textit{Ejemplo:}
$$f(x) = \sqrt{x - 2}$$

El dominio de esta función es $\{x \in \mathbb{R} \mid x \geq 2\}$, ya que no se puede tomar la raíz cuadrada de un número negativo.

\subsubsection{Logaritmos}

Los logaritmos no están definidos para valores menores o iguales a cero. Así que si la función es de la forma:
$$f(x) = \ln(x)$$

El dominio es $\{x \in \mathbb{R} \mid x > 0\}$, porque el argumento del logaritmo debe ser positivo.

\subsection{Dominio de Funciones Compuestas}

El dominio de una función compuesta $(f \circ g)(x) = f(g(x))$ es el conjunto de todos los valores $x$ para los cuales tanto $g(x)$ como $f(g(x))$ están definidos. Por ejemplo:

Si $g(x) = \sqrt{x}$ y $f(x) = \frac{1}{x}$, entonces el dominio de $f \circ g$ es $\{x \in \mathbb{R} \mid x > 0\}$, ya que debe cumplirse $\sqrt{x} > 0$.

\section{Rango}

El rango de una función es el conjunto de todos los posibles valores de salida o resultados que puede tomar la función cuando se sustituyen los valores del dominio en la expresión de la función. En otras palabras, el rango es el conjunto de todas las imágenes de los valores del dominio bajo la acción de la función.

\subsection{Definición Formal del Rango}

El rango de la función $f: A \rightarrow B$ es el conjunto de valores $y \in B$ que pueden ser obtenidos para $y = f(x)$. Es decir, el rango es el conjunto de todos los valores de $y$ que pueden ser alcanzados por la función $f$.

\subsection{Ejemplos de Determinación del Rango}

\subsubsection{Función cuadrática}

Consideremos la función $f(x) = x^2$. Esta función toma cualquier número real $x$ y devuelve su cuadrado. El valor más bajo que puede tomar esta función es 0 (cuando $x = 0$), y no hay límite superior. Entonces, el rango de la función es:
$$\text{Rango}(f) = [0, +\infty)$$

\subsubsection{Función lineal}

Para la función lineal $f(x) = 2x + 3$, como es una línea recta, no hay restricciones en los valores que puede tomar la salida de la función. Por lo tanto, el rango de la función es el conjunto de todos los números reales:
$$\text{Rango}(f) = \mathbb{R}$$

\subsubsection{Función seno}

La función $f(x) = \sin(x)$ tiene un rango limitado porque el seno de un ángulo siempre está entre $-1$ y $1$, sin importar el valor de $x$. Por lo tanto, el rango de la función seno es:
\begin{equation}
\text{Rango}(\sin) = [-1, 1]
\label{eq:rango_seno}
\end{equation}

\subsubsection{Función logaritmo}

Si $f(x) = \ln(x)$, el rango de esta función es $\mathbb{R}$, ya que el logaritmo de cualquier número positivo es un número real.

\subsection{Cálculo del Rango de Funciones Compuestas}

El rango de una función compuesta $(f \circ g)(x)$ depende tanto del rango de $g$ como del comportamiento de la función $f$. Para calcularlo, primero debes determinar el rango de $g$ y luego determinar cómo estos valores son transformados por $f$.

\section{Funciones Explícitas}

Una función explícita es aquella en la que la variable dependiente, generalmente representada por la letra $y$, se encuentra despejada y expresada de manera directa en términos de la variable independiente, que suele ser $x$. En este tipo de funciones, la relación se puede escribir de la forma $y = f(x)$, lo que significa que para cada valor de $x$ se puede calcular de manera inmediata un valor de $y$.

Por ejemplo, las funciones $y = 2x + 3$, $y = x^2 - 4x + 1$ o $y = 5e^x$ son todas funciones explícitas porque muestran claramente cómo la variable dependiente depende de la independiente.

Las funciones explícitas son las más comunes y son ampliamente utilizadas en álgebra, cálculo y física, ya que permiten evaluar, graficar y analizar relaciones de manera directa. Su principal característica es que, dado un valor de $x$, se puede calcular sin ambigüedad el valor de $y$. Además, su representación gráfica es sencilla, ya que a cada valor de $x$ le corresponde un solo valor de $y$.

Estas funciones también facilitan el uso de herramientas del cálculo, como las derivadas, los límites o las integrales, ya que la variable dependiente se encuentra aislada y permite aplicar directamente las reglas de derivación o integración.

\subsection{Aplicaciones de las Funciones Explícitas}

Las funciones explícitas se usan ampliamente en distintos campos del conocimiento:

\begin{itemize}
    \item \textbf{Física:} Describen relaciones como la posición de un objeto en función del tiempo, mediante expresiones como $y = \frac{1}{2}gt^2$.
    \item \textbf{Economía:} Una función de costo puede expresarse como $C(q) = 50q + 200$, donde se muestra explícitamente cómo el costo depende de la cantidad producida.
    \item \textbf{Ingeniería:} Las funciones explícitas permiten modelar el comportamiento de sistemas eléctricos o mecánicos donde las variables dependen de forma directa unas de otras.
\end{itemize}

Gracias a esta característica, su uso es esencial para el modelado computacional y la simulación de procesos.

\section{Funciones Implícitas}

Por otro lado, una función implícita es aquella en la que las variables $x$ e $y$ están relacionadas dentro de una misma ecuación, sin que la variable dependiente esté despejada o aislada. En este tipo de funciones, no se puede escribir fácilmente una expresión de la forma $y = f(x)$, sino que la relación se expresa como $F(x, y) = 0$. En una función implícita, la variable dependiente no aparece de forma explícita, sino que su relación con la variable independiente se encuentra implícita dentro de la ecuación.

Un ejemplo clásico de función implícita es la ecuación de una circunferencia:
\begin{equation}
x^2 + y^2 = 25
\label{eq:circunferencia}
\end{equation}

En esta relación, $x$ e $y$ están vinculadas por la ecuación, pero $y$ no está despejada. Si se desea expresar la ecuación de forma explícita, se puede despejar $y$ obteniendo $y = \pm\sqrt{25 - x^2}$. Esto demuestra que, aunque una función implícita pueda convertirse en explícita, a menudo la ecuación original es más simple o más natural en su forma implícita.

Otro ejemplo de función implícita es $x^3 + y^3 - 3xy = 0$, la cual representa una curva algebraica que no puede escribirse fácilmente en forma explícita.

\subsection{Aplicaciones de las Funciones Implícitas}

Las funciones implícitas se utilizan cuando la relación entre variables es demasiado compleja o cuando la ecuación no se puede resolver de manera directa para aislar $y$. Son muy comunes en geometría analítica, donde se usan para representar curvas como circunferencias, elipses o hipérbolas.

En cálculo diferencial, también aparecen con frecuencia al estudiar derivadas de relaciones que no están dadas de forma explícita. Para este tipo de funciones se emplea la técnica de derivación implícita, la cual permite encontrar la tasa de cambio de $y$ respecto a $x$ incluso cuando $y$ no está despejada.

\subsection{Derivación Implícita}

La derivación implícita se basa en aplicar la regla de la cadena a ambas variables. Si se parte de una ecuación $F(x, y) = 0$, se diferencia con respecto a $x$ y se obtiene la relación:
$$\frac{\partial F}{\partial x} + \frac{\partial F}{\partial y}\frac{dy}{dx} = 0$$

Despejando, se obtiene la expresión general:
\begin{equation}
\frac{dy}{dx} = -\frac{\partial F/\partial x}{\partial F/\partial y}
\label{eq:derivacion_implicita}
\end{equation}

Por ejemplo, si se toma la ecuación $x^2 + y^2 = 25$, al derivar ambos lados con respecto a $x$ se obtiene:
$$2x + 2y\frac{dy}{dx} = 0$$

Despejando resulta:
$$\frac{dy}{dx} = -\frac{x}{y}$$

De esta forma se puede obtener la pendiente de la tangente en cualquier punto de la circunferencia, sin necesidad de despejar $y$.

\subsection{Teorema de la Función Implícita}

Existe un resultado fundamental relacionado con las funciones implícitas, conocido como el \textbf{Teorema de la Función Implícita}. Este teorema establece las condiciones bajo las cuales una relación implícita puede expresarse localmente como una función explícita.

Si $F(x, y)$ es una función continua y diferenciable y se cumple que $F(a, b) = 0$ y que la derivada parcial de $F$ con respecto a $y$, es decir, $\frac{\partial F}{\partial y}(a, b)$, es distinta de cero, entonces existe una función diferenciable $y = g(x)$ definida en una vecindad del punto $(a, b)$ tal que $F(x, g(x)) = 0$.

En términos sencillos, esto significa que, aunque globalmente no se pueda despejar $y$, en una región cercana a un punto particular sí puede comportarse como una función explícita. Este teorema es una de las herramientas más importantes en cálculo avanzado y análisis matemático, ya que permite estudiar relaciones complicadas sin necesidad de resolverlas completamente.

\section{Curvas de Nivel}

Las curvas de nivel son representaciones gráficas utilizadas en matemáticas para mostrar el comportamiento de funciones de dos variables. Una curva de nivel se define como el conjunto de puntos en el plano donde una función de dos variables toma un mismo valor constante.

En otras palabras, si se tiene una función $f(x, y)$, una curva de nivel se forma uniendo todos los puntos $(x, y)$ tales que $f(x, y) = c$, donde $c$ es una constante real. Estas curvas permiten visualizar cómo cambia una función en el espacio bidimensional, ya que muestran regiones donde la función tiene el mismo valor.

\subsection{Definición Formal}

Formalmente, si se considera una función $f: \mathbb{R}^2 \rightarrow \mathbb{R}$, las curvas de nivel se obtienen al resolver la ecuación:
\begin{equation}
f(x, y) = c
\label{eq:curva_nivel}
\end{equation}

para distintos valores de $c$. Cada valor de $c$ genera una curva diferente, y el conjunto de todas estas curvas proporciona una vista general del comportamiento de la función. De manera análoga, en tres dimensiones, las superficies de nivel representan el conjunto de puntos donde $f(x, y, z) = c$.

\subsection{Aplicaciones}

El concepto de curva de nivel tiene una gran utilidad en la comprensión de funciones multivariables, ya que permite analizar cómo varía la altura o el valor de la función en función de las variables de entrada.

En física y geografía, por ejemplo, las curvas de nivel se utilizan en los mapas topográficos para representar la elevación del terreno. En este caso, cada curva de nivel representa todos los puntos que se encuentran a la misma altitud. Cuanto más juntas se encuentren las curvas, mayor es la pendiente o el cambio en la elevación.

En matemáticas, se utilizan para estudiar el crecimiento o decrecimiento de funciones, la existencia de máximos y mínimos locales, y la forma de las superficies que describen.

\subsection{Ejemplos}

Para ilustrar este concepto, consideremos la función:
$$f(x, y) = x^2 + y^2$$

Si se elige un valor constante $c = r^2$, la ecuación de la curva de nivel será $x^2 + y^2 = r^2$. Esta expresión representa una familia de circunferencias concéntricas con centro en el origen y radio igual a $r$. Cada curva corresponde a un valor distinto de la función, y su disposición muestra cómo la función aumenta conforme nos alejamos del origen.

Este tipo de análisis es fundamental en el cálculo multivariable, donde las curvas de nivel permiten comprender de manera visual las propiedades locales y globales de una función.

En otras aplicaciones, las curvas de nivel permiten analizar funciones más complejas. Por ejemplo, si se considera $f(x, y) = x^2 - y^2$, las curvas de nivel estarán formadas por hipérbolas, ya que la ecuación $x^2 - y^2 = c$ define una familia de curvas hiperbólicas que se abren en distintas direcciones dependiendo del signo de $c$.

\subsection{Relación con el Gradiente}

El análisis de curvas de nivel tiene también una interpretación práctica en el estudio de gradientes. El gradiente de una función, denotado por $\nabla f$, es un vector que apunta en la dirección del máximo crecimiento de la función, y es siempre perpendicular a las curvas de nivel en un punto dado.

Esta propiedad es muy importante en optimización y análisis de superficies, ya que permite identificar puntos de máximo o mínimo y direcciones de mayor variación.

\section{Límites}

El concepto de límite es uno de los pilares fundamentales del cálculo diferencial e integral. El límite describe el comportamiento de una función cuando la variable independiente se aproxima a un cierto valor. En términos sencillos, el límite permite estudiar lo que sucede con el valor de una función cuando $x$ se acerca a un número determinado, incluso si la función no está definida exactamente en ese punto.

Los límites proporcionan la base para definir de manera rigurosa conceptos esenciales como la continuidad, la derivada y la integral.

\subsection{Definición Formal}

Formalmente, se dice que el límite de una función $f(x)$ cuando $x$ tiende a un número $a$ es igual a $L$, y se escribe como:
\begin{equation}
\lim_{x \to a} f(x) = L
\label{eq:definicion_limite}
\end{equation}

Esto significa que los valores de $f(x)$ pueden hacerse tan cercanos a $L$ como se desee, siempre que $x$ se acerque suficientemente a $a$, pero sin necesidad de que $x$ tome exactamente ese valor. Si el límite existe, se dice que la función tiene un comportamiento estable en torno a ese punto. Si el límite no existe, la función puede presentar discontinuidades, saltos o comportamientos infinitos.

\subsection{Interpretación Intuitiva}

El límite se puede entender intuitivamente como la tendencia de una función a un valor determinado. Por ejemplo, si se considera la función:
$$f(x) = \frac{x^2 - 1}{x - 1}$$

Al sustituir $x = 1$ se obtiene una indeterminación $\frac{0}{0}$, ya que el denominador se vuelve cero. Sin embargo, al analizar los valores de $x$ cercanos a 1, se observa que la función se aproxima a 2. Por lo tanto, el límite de la función cuando $x$ tiende a 1 es 2, es decir:
$$\lim_{x \to 1} \frac{x^2 - 1}{x - 1} = 2$$

Este ejemplo demuestra que el límite permite estudiar el comportamiento de las funciones en puntos donde no están definidas.

\subsection{Límites en Funciones de Varias Variables}

El concepto de límite no solo se aplica a funciones de una variable, sino también a funciones de varias variables. En estos casos, el límite estudia el comportamiento de una función $f(x, y)$ cuando el punto $(x, y)$ se aproxima a un valor $(a, b)$.

Para que el límite exista, el valor al que se aproxima la función debe ser el mismo sin importar la dirección desde la cual $(x, y)$ se acerca al punto $(a, b)$.

Por ejemplo, para la función:
$$f(x, y) = \frac{xy}{x^2 + y^2}$$

El límite cuando $(x, y)$ tiende al origen depende del camino seguido. Si se toma la recta $y = x$, el límite es $\frac{1}{2}$, pero si se toma $y = 0$, el límite es 0. Como el resultado depende del camino, el límite no existe.

Esto demuestra que, en funciones de varias variables, la existencia del límite requiere que el valor sea independiente de la trayectoria de aproximación.

\subsection{Importancia de los Límites}

El estudio de límites es esencial porque permite definir otros conceptos fundamentales del cálculo. Una función se dice continua en un punto $a$ si el límite de la función en ese punto es igual al valor de la función:
\begin{equation}
\lim_{x \to a} f(x) = f(a)
\label{eq:continuidad}
\end{equation}

Asimismo, el límite es la base para definir la derivada, que representa la tasa de cambio instantánea de una función. La derivada de una función $f$ en un punto $a$ se define como el límite del cociente incremental cuando el incremento $h$ tiende a cero:
\begin{equation}
f'(a) = \lim_{h \to 0} \frac{f(a + h) - f(a)}{h}
\label{eq:derivada}
\end{equation}

De manera similar, en el cálculo integral, el límite se utiliza para expresar el proceso de acumulación continua de cantidades infinitesimales, lo que da origen a la integral definida.

\subsection{Límites al Infinito y Comportamiento Asintótico}

El límite también se aplica al estudio de infinitos y al comportamiento asintótico de las funciones. Por ejemplo, si una función $f(x)$ se aproxima indefinidamente a un valor constante $L$ cuando $x$ crece sin límite, se dice que tiene un límite finito al infinito:
$$\lim_{x \to \infty} f(x) = L$$

En cambio, si la función crece sin límite cuando $x$ tiende a cierto valor $a$, se dice que el límite es infinito:
$$\lim_{x \to a} f(x) = \infty$$

Estos conceptos permiten analizar el crecimiento, decrecimiento y comportamiento global de funciones en el análisis matemático.

\subsection{Síntesis}

En síntesis, el concepto de límite constituye el fundamento del cálculo moderno. Permite describir con precisión el comportamiento local de las funciones, estudiar su continuidad, definir la derivada y la integral, y analizar fenómenos de cambio continuo en la naturaleza. Su comprensión es esencial para cualquier rama de la ciencia que utilice modelos matemáticos, desde la física y la ingeniería hasta la economía y la biología.

% ==================== UNIDAD II ====================
\chapter{Unidad II: Derivadas Parciales}

Las derivadas parciales constituyen una de las herramientas fundamentales del cálculo multivariable, permitiendo analizar de manera precisa cómo una función de varias variables cambia respecto a cada una de sus variables independientes. En el contexto de la ingeniería moderna, el dominio de estos conceptos es esencial para modelar fenómenos complejos que dependen de múltiples parámetros simultáneamente.

Desde el análisis de transferencia de calor en estructuras hasta la optimización de procesos de producción, las derivadas parciales proporcionan el marco matemático necesario para comprender y resolver problemas del mundo real. Este capítulo explora los conceptos fundamentales de las derivadas parciales, sus propiedades matemáticas, las reglas de derivación aplicables y sus múltiples aplicaciones en ciencia e ingeniería.

\section{Derivadas Parciales de Primer Orden}

\subsection{Definición y Concepto}

Cuando una función depende de dos o más variables independientes, por ejemplo $x$ e $y$, el estudio de cómo cambia la función al variar solo una de esas variables conduce al concepto de derivada parcial. A diferencia del cálculo de una variable, donde existe una única dirección de cambio, en funciones multivariables es necesario analizar el cambio en múltiples direcciones.

Para una función $f(x, y)$ de dos variables, la derivada parcial con respecto a $x$ se define como el límite:
\begin{equation}
\frac{\partial f}{\partial x} = \lim_{h \to 0} \frac{f(x + h, y) - f(x, y)}{h}
\label{eq:derivada_parcial_x}
\end{equation}

Esta expresión representa la tasa de cambio instantánea de la función $f$ cuando la variable $x$ se modifica mientras la variable $y$ permanece constante. De manera análoga, la derivada parcial con respecto a $y$ se define como:
\begin{equation}
\frac{\partial f}{\partial y} = \lim_{h \to 0} \frac{f(x, y + h) - f(x, y)}{h}
\label{eq:derivada_parcial_y}
\end{equation}

\subsection{Interpretación Geométrica}

Geométricamente, una función de dos variables $f(x, y)$ puede representarse como una superficie en el espacio tridimensional. La derivada parcial $\frac{\partial f}{\partial x}$ en un punto $(x_0, y_0)$ representa la pendiente de la curva que resulta de la intersección de la superficie con el plano $y = y_0$. Esta curva muestra cómo varía la función cuando nos movemos en la dirección del eje $x$ manteniendo $y$ fijo.

De forma similar, $\frac{\partial f}{\partial y}$ representa la pendiente de la curva resultante de intersectar la superficie con el plano $x = x_0$, mostrando la variación en la dirección del eje $y$.

Esta interpretación es particularmente útil en ingeniería cuando se analizan superficies topográficas, distribuciones de temperatura en placas, o cualquier fenómeno físico que pueda modelarse como una superficie en el espacio.

\subsection{Notación}

Existen varias notaciones equivalentes para las derivadas parciales, cada una con ventajas particulares según el contexto:

\begin{itemize}
    \item Notación de Leibniz: $\frac{\partial f}{\partial x}$, $\frac{\partial f}{\partial y}$
    \item Notación con subíndices: $f_x$, $f_y$
    \item Notación con índices: $D_1 f$, $D_2 f$ (donde el índice indica la variable)
    \item Notación de gradiente: $\nabla f = \left(\frac{\partial f}{\partial x}, \frac{\partial f}{\partial y}\right)$
\end{itemize}

La notación de Leibniz con el símbolo $\partial$ (d redondeada) es la más común y ayuda a distinguir visualmente las derivadas parciales de las derivadas totales u ordinarias.

\subsection{Ejemplo 1: Cálculo de Derivadas Parciales de Primer Orden}

\textbf{Problema:} Dada la función $f(x, y) = x^3 y^2 + 2xy - 5y^3$, calcular las derivadas parciales de primer orden $\frac{\partial f}{\partial x}$ y $\frac{\partial f}{\partial y}$.

\textbf{Solución:}

\textit{Paso 1: Calcular $\frac{\partial f}{\partial x}$}

Para calcular la derivada parcial con respecto a $x$, tratamos a $y$ como una constante y derivamos término por término:

\begin{align}
f(x, y) &= x^3 y^2 + 2xy - 5y^3 \notag\\
\frac{\partial f}{\partial x} &= \frac{\partial}{\partial x}(x^3 y^2) + \frac{\partial}{\partial x}(2xy) - \frac{\partial}{\partial x}(5y^3) \notag
\end{align}

Aplicando las reglas de derivación:
\begin{itemize}
    \item Para $x^3 y^2$: tratamos $y^2$ como constante, entonces $\frac{\partial}{\partial x}(x^3 y^2) = 3x^2 y^2$
    \item Para $2xy$: tratamos $y$ como constante, entonces $\frac{\partial}{\partial x}(2xy) = 2y$
    \item Para $5y^3$: es constante respecto a $x$, entonces $\frac{\partial}{\partial x}(5y^3) = 0$
\end{itemize}

Por lo tanto:
\begin{equation}
\frac{\partial f}{\partial x} = 3x^2 y^2 + 2y
\label{eq:ejemplo1_fx}
\end{equation}

\textit{Paso 2: Calcular $\frac{\partial f}{\partial y}$}

Ahora derivamos con respecto a $y$, tratando a $x$ como constante:

\begin{align}
\frac{\partial f}{\partial y} &= \frac{\partial}{\partial y}(x^3 y^2) + \frac{\partial}{\partial y}(2xy) - \frac{\partial}{\partial y}(5y^3) \notag
\end{align}

Aplicando las reglas de derivación:
\begin{itemize}
    \item Para $x^3 y^2$: tratamos $x^3$ como constante, entonces $\frac{\partial}{\partial y}(x^3 y^2) = x^3 \cdot 2y = 2x^3 y$
    \item Para $2xy$: tratamos $x$ como constante, entonces $\frac{\partial}{\partial y}(2xy) = 2x$
    \item Para $5y^3$: aplicamos la regla de la potencia, $\frac{\partial}{\partial y}(5y^3) = 15y^2$
\end{itemize}

Por lo tanto:
\begin{equation}
\frac{\partial f}{\partial y} = 2x^3 y + 2x - 15y^2
\label{eq:ejemplo1_fy}
\end{equation}

\textbf{Verificación en un punto:} Si evaluamos en $(x, y) = (1, 2)$:
\begin{align}
\frac{\partial f}{\partial x}\bigg|_{(1,2)} &= 3(1)^2(2)^2 + 2(2) = 12 + 4 = 16 \notag\\
\frac{\partial f}{\partial y}\bigg|_{(1,2)} &= 2(1)^3(2) + 2(1) - 15(2)^2 = 4 + 2 - 60 = -54 \notag
\end{align}

Esto indica que en el punto $(1, 2)$, la función aumenta a razón de 16 unidades por unidad de cambio en $x$, y disminuye a razón de 54 unidades por unidad de cambio en $y$.

\section{Derivadas Parciales de Segundo Orden}

\subsection{Definición y Tipos}

Las derivadas parciales de segundo orden resultan de aplicar nuevamente la operación de derivación parcial a una derivada parcial de primer orden. Para una función de dos variables $f(x, y)$, existen cuatro posibles derivadas parciales de segundo orden:

\begin{itemize}
    \item \textbf{Derivada segunda respecto a $x$:} 
    \begin{equation}
    \frac{\partial^2 f}{\partial x^2} = \frac{\partial}{\partial x}\left(\frac{\partial f}{\partial x}\right) = f_{xx}
    \label{eq:derivada_segunda_xx}
    \end{equation}
    
    \item \textbf{Derivada segunda respecto a $y$:}
    \begin{equation}
    \frac{\partial^2 f}{\partial y^2} = \frac{\partial}{\partial y}\left(\frac{\partial f}{\partial y}\right) = f_{yy}
    \label{eq:derivada_segunda_yy}
    \end{equation}
    
    \item \textbf{Derivada mixta (primero $x$, luego $y$):}
    \begin{equation}
    \frac{\partial^2 f}{\partial y \partial x} = \frac{\partial}{\partial y}\left(\frac{\partial f}{\partial x}\right) = f_{xy}
    \label{eq:derivada_mixta_xy}
    \end{equation}
    
    \item \textbf{Derivada mixta (primero $y$, luego $x$):}
    \begin{equation}
    \frac{\partial^2 f}{\partial x \partial y} = \frac{\partial}{\partial x}\left(\frac{\partial f}{\partial y}\right) = f_{yx}
    \label{eq:derivada_mixta_yx}
    \end{equation}
\end{itemize}

\subsection{Teorema de Schwarz (Igualdad de las Derivadas Mixtas)}

Una propiedad fundamental de las derivadas parciales de segundo orden es que, bajo condiciones adecuadas de regularidad, las derivadas mixtas son iguales independientemente del orden de derivación. Este resultado se conoce como el Teorema de Schwarz o Teorema de Clairaut.

\textbf{Teorema:} Si $f$ es una función de dos variables y las derivadas parciales $f_{xy}$ y $f_{yx}$ son continuas en un conjunto abierto, entonces:
\begin{equation}
\frac{\partial^2 f}{\partial x \partial y} = \frac{\partial^2 f}{\partial y \partial x}
\label{eq:teorema_schwarz}
\end{equation}

Este teorema simplifica significativamente el cálculo de derivadas parciales de orden superior, ya que garantiza que para funciones suficientemente suaves, solo necesitamos calcular tres derivadas de segundo orden distintas en lugar de cuatro.

\subsection{Interpretación y Aplicaciones}

Las derivadas de segundo orden proporcionan información sobre la curvatura de la superficie representada por la función:

\begin{itemize}
    \item $\frac{\partial^2 f}{\partial x^2}$ mide la curvatura en la dirección $x$
    \item $\frac{\partial^2 f}{\partial y^2}$ mide la curvatura en la dirección $y$
    \item Las derivadas mixtas miden la torsión de la superficie
\end{itemize}

Esta información es crucial para el análisis de extremos (máximos, mínimos y puntos de silla) mediante el criterio de la segunda derivada, que involucra la matriz Hessiana:

\begin{equation}
H = \begin{pmatrix}
\frac{\partial^2 f}{\partial x^2} & \frac{\partial^2 f}{\partial x \partial y} \\
\frac{\partial^2 f}{\partial y \partial x} & \frac{\partial^2 f}{\partial y^2}
\end{pmatrix}
\label{eq:matriz_hessiana}
\end{equation}

\subsection{Ejemplo 2: Cálculo de Derivadas Parciales de Segundo Orden}

\textbf{Problema:} Para la función $f(x, y) = e^{xy} + x^2 \sin(y)$, calcular todas las derivadas parciales de segundo orden y verificar el Teorema de Schwarz.

\textbf{Solución:}

\textit{Paso 1: Calcular las derivadas parciales de primer orden}

\begin{align}
\frac{\partial f}{\partial x} &= \frac{\partial}{\partial x}(e^{xy}) + \frac{\partial}{\partial x}(x^2 \sin(y)) \notag\\
&= y e^{xy} + 2x \sin(y)
\label{eq:ejemplo2_fx}
\end{align}

\begin{align}
\frac{\partial f}{\partial y} &= \frac{\partial}{\partial y}(e^{xy}) + \frac{\partial}{\partial y}(x^2 \sin(y)) \notag\\
&= x e^{xy} + x^2 \cos(y)
\label{eq:ejemplo2_fy}
\end{align}

\textit{Paso 2: Calcular $\frac{\partial^2 f}{\partial x^2}$}

Derivamos $\frac{\partial f}{\partial x} = y e^{xy} + 2x \sin(y)$ con respecto a $x$:

\begin{align}
\frac{\partial^2 f}{\partial x^2} &= \frac{\partial}{\partial x}(y e^{xy}) + \frac{\partial}{\partial x}(2x \sin(y)) \notag\\
&= y \cdot y e^{xy} + 2\sin(y) \notag\\
&= y^2 e^{xy} + 2\sin(y)
\label{eq:ejemplo2_fxx}
\end{align}

\textit{Paso 3: Calcular $\frac{\partial^2 f}{\partial y^2}$}

Derivamos $\frac{\partial f}{\partial y} = x e^{xy} + x^2 \cos(y)$ con respecto a $y$:

\begin{align}
\frac{\partial^2 f}{\partial y^2} &= \frac{\partial}{\partial y}(x e^{xy}) + \frac{\partial}{\partial y}(x^2 \cos(y)) \notag\\
&= x \cdot x e^{xy} + x^2 \cdot (-\sin(y)) \notag\\
&= x^2 e^{xy} - x^2 \sin(y)
\label{eq:ejemplo2_fyy}
\end{align}

\textit{Paso 4: Calcular $\frac{\partial^2 f}{\partial y \partial x}$}

Derivamos $\frac{\partial f}{\partial x} = y e^{xy} + 2x \sin(y)$ con respecto a $y$:

\begin{align}
\frac{\partial^2 f}{\partial y \partial x} &= \frac{\partial}{\partial y}(y e^{xy}) + \frac{\partial}{\partial y}(2x \sin(y)) \notag\\
&= e^{xy} + y \cdot x e^{xy} + 2x \cos(y) \notag\\
&= e^{xy} + xy e^{xy} + 2x \cos(y) \notag\\
&= e^{xy}(1 + xy) + 2x \cos(y)
\label{eq:ejemplo2_fxy}
\end{align}

\textit{Paso 5: Calcular $\frac{\partial^2 f}{\partial x \partial y}$}

Derivamos $\frac{\partial f}{\partial y} = x e^{xy} + x^2 \cos(y)$ con respecto a $x$:

\begin{align}
\frac{\partial^2 f}{\partial x \partial y} &= \frac{\partial}{\partial x}(x e^{xy}) + \frac{\partial}{\partial x}(x^2 \cos(y)) \notag\\
&= e^{xy} + x \cdot y e^{xy} + 2x \cos(y) \notag\\
&= e^{xy} + xy e^{xy} + 2x \cos(y) \notag\\
&= e^{xy}(1 + xy) + 2x \cos(y)
\label{eq:ejemplo2_fyx}
\end{align}

\textit{Paso 6: Verificación del Teorema de Schwarz}

Observamos que:
\begin{equation}
\frac{\partial^2 f}{\partial y \partial x} = \frac{\partial^2 f}{\partial x \partial y} = e^{xy}(1 + xy) + 2x \cos(y)
\label{eq:ejemplo2_verificacion}
\end{equation}

Esto confirma el Teorema de Schwarz para esta función, ya que ambas derivadas mixtas son idénticas. Esta igualdad es válida porque las derivadas parciales de $f$ son continuas en todo $\mathbb{R}^2$.

\section{Reglas de Derivación}

Al calcular derivadas parciales, se aplican las mismas reglas fundamentales del cálculo de una variable, con la importante consideración de que todas las variables excepto aquella respecto a la cual se deriva deben tratarse como constantes.

\subsection{Regla de la Potencia}

Para una función que contiene una potencia de una variable, la regla de la potencia se aplica de la siguiente manera:

Si $f(x, y) = x^n g(y)$, donde $n$ es una constante y $g(y)$ es una función de $y$, entonces:
\begin{equation}
\frac{\partial f}{\partial x} = n x^{n-1} g(y)
\label{eq:regla_potencia}
\end{equation}

Al derivar respecto a $x$, $g(y)$ se trata como una constante multiplicativa. De manera análoga, al derivar respecto a $y$, $x^n$ actúa como constante.

\textbf{Ejemplo:} Para $f(x, y) = x^4 y^3$:
\begin{align}
\frac{\partial f}{\partial x} &= 4x^3 y^3 \quad \text{(tratando } y^3 \text{ como constante)} \notag\\
\frac{\partial f}{\partial y} &= x^4 \cdot 3y^2 = 3x^4 y^2 \quad \text{(tratando } x^4 \text{ como constante)} \notag
\end{align}

\subsection{Regla del Producto}

Cuando una función es el producto de dos funciones que dependen de la variable de derivación, se aplica la regla del producto:

Si $f(x, y) = u(x, y) \cdot v(x, y)$, entonces:
\begin{equation}
\frac{\partial f}{\partial x} = \frac{\partial u}{\partial x} \cdot v + u \cdot \frac{\partial v}{\partial x}
\label{eq:regla_producto}
\end{equation}

\textbf{Ejemplo:} Para $f(x, y) = x^2 e^y$:
\begin{align}
\frac{\partial f}{\partial x} &= 2x \cdot e^y + x^2 \cdot 0 = 2x e^y \notag\\
\frac{\partial f}{\partial y} &= 0 \cdot e^y + x^2 \cdot e^y = x^2 e^y \notag
\end{align}

Note que al derivar $e^y$ respecto a $x$, el resultado es cero porque $e^y$ es constante respecto a $x$.

\subsection{Regla de la Cadena}

La regla de la cadena es fundamental cuando se trabaja con composiciones de funciones. Para funciones de varias variables, la regla de la cadena toma varias formas según el contexto.

\textbf{Caso 1: Composición con una función de una variable}

Si $z = f(g(x, y))$, donde $g$ es una función de $x$ e $y$, y $f$ es una función de una variable, entonces:
\begin{equation}
\frac{\partial z}{\partial x} = f'(g(x, y)) \cdot \frac{\partial g}{\partial x}
\label{eq:regla_cadena_caso1}
\end{equation}

\textbf{Ejemplo:} Para $f(x, y) = (x^2 + y^2)^3$:

Aquí $g(x, y) = x^2 + y^2$ y la función exterior es $u^3$. Aplicando la regla de la cadena:
\begin{align}
\frac{\partial f}{\partial x} &= 3(x^2 + y^2)^2 \cdot \frac{\partial}{\partial x}(x^2 + y^2) \notag\\
&= 3(x^2 + y^2)^2 \cdot 2x = 6x(x^2 + y^2)^2 \notag
\end{align}

\textbf{Caso 2: Regla de la cadena general}

Si $z = f(u, v)$ donde $u = u(x, y)$ y $v = v(x, y)$, entonces:
\begin{equation}
\frac{\partial z}{\partial x} = \frac{\partial f}{\partial u} \cdot \frac{\partial u}{\partial x} + \frac{\partial f}{\partial v} \cdot \frac{\partial v}{\partial x}
\label{eq:regla_cadena_general}
\end{equation}

Esta forma de la regla de la cadena es especialmente importante en cambios de coordenadas y en el cálculo de derivadas direccionales.

\subsection{Regla del Cociente}

Para el cociente de dos funciones $f(x, y) = \frac{u(x, y)}{v(x, y)}$, donde $v(x, y) \neq 0$:
\begin{equation}
\frac{\partial f}{\partial x} = \frac{\frac{\partial u}{\partial x} \cdot v - u \cdot \frac{\partial v}{\partial x}}{v^2}
\label{eq:regla_cociente}
\end{equation}

\textbf{Ejemplo:} Para $f(x, y) = \frac{x^2}{x + y}$:
\begin{align}
\frac{\partial f}{\partial x} &= \frac{2x(x + y) - x^2 \cdot 1}{(x + y)^2} = \frac{2x^2 + 2xy - x^2}{(x + y)^2} = \frac{x^2 + 2xy}{(x + y)^2} \notag
\end{align}

\section{Historia de las Derivadas Parciales}

\subsection{Orígenes del Cálculo Multivariable}

El desarrollo histórico de las derivadas parciales está íntimamente ligado a la evolución del cálculo y de la física matemática. Aunque el cálculo diferencial fue desarrollado en el siglo XVII por Isaac Newton y Gottfried Wilhelm Leibniz, sus trabajos se enfocaron principalmente en funciones de una sola variable. La necesidad de estudiar funciones de múltiples variables surgió naturalmente de problemas físicos y geométricos que no podían ser resueltos con las herramientas del cálculo univariable.

\subsection{Desarrollo en el Siglo XVIII}

Durante el siglo XVIII, varios matemáticos comenzaron a trabajar con funciones de múltiples variables sin tener aún una notación o teoría formal completamente desarrollada:

\textbf{Jean Le Rond d'Alembert (1746):} En su trabajo sobre la vibración de cuerdas, d'Alembert derivó lo que ahora conocemos como la ecuación de onda, una ecuación diferencial parcial que involucra derivadas respecto al tiempo y al espacio. Este fue uno de los primeros usos explícitos de derivadas parciales en física matemática.

\textbf{Leonhard Euler (1755):} Euler desarrolló métodos para trabajar con funciones de varias variables y fue uno de los primeros en usar notación específica para distinguir entre diferentes tipos de derivadas, aunque la notación moderna aún no existía.

\textbf{Joseph-Louis Lagrange (1788):} Introdujo el cálculo de variaciones de manera sistemática, trabajando extensivamente con funciones de múltiples variables y sus derivadas.

\subsection{Consolidación de la Notación}

La notación moderna para derivadas parciales tiene su propia historia fascinante:

\textbf{Marquis de Condorcet (1770):} Fue uno de los primeros en usar el símbolo $\partial$ para denotar diferencias parciales, aunque su uso no fue consistente ni ampliamente adoptado en ese momento.

\textbf{Adrien-Marie Legendre (1786):} Adoptó brevemente el símbolo $\partial$, pero posteriormente lo abandonó en favor de otras notaciones.

\textbf{Carl Gustav Jacob Jacobi (1841):} Reintrodujo y estandarizó el uso del símbolo $\partial$ para derivadas parciales. Jacobi también contribuyó significativamente a la teoría con su trabajo sobre determinantes jacobianos y transformaciones de coordenadas.

\subsection{Formalización en el Siglo XIX}

El siglo XIX vio la formalización rigurosa del análisis matemático:

\textbf{Augustin-Louis Cauchy (1820s):} Proporcionó las primeras definiciones rigurosas de límites y continuidad, estableciendo las bases para una teoría formal de derivadas parciales.

\textbf{Karl Weierstrass (1860s):} Desarrolló una teoría aún más rigurosa del análisis, incluyendo condiciones precisas para la existencia y continuidad de derivadas parciales.

\textbf{Bernhard Riemann (1850s):} Sus trabajos en geometría diferencial y análisis complejo ampliaron enormemente el alcance de las aplicaciones de derivadas parciales.

\subsection{Siglo XX y Aplicaciones Modernas}

En el siglo XX, la teoría de derivadas parciales se convirtió en una rama central de las matemáticas, con desarrollos en:

\begin{itemize}
    \item Teoría de ecuaciones en derivadas parciales (EDP)
    \item Análisis funcional y espacios de Hilbert
    \item Métodos numéricos y computacionales
    \item Aplicaciones en mecánica cuántica, relatividad y teoría de campos
    \item Optimización y teoría de control
\end{itemize}

Hoy en día, las derivadas parciales son fundamentales en prácticamente todas las áreas de la ciencia, ingeniería, economía y ciencia de datos, representando una de las herramientas matemáticas más versátiles y poderosas disponibles.

\section{Aplicaciones}

Las derivadas parciales encuentran aplicación en una amplia variedad de campos, desde las ciencias físicas hasta las ciencias sociales y la tecnología moderna. A continuación se presentan algunas de las aplicaciones más relevantes.

\subsection{Física e Ingeniería}

\subsubsection{Transferencia de Calor}

La ecuación del calor es una de las ecuaciones en derivadas parciales más importantes en física e ingeniería:
\begin{equation}
\frac{\partial T}{\partial t} = \alpha \left(\frac{\partial^2 T}{\partial x^2} + \frac{\partial^2 T}{\partial y^2} + \frac{\partial^2 T}{\partial z^2}\right)
\label{eq:ecuacion_calor}
\end{equation}

donde $T(x, y, z, t)$ es la temperatura, $t$ es el tiempo, y $\alpha$ es la difusividad térmica del material. Esta ecuación describe cómo se distribuye el calor en un cuerpo sólido a lo largo del tiempo. Las derivadas parciales espaciales ($\frac{\partial^2 T}{\partial x^2}$, etc.) describen la curvatura de la distribución de temperatura, mientras que $\frac{\partial T}{\partial t}$ representa la tasa de cambio temporal.

En ingeniería, esta ecuación se utiliza para diseñar sistemas de enfriamiento, analizar aislamiento térmico en edificios, y optimizar procesos industriales que involucran transferencia de calor.

\subsubsection{Mecánica de Fluidos}

Las ecuaciones de Navier-Stokes, que gobiernan el movimiento de fluidos, son sistemas complejos de ecuaciones en derivadas parciales:
\begin{equation}
\rho \left(\frac{\partial \mathbf{v}}{\partial t} + (\mathbf{v} \cdot \nabla)\mathbf{v}\right) = -\nabla p + \mu \nabla^2 \mathbf{v} + \mathbf{f}
\label{eq:navier_stokes}
\end{equation}

donde $\mathbf{v}$ es el campo de velocidad, $p$ es la presión, $\rho$ es la densidad, $\mu$ es la viscosidad dinámica, y $\mathbf{f}$ representa las fuerzas externas. Estas ecuaciones son fundamentales para el diseño de aeronaves, turbinas, sistemas de tuberías, y predicción meteorológica.

\subsubsection{Electromagnetismo}

Las ecuaciones de Maxwell, que describen todos los fenómenos electromagnéticos, están formuladas en términos de derivadas parciales:
\begin{align}
\nabla \cdot \mathbf{E} &= \frac{\rho}{\epsilon_0} \label{eq:maxwell1}\\
\nabla \cdot \mathbf{B} &= 0 \label{eq:maxwell2}\\
\nabla \times \mathbf{E} &= -\frac{\partial \mathbf{B}}{\partial t} \label{eq:maxwell3}\\
\nabla \times \mathbf{B} &= \mu_0 \mathbf{J} + \mu_0 \epsilon_0 \frac{\partial \mathbf{E}}{\partial t} \label{eq:maxwell4}
\end{align}

donde $\mathbf{E}$ es el campo eléctrico, $\mathbf{B}$ es el campo magnético, $\rho$ es la densidad de carga, y $\mathbf{J}$ es la densidad de corriente. Estas ecuaciones son la base del diseño de antenas, circuitos de alta frecuencia, dispositivos de comunicación inalámbrica y tecnología de radares.

\subsection{Geometría y Topografía}

En geometría diferencial, las derivadas parciales se utilizan para calcular propiedades de superficies como la curvatura y las normales. Para una superficie parametrizada por $\mathbf{r}(u, v)$, el vector normal se calcula mediante:
\begin{equation}
\mathbf{n} = \frac{\partial \mathbf{r}}{\partial u} \times \frac{\partial \mathbf{r}}{\partial v}
\label{eq:vector_normal}
\end{equation}

En topografía y cartografía, las curvas de nivel que estudiamos anteriormente se definen implícitamente mediante derivadas parciales. La pendiente del terreno en cualquier dirección se puede calcular usando el gradiente:
\begin{equation}
\nabla h = \left(\frac{\partial h}{\partial x}, \frac{\partial h}{\partial y}\right)
\label{eq:gradiente_altura}
\end{equation}

donde $h(x, y)$ representa la altitud en cada punto.

\subsection{Economía y Finanzas}

\subsubsection{Teoría del Consumidor}

En economía, la función de utilidad $U(x_1, x_2, \ldots, x_n)$ representa la satisfacción de un consumidor al consumir cantidades $x_i$ de diferentes bienes. Las derivadas parciales de la función de utilidad se conocen como utilidades marginales:
\begin{equation}
\frac{\partial U}{\partial x_i} = \text{Utilidad marginal del bien } i
\label{eq:utilidad_marginal}
\end{equation}

La tasa marginal de sustitución entre dos bienes se define como el cociente de sus utilidades marginales:
\begin{equation}
\text{TMS}_{12} = -\frac{\partial U / \partial x_1}{\partial U / \partial x_2}
\label{eq:tms}
\end{equation}

Este concepto es fundamental para entender las decisiones de consumo y la formación de precios en mercados competitivos.

\subsubsection{Ecuación de Black-Scholes}

En finanzas cuantitativas, la valoración de opciones se basa en la ecuación de Black-Scholes, una ecuación en derivadas parciales:
\begin{equation}
\frac{\partial V}{\partial t} + \frac{1}{2}\sigma^2 S^2 \frac{\partial^2 V}{\partial S^2} + rS\frac{\partial V}{\partial S} - rV = 0
\label{eq:black_scholes}
\end{equation}

donde $V(S, t)$ es el precio de la opción, $S$ es el precio del activo subyacente, $\sigma$ es la volatilidad, y $r$ es la tasa de interés libre de riesgo. Las derivadas parciales en esta ecuación tienen interpretaciones financieras específicas conocidas como "las griegas": Delta ($\frac{\partial V}{\partial S}$), Gamma ($\frac{\partial^2 V}{\partial S^2}$), Theta ($\frac{\partial V}{\partial t}$), entre otras.

\subsection{Inteligencia Artificial y Machine Learning}

\subsubsection{Redes Neuronales y Retropropagación}

En el entrenamiento de redes neuronales, el algoritmo de retropropagación (backpropagation) se basa fundamentalmente en el cálculo de derivadas parciales mediante la regla de la cadena. Para una función de pérdida $L(\mathbf{w})$ que depende de los pesos $\mathbf{w}$ de la red, el gradiente:
\begin{equation}
\nabla L = \left(\frac{\partial L}{\partial w_1}, \frac{\partial L}{\partial w_2}, \ldots, \frac{\partial L}{\partial w_n}\right)
\label{eq:gradiente_perdida}
\end{equation}

indica la dirección de mayor crecimiento de la función de pérdida. El algoritmo de descenso del gradiente utiliza esta información para actualizar los pesos:
\begin{equation}
w_i^{(k+1)} = w_i^{(k)} - \eta \frac{\partial L}{\partial w_i}
\label{eq:descenso_gradiente}
\end{equation}

donde $\eta$ es la tasa de aprendizaje. Este proceso iterativo minimiza la función de pérdida, mejorando el rendimiento del modelo.

\subsubsection{Optimización}

En problemas de optimización multivariable, las derivadas parciales son esenciales para encontrar puntos críticos. Para encontrar los extremos de una función $f(x, y)$, se resuelve el sistema:
\begin{align}
\frac{\partial f}{\partial x} &= 0 \label{eq:extremo_x}\\
\frac{\partial f}{\partial y} &= 0 \label{eq:extremo_y}
\end{align}

La naturaleza del punto crítico (máximo, mínimo o punto de silla) se determina mediante el análisis de la matriz Hessiana y el cálculo del determinante:
\begin{equation}
D = \frac{\partial^2 f}{\partial x^2} \cdot \frac{\partial^2 f}{\partial y^2} - \left(\frac{\partial^2 f}{\partial x \partial y}\right)^2
\label{eq:determinante_hessiano}
\end{equation}

Este criterio se utiliza ampliamente en algoritmos de optimización para problemas de regresión, clasificación, y aprendizaje por refuerzo.

\subsection{Visión por Computadora}

En procesamiento de imágenes, las derivadas parciales se utilizan para detectar bordes y características. El operador de Sobel, por ejemplo, aproxima el gradiente de la intensidad de la imagen:
\begin{equation}
\nabla I = \left(\frac{\partial I}{\partial x}, \frac{\partial I}{\partial y}\right)
\label{eq:gradiente_imagen}
\end{equation}

La magnitud del gradiente $|\nabla I|$ es alta en los bordes de los objetos, lo que permite su detección automática. Esta técnica es fundamental en reconocimiento de patrones, seguimiento de objetos y reconstrucción 3D.

\subsection{Biología y Medicina}

En biología matemática, las derivadas parciales modelan fenómenos como la difusión de sustancias químicas, el crecimiento de poblaciones en medios heterogéneos, y la propagación de enfermedades. La ecuación de reacción-difusión de Fisher-KPP:
\begin{equation}
\frac{\partial u}{\partial t} = D\frac{\partial^2 u}{\partial x^2} + ru(1 - u)
\label{eq:fisher_kpp}
\end{equation}

describe la dispersión espacial de una población con crecimiento logístico, donde $u(x, t)$ es la densidad de población, $D$ es el coeficiente de difusión, y $r$ es la tasa de crecimiento.

En imagenología médica, las derivadas parciales se utilizan en algoritmos de reconstrucción tomográfica, análisis de resonancias magnéticas, y modelado de flujo sanguíneo.

\subsection{Síntesis}

Las aplicaciones de las derivadas parciales son prácticamente ilimitadas en la ciencia y la ingeniería modernas. Desde la predicción del clima hasta el diseño de chips de computadora, desde la optimización de carteras de inversión hasta el desarrollo de fármacos, las derivadas parciales proporcionan el lenguaje matemático necesario para describir, analizar y predecir el comportamiento de sistemas complejos que dependen de múltiples variables. Su dominio es esencial para cualquier profesional que trabaje con modelos matemáticos del mundo real.

% ==================== UNIDAD III (PLACEHOLDER) ====================
\chapter{Unidad III}

\textit{[Aquí se incluirá el contenido de la Unidad III una vez completada la investigación]}

% ==================== REFERENCIAS ====================
\begin{thebibliography}{99}

\bibitem{wolfram2025}
Wolfram MathWorld. (2025). \textit{Function}. Recuperado el 10 de octubre de 2025, de \url{https://mathworld.wolfram.com/Function.html}

\bibitem{stewart2016}
Stewart, J. (2016). \textit{Cálculo de una variable: Trascendentes tempranas} (7a ed.). Cengage Learning.

\bibitem{niven2006}
Niven, I., \& Zuckerman, H. S. (2006). \textit{Mathematics: A practical introduction} (2nd ed.). Wiley.

\bibitem{wikipedia_dominio}
Wikipedia. (2025). Dominio de una función. Recuperado el 10 de octubre de 2025, de \url{https://es.wikipedia.org/wiki/Dominio_de_una_función}

\bibitem{wikipedia_rango}
Wikipedia. (2025). Rango (matemáticas). Recuperado el 10 de octubre de 2025, de \url{https://es.wikipedia.org/wiki/Rango_(matemáticas)}

\bibitem{disfruta2025}
Disfruta las matemáticas. (2025). Dominio, codominio y rango. Recuperado el 10 de octubre de 2025, de \url{https://www.disfrutalasmatematicas.com/conjuntos/dominio-rango-codominio.html}

\bibitem{libretexts2022}
LibreTexts. (2022). 4.7: Dominio y rango de una función. Recuperado el 10 de octubre de 2025, de \url{https://espanol.libretexts.org/Bookshelves/Matematicas/Matematicas_Aplicadas/Calculo_para_Negocios_y_Ciencias_Sociales_Corequisite_Workbook_(Dominguez_Martinez_y_Saykali)/04:_Funciones/4.07:_Dominio_y_rango_de_una_función}

\bibitem{larson2017}
Larson, R., \& Edwards, B. H. (2017). \textit{Cálculo: Trascendentes tempranas} (9ª ed.). Cengage Learning.

\bibitem{wikipedia_implicita}
Wikipedia. (2025). Función implícita. En \textit{Wikipedia: La enciclopedia libre}. Recuperado el 15 de octubre de 2025, de \url{https://es.wikipedia.org/wiki/Función_implícita}

\bibitem{libretexts_implicita}
LibreTexts. (2023). Funciones explícitas e implícitas en cálculo. Recuperado el 15 de octubre de 2025, de \url{https://espanol.libretexts.org/Bookshelves/Matematicas/Calculo}

\bibitem{khan_academy}
Khan Academy. (2025). Funciones explícitas e implícitas. Recuperado el 15 de octubre de 2025, de \url{https://es.khanacademy.org/math/differential-calculus}

\bibitem{wikipedia_curvas}
Wikipedia. (2025). Curva de nivel. En \textit{Wikipedia: La enciclopedia libre}. Recuperado el 15 de octubre de 2025, de \url{https://es.wikipedia.org/wiki/Curva_de_nivel}

\bibitem{wikipedia_limite}
Wikipedia. (2025). Límite (matemáticas). En \textit{Wikipedia: La enciclopedia libre}. Recuperado el 15 de octubre de 2025, de \url{https://es.wikipedia.org/wiki/Límite_(matemáticas)}

\bibitem{thomas2018}
Thomas, G. B., Weir, M. D., Hass, J., \& Giordano, F. R. (2018). \textit{Thomas' Calculus} (14ª ed.). Pearson Education.

\end{thebibliography}

\end{document}
