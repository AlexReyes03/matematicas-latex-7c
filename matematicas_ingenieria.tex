\documentclass[12pt,letterpaper]{report}

% ==================== PAQUETES ====================
\usepackage[spanish,es-tabla]{babel}
\usepackage[utf8]{inputenc}
\usepackage[T1]{fontenc}
\usepackage{amsmath,amssymb,amsthm}
\usepackage{graphicx}
\usepackage[left=2.5cm,right=2.5cm,top=3cm,bottom=3cm]{geometry}
\usepackage{setspace}
\usepackage{fancyhdr}
\usepackage{titlesec}
\usepackage{tocloft}
\usepackage[hidelinks]{hyperref}
\usepackage{float}
\usepackage{caption}
\usepackage{subcaption}
\usepackage{enumitem}
\usepackage{csquotes}

% ==================== CONFIGURACIONES ====================
% Interlineado
\onehalfspacing

% Encabezados y pies de página
\pagestyle{fancy}
\fancyhf{}
\fancyhead[L]{\leftmark}
\fancyhead[R]{\thepage}
\renewcommand{\headrulewidth}{0.5pt}

% Formato de capítulos y secciones
\titleformat{\chapter}[display]
  {\normalfont\huge\bfseries}{\chaptertitlename\ \thechapter}{20pt}{\Huge}
\titlespacing*{\chapter}{0pt}{0pt}{40pt}

% Configuración de índices
\renewcommand{\contentsname}{Índice}
\renewcommand{\listfigurename}{Índice de Figuras}
\renewcommand{\listtablename}{Índice de Tablas}
\renewcommand{\bibname}{Referencias}

% Numeración de ecuaciones por capítulo
\numberwithin{equation}{chapter}
\numberwithin{figure}{chapter}
\numberwithin{table}{chapter}

% ==================== INFORMACIÓN DEL DOCUMENTO ====================
\title{Investigación de Matemáticas para la Ingeniería I}
\author{
    Aguilar García Angel Gabriel \\
    Bahena González Jafet \\
    Carrate Bahena Daniela \\
    Carrillo Juárez Loreley \\
    Reyes Vargas Isael Alejandro \\
    Sánchez Martínez Danna Paola
    
    
}
\date{\today}

% ==================== INICIO DEL DOCUMENTO ====================
\begin{document}

% ==================== PORTADA ====================
\begin{titlepage}
    \centering

    % Logos en la parte superior
    \begin{minipage}{0.4\textwidth}
        \centering
        \includegraphics[width=0.7\textwidth]{img/logo-utez.png}
    \end{minipage}
    \hfill
    \begin{minipage}{0.4\textwidth}
        \centering
        \includegraphics[width=0.7\textwidth]{img/logo-datid.png}
    \end{minipage}

    \vspace{1cm}

    {\Large\textbf{UNIVERSIDAD TECNOLÓGICA EMILIANO ZAPATA}}\\[0.5cm]
    {\large División Académica de Tecnologías de la Información y Diseño}\\[0.5cm]
    {\large Ingeniería en Desarrollo y Gestión de Software}\\[2cm]
    
    \rule{\linewidth}{0.5mm}\\[0.4cm]
    {\huge\bfseries Integradora de Matemáticas\\[0.2cm]para la Ingeniería I}\\[0.2cm]
    \rule{\linewidth}{0.5mm}\\[2cm]
    
    {\Large\textbf{Proyecto Integrador}}\\[1.5cm]
    
    \begin{minipage}{0.8\textwidth}
        \begin{flushleft}
            \textbf{Materia:}\\
            Matemáticas para la Ingeniería I\\[0.5cm]
            
            \textbf{Profesor:}\\
            M.C. Jorge Yusef Colin Castillo\\[0.5cm]
            
            \textbf{Grupo:} 7 C IDyGS\\[0.5cm]
            
            \textbf{Integrantes del Equipo:}\\
            \begin{itemize}[leftmargin=1cm]
                \item Aguilar García Angel Gabriel
                \item Bahena González Jafet
                \item Carrate Bahena Daniela
                \item Carrillo Juárez Loreley
                \item Reyes Vargas Isael Alejandro
                \item Sánchez Martínez Danna Paola
            \end{itemize}
        \end{flushleft}
    \end{minipage}
    
    \vfill
    
    {\large Emiliano Zapata, Morelos, México}\\
    {\large \today}
\end{titlepage}

% ==================== ÍNDICES ====================
\tableofcontents
\cleardoublepage 

\listoffigures
\cleardoublepage 

% ==================== UNIDAD I ====================
\chapter{Unidad I: Funciones y Límites}

\section{Funciones}

La definición formal moderna de una función proviene de la teoría de conjuntos y está basada en la relación entre dos conjuntos. Sin embargo, la noción de función no siempre fue tan precisa. Históricamente, los matemáticos comenzaron a tratar con funciones sin formalizar completamente su definición.

La función se puede conceptualizar como una relación entre dos conjuntos, en la cual a cada elemento del primer conjunto (dominio) se le asigna un único elemento del segundo conjunto (contradominio).

En términos matemáticos, una función se puede definir como:
\begin{equation}
f: A \rightarrow B
\label{eq:definicion_funcion}
\end{equation}

Donde:
\begin{itemize}
    \item $A$ es el dominio de la función, que es el conjunto de todos los valores posibles de entrada.
    \item $B$ es el codominio, que es el conjunto de posibles valores de salida.
    \item Para cada $x \in A$, existe un único valor $f(x) \in B$.
\end{itemize}

\subsection{Historia}

El concepto de función comenzó a desarrollarse en el siglo XVII con los trabajos de matemáticos como René Descartes e Isaac Newton, pero su formalización no llegó hasta el siglo XIX. En la antigua Grecia, los matemáticos utilizaban nociones de relación entre cantidades, pero no se hablaba de funciones como una entidad matemática en sí misma.

\textbf{Leibniz (1680s):} Introdujo el concepto de una función en su trabajo sobre cálculo, lo que llevó a la adopción de la notación moderna que ahora usamos.

\textbf{Euler (1740s):} Definió funciones de manera más general y comenzó a utilizar la notación $f(x)$ para denotar el valor de la función $f$ para un valor dado de $x$.

\textbf{Cauchy y Weierstrass (Siglo XIX):} Formalizaron el concepto de función, especialmente con la definición de funciones continuas y diferenciables.

\subsection{Propiedades Fundamentales de las Funciones}

\subsubsection{Unicidad}

Una de las propiedades clave de las funciones es que a cada valor del dominio debe corresponder un único valor en el contradominio. Esta propiedad asegura que no haya ambigüedades en la asignación de valores.

\textit{Ejemplo:} En la función cuadrática $f(x) = x^2$, para $x = 2$, la imagen es 4, y no puede ser otro valor.

\subsubsection{Existencia y Unicidad de la Función}

Cuando decimos que una función está bien definida, esto significa que para cada $x$ en el dominio, existe un único valor $f(x)$ asociado. Este valor es único para ese $x$.

\subsubsection{Imagen y Pre-imagen}

\begin{itemize}
    \item La \textbf{imagen} de la función es el conjunto de todos los valores que puede tomar la función $f(x)$ a medida que $x$ recorre todo el dominio.
    \item La \textbf{pre-imagen} de un valor $y$ es el conjunto de todos los $x$ tal que $f(x) = y$.
\end{itemize}

\subsection{Aplicaciones de las Funciones}

Las funciones tienen una amplia variedad de aplicaciones en diferentes campos. Algunas de las más comunes son:

\begin{itemize}
    \item \textbf{Cálculo Diferencial e Integral:} Las funciones son esenciales para entender el cambio y el área bajo las curvas.
    \item \textbf{Ingeniería:} Las funciones se utilizan para modelar fenómenos físicos, como el comportamiento de circuitos eléctricos o el movimiento de partículas.
    \item \textbf{Economía:} Se utilizan para modelar la oferta y demanda, el costo de producción, y las tasas de crecimiento.
    \item \textbf{Biología:} Las funciones modelan el crecimiento poblacional, la dispersión de enfermedades y la dinámica de ecosistemas.
\end{itemize}

\subsection{Ejercicios: Identificar Variables Dependientes e Independientes}

A continuación se presentan ejercicios para identificar las variables dependientes e independientes en distintas expresiones matemáticas.

\textbf{Ejercicio 1.} $y = 2 + x^2$

\textit{Solución:}
\begin{itemize}
    \item Variable independiente: $x$
    \item Variable dependiente: $y$
\end{itemize}

\textbf{Ejercicio 2.} $z = y + 9$

\textit{Solución:}
\begin{itemize}
    \item Variable independiente: $y$
    \item Variable dependiente: $z$
\end{itemize}

\textbf{Ejercicio 3.} $y = 2x + 3x^2 + 9$

\textit{Solución:}
\begin{itemize}
    \item Variable independiente: $x$
    \item Variable dependiente: $y$
\end{itemize}

\textbf{Ejercicio 4.} $x = 2t + 5$

\textit{Solución:}
\begin{itemize}
    \item Variable independiente: $t$
    \item Variable dependiente: $x$
\end{itemize}

\textbf{Ejercicio 5.} $9x + 20 = z$

\textit{Solución:}
\begin{itemize}
    \item Variable independiente: $x$
    \item Variable dependiente: $z$
\end{itemize}

\textbf{Ejercicio 6.} $z = 3t + 4y + 2x$

\textit{Solución:}
\begin{itemize}
    \item Variables independientes: $t, y, x$
    \item Variable dependiente: $z$
\end{itemize}

\textbf{Ejercicio 7.} $y^2 + 2x + 8y = t$

\textit{Solución:}
\begin{itemize}
    \item Variables independientes: $y, x$
    \item Variable dependiente: $t$
\end{itemize}

\begin{figure}[H]
    \centering
    \includegraphics[width=0.7\textwidth]{img/1_1_imagen_funciones.jpg}
    \caption{Funciones}
\end{figure}

La gráfica muestra la función $f(x) = 0.5x^2 - 2$, una parábola suave donde para cada valor de entrada $x$ existe una única salida $y$. Este ejemplo ilustra el concepto fundamental de función: una relación que asigna a cada elemento del dominio exactamente un elemento del rango.

\section{Dominio}

El dominio de una función se refiere al conjunto de todos los valores posibles que pueden ser entradas válidas para la función. Dicho de otra manera, es el conjunto de valores que podemos sustituir en la variable independiente sin que se produzca una contradicción o un valor indefinido.

\subsection{Definición Formal del Dominio}

Si tenemos una función $f: A \rightarrow B$, entonces el dominio de $f$ es el conjunto $A$, que contiene todos los valores de $x$ para los cuales la función está bien definida. Esto significa que, para cada $x$ en el dominio, la expresión matemática de la función no debe involucrar ninguna operación inválida, como la división entre cero o la raíz cuadrada de un número negativo (en el caso de los números reales).

\subsection{Ejemplos de Restricciones en el Dominio}

Algunas restricciones comunes que pueden existir en el dominio de una función incluyen:

\subsubsection{División por cero}

Una función no puede tener como entrada un valor que cause que su expresión tenga un denominador igual a cero.

\textit{Ejemplo:}
\begin{equation}
f(x) = \frac{1}{x - 1}
\label{eq:division_cero}
\end{equation}

El dominio de esta función es $\{x \in \mathbb{R} \mid x \neq 1\}$, ya que en $x = 1$, el denominador se vuelve cero.

\subsubsection{Raíces de números negativos}

Las funciones que involucran raíces cuadradas o de orden par no pueden tener números negativos en su dominio si estamos trabajando con números reales.

\textit{Ejemplo:}
$$f(x) = \sqrt{x - 2}$$

El dominio de esta función es $\{x \in \mathbb{R} \mid x \geq 2\}$, ya que no se puede tomar la raíz cuadrada de un número negativo.

\subsubsection{Logaritmos}

Los logaritmos no están definidos para valores menores o iguales a cero. Así que si la función es de la forma:
$$f(x) = \ln(x)$$

El dominio es $\{x \in \mathbb{R} \mid x > 0\}$, porque el argumento del logaritmo debe ser positivo.

\subsection{Dominio de Funciones Compuestas}

El dominio de una función compuesta $(f \circ g)(x) = f(g(x))$ es el conjunto de todos los valores $x$ para los cuales tanto $g(x)$ como $f(g(x))$ están definidos. Por ejemplo:

Si $g(x) = \sqrt{x}$ y $f(x) = \frac{1}{x}$, entonces el dominio de $f \circ g$ es $\{x \in \mathbb{R} \mid x > 0\}$, ya que debe cumplirse $\sqrt{x} > 0$.

\begin{figure}[H]
    \centering
    \includegraphics[width=0.7\textwidth]{img/1_2_imagen_dominio.jpg}
    \caption{Dominio}
\end{figure}

La función $g(x) = \sqrt{x + 3}$ muestra una curva que inicia en $x = -3$, evidenciando visualmente la restricción del dominio. Para valores menores a $-3$, la función no está definida en los números reales, ya que la raíz cuadrada de un número negativo no existe en este conjunto.

\section{Rango}

El rango de una función es el conjunto de todos los posibles valores de salida o resultados que puede tomar la función cuando se sustituyen los valores del dominio en la expresión de la función. En otras palabras, el rango es el conjunto de todas las imágenes de los valores del dominio bajo la acción de la función.

\subsection{Definición Formal del Rango}

El rango de la función $f: A \rightarrow B$ es el conjunto de valores $y \in B$ que pueden ser obtenidos para $y = f(x)$. Es decir, el rango es el conjunto de todos los valores de $y$ que pueden ser alcanzados por la función $f$.

\subsection{Ejemplos de Determinación del Rango}

\subsubsection{Función cuadrática}

Consideremos la función $f(x) = x^2$. Esta función toma cualquier número real $x$ y devuelve su cuadrado. El valor más bajo que puede tomar esta función es 0 (cuando $x = 0$), y no hay límite superior. Entonces, el rango de la función es:
$$\text{Rango}(f) = [0, +\infty)$$

\subsubsection{Función lineal}

Para la función lineal $f(x) = 2x + 3$, como es una línea recta, no hay restricciones en los valores que puede tomar la salida de la función. Por lo tanto, el rango de la función es el conjunto de todos los números reales:
$$\text{Rango}(f) = \mathbb{R}$$

\subsubsection{Función seno}

La función $f(x) = \sin(x)$ tiene un rango limitado porque el seno de un ángulo siempre está entre $-1$ y $1$, sin importar el valor de $x$. Por lo tanto, el rango de la función seno es:
\begin{equation}
\text{Rango}(\sin) = [-1, 1]
\label{eq:rango_seno}
\end{equation}

\subsubsection{Función logaritmo}

Si $f(x) = \ln(x)$, el rango de esta función es $\mathbb{R}$, ya que el logaritmo de cualquier número positivo es un número real.

\subsection{Cálculo del Rango de Funciones Compuestas}

El rango de una función compuesta $(f \circ g)(x)$ depende tanto del rango de $g$ como del comportamiento de la función $f$. Para calcularlo, primero debes determinar el rango de $g$ y luego determinar cómo estos valores son transformados por $f$.

\subsection{Ejercicios: Dominio y Rango}

A continuación se presentan ejercicios para determinar el dominio y rango de diversas funciones.

\textbf{Ejercicio 1.} $f(x) = 2x - 1$

\textit{Solución:}
\begin{itemize}
    \item $\text{Dom}(f) = (-\infty, \infty)$
    \item $\text{Ran}(f) = (-\infty, \infty)$
\end{itemize}

\textbf{Ejercicio 2.} $f(x) = \frac{x-6}{x-7}$

\textit{Solución:}
\begin{itemize}
    \item $\text{Dom}(f) = (-\infty, 7) \cup (7, \infty)$
    \item $\text{Ran}(f) = (-\infty, 1) \cup (1, \infty)$
\end{itemize}

\textbf{Ejercicio 3.} $f(x) = \sqrt{x + 3}$

\textit{Solución:}
\begin{itemize}
    \item $\text{Dom}(f) = [-3, \infty)$
    \item $\text{Ran}(f) = [0, \infty)$
\end{itemize}

\textbf{Ejercicio 4.} $f(x) = \frac{x}{x^2 - 5x + 6}$

\textit{Solución:}
\begin{itemize}
    \item $\text{Dom}(f) = (-\infty, 2) \cup (2, 3) \cup (3, \infty)$
    \item $\text{Ran}(f) = (-\infty, \infty)$
\end{itemize}

\textbf{Ejercicio 5.} $f(x) = x^2 + 3x + 2$

\textit{Solución:}
\begin{itemize}
    \item $\text{Dom}(f) = (-\infty, \infty)$
    \item $\text{Ran}(f) = (-\infty, \infty)$
\end{itemize}

\textbf{Ejercicio 6.} $f(x) = \frac{x+2}{x+3}$

\textit{Solución:}
\begin{itemize}
    \item $\text{Dom}(f) = (-\infty, -3) \cup (-3, \infty)$
    \item $\text{Ran}(f) = (-\infty, 1) \cup (1, \infty)$
\end{itemize}

\textbf{Ejercicio 7.} $f(x) = \frac{2x-1}{x^2 - x - 12}$

\textit{Solución:}
\begin{itemize}
    \item $\text{Dom}(f) = (-\infty, -3) \cup (-3, 4) \cup (4, \infty)$
    \item $\text{Ran}(f) = (-\infty, \infty)$
\end{itemize}

\textbf{Ejercicio 8.} $f(x) = \sqrt{x} + 3$

\textit{Solución:}
\begin{itemize}
    \item $\text{Dom}(f) = [0, \infty)$
    \item $\text{Ran}(f) = [3, \infty)$
\end{itemize}

\textbf{Ejercicio 9.} $f(x) = \sqrt{x + 8}$

\textit{Solución:}
\begin{itemize}
    \item $\text{Dom}(f) = [-8, \infty)$
    \item $\text{Ran}(f) = [0, \infty)$
\end{itemize}

\begin{figure}[H]
    \centering
    \includegraphics[width=0.7\textwidth]{img/1_3_imagen_rango.jpg}
    \caption{Rango}
\end{figure}

La función $h(x) = 2\sin(x) + 1$ representa una onda oscilatoria acotada superior e inferiormente. El rango de esta función está restringido al intervalo $[-1, 3]$, donde $-1$ es el valor mínimo alcanzado cuando $\sin(x) = -1$ y $3$ es el valor máximo cuando $\sin(x) = 1$.

\section{Funciones Explícitas}

Una función explícita es aquella en la que la variable dependiente, generalmente representada por la letra $y$, se encuentra despejada y expresada de manera directa en términos de la variable independiente, que suele ser $x$. En este tipo de funciones, la relación se puede escribir de la forma $y = f(x)$, lo que significa que para cada valor de $x$ se puede calcular de manera inmediata un valor de $y$.

Por ejemplo, las funciones $y = 2x + 3$, $y = x^2 - 4x + 1$ o $y = 5e^x$ son todas funciones explícitas porque muestran claramente cómo la variable dependiente depende de la independiente.

Las funciones explícitas son las más comunes y son ampliamente utilizadas en álgebra, cálculo y física, ya que permiten evaluar, graficar y analizar relaciones de manera directa. Su principal característica es que, dado un valor de $x$, se puede calcular sin ambigüedad el valor de $y$. Además, su representación gráfica es sencilla, ya que a cada valor de $x$ le corresponde un solo valor de $y$.

Estas funciones también facilitan el uso de herramientas del cálculo, como las derivadas, los límites o las integrales, ya que la variable dependiente se encuentra aislada y permite aplicar directamente las reglas de derivación o integración.

\subsection{Aplicaciones de las Funciones Explícitas}

Las funciones explícitas se usan ampliamente en distintos campos del conocimiento:

\begin{itemize}
    \item \textbf{Física:} Describen relaciones como la posición de un objeto en función del tiempo, mediante expresiones como $y = \frac{1}{2}gt^2$.
    \item \textbf{Economía:} Una función de costo puede expresarse como $C(q) = 50q + 200$, donde se muestra explícitamente cómo el costo depende de la cantidad producida.
    \item \textbf{Ingeniería:} Las funciones explícitas permiten modelar el comportamiento de sistemas eléctricos o mecánicos donde las variables dependen de forma directa unas de otras.
\end{itemize}

Gracias a esta característica, su uso es esencial para el modelado computacional y la simulación de procesos.

\begin{figure}[H]
    \centering
    \includegraphics[width=0.7\textwidth]{img/1_4_imagen_funciones_explicitas.jpg}
    \caption{Funciones Explícitas}
\end{figure}

La función cúbica $p(x) = x^3 - 3x$ se presenta en su forma estándar $y = f(x)$, donde la variable dependiente $y$ está completamente despejada. Esta representación explícita permite calcular directamente el valor de $y$ para cualquier valor dado de $x$, facilitando el análisis y la graficación de la función.

\section{Funciones Implícitas}

Por otro lado, una función implícita es aquella en la que las variables $x$ e $y$ están relacionadas dentro de una misma ecuación, sin que la variable dependiente esté despejada o aislada. En este tipo de funciones, no se puede escribir fácilmente una expresión de la forma $y = f(x)$, sino que la relación se expresa como $F(x, y) = 0$. En una función implícita, la variable dependiente no aparece de forma explícita, sino que su relación con la variable independiente se encuentra implícita dentro de la ecuación.

Un ejemplo clásico de función implícita es la ecuación de una circunferencia:
\begin{equation}
x^2 + y^2 = 25
\label{eq:circunferencia}
\end{equation}

En esta relación, $x$ e $y$ están vinculadas por la ecuación, pero $y$ no está despejada. Si se desea expresar la ecuación de forma explícita, se puede despejar $y$ obteniendo $y = \pm\sqrt{25 - x^2}$. Esto demuestra que, aunque una función implícita pueda convertirse en explícita, a menudo la ecuación original es más simple o más natural en su forma implícita.

Otro ejemplo de función implícita es $x^3 + y^3 - 3xy = 0$, la cual representa una curva algebraica que no puede escribirse fácilmente en forma explícita.

\subsection{Aplicaciones de las Funciones Implícitas}

Las funciones implícitas se utilizan cuando la relación entre variables es demasiado compleja o cuando la ecuación no se puede resolver de manera directa para aislar $y$. Son muy comunes en geometría analítica, donde se usan para representar curvas como circunferencias, elipses o hipérbolas.

En cálculo diferencial, también aparecen con frecuencia al estudiar derivadas de relaciones que no están dadas de forma explícita. Para este tipo de funciones se emplea la técnica de derivación implícita, la cual permite encontrar la tasa de cambio de $y$ respecto a $x$ incluso cuando $y$ no está despejada.

\subsection{Derivación Implícita}

La derivación implícita se basa en aplicar la regla de la cadena a ambas variables. Si se parte de una ecuación $F(x, y) = 0$, se diferencia con respecto a $x$ y se obtiene la relación:
$$\frac{\partial F}{\partial x} + \frac{\partial F}{\partial y}\frac{dy}{dx} = 0$$

Despejando, se obtiene la expresión general:
\begin{equation}
\frac{dy}{dx} = -\frac{\partial F/\partial x}{\partial F/\partial y}
\label{eq:derivacion_implicita}
\end{equation}

Por ejemplo, si se toma la ecuación $x^2 + y^2 = 25$, al derivar ambos lados con respecto a $x$ se obtiene:
$$2x + 2y\frac{dy}{dx} = 0$$

Despejando resulta:
$$\frac{dy}{dx} = -\frac{x}{y}$$

De esta forma se puede obtener la pendiente de la tangente en cualquier punto de la circunferencia, sin necesidad de despejar $y$.

\subsection{Teorema de la Función Implícita}

Existe un resultado fundamental relacionado con las funciones implícitas, conocido como el \textbf{Teorema de la Función Implícita}. Este teorema establece las condiciones bajo las cuales una relación implícita puede expresarse localmente como una función explícita.

Si $F(x, y)$ es una función continua y diferenciable y se cumple que $F(a, b) = 0$ y que la derivada parcial de $F$ con respecto a $y$, es decir, $\frac{\partial F}{\partial y}(a, b)$, es distinta de cero, entonces existe una función diferenciable $y = g(x)$ definida en una vecindad del punto $(a, b)$ tal que $F(x, g(x)) = 0$.

En términos sencillos, esto significa que, aunque globalmente no se pueda despejar $y$, en una región cercana a un punto particular sí puede comportarse como una función explícita. Este teorema es una de las herramientas más importantes en cálculo avanzado y análisis matemático, ya que permite estudiar relaciones complicadas sin necesidad de resolverlas completamente.

\begin{figure}[H]
    \centering
    \includegraphics[width=0.7\textwidth]{img/1_5_imagen_funciones_implicitas.jpg}
    \caption{Funciones Implícitas}
\end{figure}

La ecuación $x^2 + y^2 = 16$ define un círculo de radio 4 centrado en el origen, donde la variable $y$ no está despejada. Esta representación implícita no puede expresarse como una única función explícita $y = f(x)$, sino que requiere dos funciones: $y = \sqrt{16 - x^2}$ (semicírculo superior) y $y = -\sqrt{16 - x^2}$ (semicírculo inferior).

\section{Curvas de Nivel}

Las curvas de nivel son representaciones gráficas utilizadas en matemáticas para mostrar el comportamiento de funciones de dos variables. Una curva de nivel se define como el conjunto de puntos en el plano donde una función de dos variables toma un mismo valor constante.

En otras palabras, si se tiene una función $f(x, y)$, una curva de nivel se forma uniendo todos los puntos $(x, y)$ tales que $f(x, y) = c$, donde $c$ es una constante real. Estas curvas permiten visualizar cómo cambia una función en el espacio bidimensional, ya que muestran regiones donde la función tiene el mismo valor.

\subsection{Definición Formal}

Formalmente, si se considera una función $f: \mathbb{R}^2 \rightarrow \mathbb{R}$, las curvas de nivel se obtienen al resolver la ecuación:
\begin{equation}
f(x, y) = c
\label{eq:curva_nivel}
\end{equation}

para distintos valores de $c$. Cada valor de $c$ genera una curva diferente, y el conjunto de todas estas curvas proporciona una vista general del comportamiento de la función. De manera análoga, en tres dimensiones, las superficies de nivel representan el conjunto de puntos donde $f(x, y, z) = c$.

\subsection{Aplicaciones}

El concepto de curva de nivel tiene una gran utilidad en la comprensión de funciones multivariables, ya que permite analizar cómo varía la altura o el valor de la función en función de las variables de entrada.

En física y geografía, por ejemplo, las curvas de nivel se utilizan en los mapas topográficos para representar la elevación del terreno. En este caso, cada curva de nivel representa todos los puntos que se encuentran a la misma altitud. Cuanto más juntas se encuentren las curvas, mayor es la pendiente o el cambio en la elevación.

En matemáticas, se utilizan para estudiar el crecimiento o decrecimiento de funciones, la existencia de máximos y mínimos locales, y la forma de las superficies que describen.

\subsection{Ejemplos}

Para ilustrar este concepto, consideremos la función:
$$f(x, y) = x^2 + y^2$$

Si se elige un valor constante $c = r^2$, la ecuación de la curva de nivel será $x^2 + y^2 = r^2$. Esta expresión representa una familia de circunferencias concéntricas con centro en el origen y radio igual a $r$. Cada curva corresponde a un valor distinto de la función, y su disposición muestra cómo la función aumenta conforme nos alejamos del origen.

Este tipo de análisis es fundamental en el cálculo multivariable, donde las curvas de nivel permiten comprender de manera visual las propiedades locales y globales de una función.

En otras aplicaciones, las curvas de nivel permiten analizar funciones más complejas. Por ejemplo, si se considera $f(x, y) = x^2 - y^2$, las curvas de nivel estarán formadas por hipérbolas, ya que la ecuación $x^2 - y^2 = c$ define una familia de curvas hiperbólicas que se abren en distintas direcciones dependiendo del signo de $c$.

\subsection{Relación con el Gradiente}

El análisis de curvas de nivel tiene también una interpretación práctica en el estudio de gradientes. El gradiente de una función, denotado por $\nabla f$, es un vector que apunta en la dirección del máximo crecimiento de la función, y es siempre perpendicular a las curvas de nivel en un punto dado.

Esta propiedad es muy importante en optimización y análisis de superficies, ya que permite identificar puntos de máximo o mínimo y direcciones de mayor variación.

\subsection{Ejercicios: Curvas de Nivel}

A continuación se presentan ejercicios para determinar las curvas de nivel de funciones de dos variables.

\textbf{Ejercicio 1.} $f(x, y) = x^2 + y^2$, $z = 2$

\textit{Solución:}

Igualamos la función al valor constante:
$$x^2 + y^2 = 2$$

Dividimos ambos lados entre 2:
$$\frac{1}{2}x^2 + \frac{1}{2}y^2 = 1$$

Expresamos en forma estándar:
$$\frac{x^2}{2} + \frac{y^2}{2} = 1$$

$$\frac{x^2}{(\sqrt{2})^2} + \frac{y^2}{(\sqrt{2})^2} = 1$$

Por lo tanto, $a = \sqrt{2} \approx 1.4$ y $b = \sqrt{2} \approx 1.4$. La curva de nivel es una circunferencia de radio $\sqrt{2}$.

\textbf{Ejercicio 2.} $f(x, y) = x^2 + y^2$, $z = 3$

\textit{Solución:}

$$x^2 + y^2 = 3$$

$$\frac{1}{3}x^2 + \frac{1}{3}y^2 = 1$$

$$\frac{x^2}{3} + \frac{y^2}{3} = 1$$

$$\frac{x^2}{(\sqrt{3})^2} + \frac{y^2}{(\sqrt{3})^2} = 1$$

Por lo tanto, $a = \sqrt{3} \approx 1.7$ y $b = \sqrt{3} \approx 1.7$. La curva de nivel es una circunferencia de radio $\sqrt{3}$.

\textbf{Ejercicio 3.} $f(x, y) = x^2 + y^2$, $z = 4$

\textit{Solución:}

$$x^2 + y^2 = 4$$

$$\frac{1}{4}x^2 + \frac{1}{4}y^2 = 1$$

$$\frac{x^2}{4} + \frac{y^2}{4} = 1$$

$$\frac{x^2}{(\sqrt{4})^2} + \frac{y^2}{(\sqrt{4})^2} = 1$$

Por lo tanto, $a = \sqrt{4} = 2$ y $b = \sqrt{4} = 2$. La curva de nivel es una circunferencia de radio 2.

\textbf{Ejercicio 4.} $f(x, y) = x^2 + y^2$, $z = 5$

\textit{Solución:}

$$x^2 + y^2 = 5$$

$$\frac{1}{5}x^2 + \frac{1}{5}y^2 = 1$$

$$\frac{x^2}{5} + \frac{y^2}{5} = 1$$

$$\frac{x^2}{(\sqrt{5})^2} + \frac{y^2}{(\sqrt{5})^2} = 1$$

Por lo tanto, $a = \sqrt{5} \approx 2.2$ y $b = \sqrt{5} \approx 2.2$. La curva de nivel es una circunferencia de radio $\sqrt{5}$.

\textbf{Ejercicio 5.} $f(x, y) = x^2 + y^2$, $z = 6$

\textit{Solución:}

$$x^2 + y^2 = 6$$

$$\frac{1}{6}x^2 + \frac{1}{6}y^2 = 1$$

$$\frac{x^2}{6} + \frac{y^2}{6} = 1$$

$$\frac{x^2}{(\sqrt{6})^2} + \frac{y^2}{(\sqrt{6})^2} = 1$$

Por lo tanto, $a = \sqrt{6} \approx 2.4$ y $b = \sqrt{6} \approx 2.4$. La curva de nivel es una circunferencia de radio $\sqrt{6}$.

\begin{figure}[H]
    \centering
    \includegraphics[width=0.7\textwidth]{img/1_6_imagen_curvas_nivel.jpg}
    \caption{Curvas de Nivel}
\end{figure}

Las curvas de nivel $x^2 + y^2 = 1$, $x^2 + y^2 = 4$ y $x^2 + y^2 = 9$ forman círculos concéntricos de radios 1, 2 y 3 respectivamente. Estas curvas representan cortes horizontales a diferentes alturas de una superficie tridimensional, como un cono o paraboloide, permitiendo visualizar funciones de dos variables en el plano.

\section{Límites}

El concepto de límite es uno de los pilares fundamentales del cálculo diferencial e integral. El límite describe el comportamiento de una función cuando la variable independiente se aproxima a un cierto valor. En términos sencillos, el límite permite estudiar lo que sucede con el valor de una función cuando $x$ se acerca a un número determinado, incluso si la función no está definida exactamente en ese punto.

Los límites proporcionan la base para definir de manera rigurosa conceptos esenciales como la continuidad, la derivada y la integral.

\subsection{Definición Formal}

Formalmente, se dice que el límite de una función $f(x)$ cuando $x$ tiende a un número $a$ es igual a $L$, y se escribe como:
\begin{equation}
\lim_{x \to a} f(x) = L
\label{eq:definicion_limite}
\end{equation}

Esto significa que los valores de $f(x)$ pueden hacerse tan cercanos a $L$ como se desee, siempre que $x$ se acerque suficientemente a $a$, pero sin necesidad de que $x$ tome exactamente ese valor. Si el límite existe, se dice que la función tiene un comportamiento estable en torno a ese punto. Si el límite no existe, la función puede presentar discontinuidades, saltos o comportamientos infinitos.

\subsection{Interpretación Intuitiva}

El límite se puede entender intuitivamente como la tendencia de una función a un valor determinado. Por ejemplo, si se considera la función:
$$f(x) = \frac{x^2 - 1}{x - 1}$$

Al sustituir $x = 1$ se obtiene una indeterminación $\frac{0}{0}$, ya que el denominador se vuelve cero. Sin embargo, al analizar los valores de $x$ cercanos a 1, se observa que la función se aproxima a 2. Por lo tanto, el límite de la función cuando $x$ tiende a 1 es 2, es decir:
$$\lim_{x \to 1} \frac{x^2 - 1}{x - 1} = 2$$

Este ejemplo demuestra que el límite permite estudiar el comportamiento de las funciones en puntos donde no están definidas.

\subsection{Límites en Funciones de Varias Variables}

El concepto de límite no solo se aplica a funciones de una variable, sino también a funciones de varias variables. En estos casos, el límite estudia el comportamiento de una función $f(x, y)$ cuando el punto $(x, y)$ se aproxima a un valor $(a, b)$.

Para que el límite exista, el valor al que se aproxima la función debe ser el mismo sin importar la dirección desde la cual $(x, y)$ se acerca al punto $(a, b)$.

Por ejemplo, para la función:
$$f(x, y) = \frac{xy}{x^2 + y^2}$$

El límite cuando $(x, y)$ tiende al origen depende del camino seguido. Si se toma la recta $y = x$, el límite es $\frac{1}{2}$, pero si se toma $y = 0$, el límite es 0. Como el resultado depende del camino, el límite no existe.

Esto demuestra que, en funciones de varias variables, la existencia del límite requiere que el valor sea independiente de la trayectoria de aproximación.

\subsection{Importancia de los Límites}

El estudio de límites es esencial porque permite definir otros conceptos fundamentales del cálculo. Una función se dice continua en un punto $a$ si el límite de la función en ese punto es igual al valor de la función:
\begin{equation}
\lim_{x \to a} f(x) = f(a)
\label{eq:continuidad}
\end{equation}

Asimismo, el límite es la base para definir la derivada, que representa la tasa de cambio instantánea de una función. La derivada de una función $f$ en un punto $a$ se define como el límite del cociente incremental cuando el incremento $h$ tiende a cero:
\begin{equation}
f'(a) = \lim_{h \to 0} \frac{f(a + h) - f(a)}{h}
\label{eq:derivada}
\end{equation}

De manera similar, en el cálculo integral, el límite se utiliza para expresar el proceso de acumulación continua de cantidades infinitesimales, lo que da origen a la integral definida.

\subsection{Límites al Infinito y Comportamiento Asintótico}

El límite también se aplica al estudio de infinitos y al comportamiento asintótico de las funciones. Por ejemplo, si una función $f(x)$ se aproxima indefinidamente a un valor constante $L$ cuando $x$ crece sin límite, se dice que tiene un límite finito al infinito:
$$\lim_{x \to \infty} f(x) = L$$

En cambio, si la función crece sin límite cuando $x$ tiende a cierto valor $a$, se dice que el límite es infinito:
$$\lim_{x \to a} f(x) = \infty$$

Estos conceptos permiten analizar el crecimiento, decrecimiento y comportamiento global de funciones en el análisis matemático.

\subsection{Síntesis}

En síntesis, el concepto de límite constituye el fundamento del cálculo moderno. Permite describir con precisión el comportamiento local de las funciones, estudiar su continuidad, definir la derivada y la integral, y analizar fenómenos de cambio continuo en la naturaleza. Su comprensión es esencial para cualquier rama de la ciencia que utilice modelos matemáticos, desde la física y la ingeniería hasta la economía y la biología.

\subsection{Ejercicios: Límites}

A continuación se presentan ejercicios de cálculo de límites utilizando técnicas algebraicas y de racionalización.

\textbf{Ejercicio 1.} $\lim_{x \to -3} \frac{x + 3}{x^2 - 9}$

\textit{Solución:}

Factorizamos el denominador:
$$x^2 - 9 = (x - 3)(x + 3)$$

Simplificamos:
$$\frac{x + 3}{(x - 3)(x + 3)} = \frac{1}{x - 3}$$

Evaluamos el límite:
$$\lim_{x \to -3} \frac{1}{x - 3} = \frac{1}{-3 - 3} = \frac{1}{-6} = -\frac{1}{6}$$

\textbf{Ejercicio 2.} $\lim_{x \to 1} \frac{x^2 - 4x + 3}{x^2 + 3x - 4}$

\textit{Solución:}

Factorizamos numerador y denominador:
\begin{align*}
x^2 - 4x + 3 &= (x - 1)(x - 3)\\
x^2 + 3x - 4 &= (x - 1)(x + 4)
\end{align*}

Simplificamos:
$$\frac{(x - 1)(x - 3)}{(x - 1)(x + 4)} = \frac{x - 3}{x + 4}$$

Evaluamos el límite:
$$\lim_{x \to 1} \frac{x - 3}{x + 4} = \frac{1 - 3}{1 + 4} = \frac{-2}{5} = -\frac{2}{5}$$

\textbf{Ejercicio 3.} $\lim_{x \to 0} \frac{x^3 - 2x^2}{3x^2}$

\textit{Solución:}

Factorizamos el numerador:
$$\frac{x^2(x - 2)}{3x^2} = \frac{x - 2}{3}$$

Evaluamos el límite:
$$\lim_{x \to 0} \frac{x - 2}{3} = \frac{-2}{3} = -\frac{2}{3}$$

\textbf{Ejercicio 4.} $\lim_{x \to 5} \frac{x^2 - 10x + 25}{x^2 - 3x - 10}$

\textit{Solución:}

Factorizamos numerador y denominador:
\begin{align*}
x^2 - 10x + 25 &= (x - 5)^2\\
x^2 - 3x - 10 &= (x - 5)(x + 2)
\end{align*}

Simplificamos:
$$\frac{(x - 5)^2}{(x - 5)(x + 2)} = \frac{x - 5}{x + 2}$$

Evaluamos el límite:
$$\lim_{x \to 5} \frac{x - 5}{x + 2} = \frac{0}{7} = 0$$

\textbf{Ejercicio 5.} $\lim_{x \to 4} \frac{x^2 - 4x}{x^2 - 16}$

\textit{Solución:}

Factorizamos numerador y denominador:
\begin{align*}
x^2 - 4x &= x(x - 4)\\
x^2 - 16 &= (x - 4)(x + 4)
\end{align*}

Simplificamos:
$$\frac{x(x - 4)}{(x - 4)(x + 4)} = \frac{x}{x + 4}$$

Evaluamos el límite:
$$\lim_{x \to 4} \frac{x}{x + 4} = \frac{4}{8} = \frac{1}{2}$$

\textbf{Ejercicio 6.} $\lim_{x \to 0^+} \frac{x^3 - 5x^2 + 3x}{4x^2 - 4x}$

\textit{Solución:}

Factorizamos numerador y denominador:
\begin{align*}
x^3 - 5x^2 + 3x &= x(x^2 - 5x + 3)\\
4x^2 - 4x &= 4x(x - 1)
\end{align*}

Simplificamos:
$$\frac{x(x^2 - 5x + 3)}{4x(x - 1)} = \frac{x^2 - 5x + 3}{4(x - 1)}$$

Evaluamos el límite:
$$\lim_{x \to 0^+} \frac{3}{4(-1)} = -\frac{3}{4}$$

\textbf{Ejercicio 7.} $\lim_{x \to 0} \frac{\sqrt{9 + x} - 3}{x}$

\textit{Solución:}

Multiplicamos por el conjugado:
$$\frac{\sqrt{9 + x} - 3}{x} \cdot \frac{\sqrt{9 + x} + 3}{\sqrt{9 + x} + 3} = \frac{(9 + x) - 9}{x(\sqrt{9 + x} + 3)} = \frac{x}{x(\sqrt{9 + x} + 3)}$$

Simplificamos:
$$= \frac{1}{\sqrt{9 + x} + 3}$$

Evaluamos el límite:
$$\lim_{x \to 0} \frac{1}{\sqrt{9} + 3} = \frac{1}{3 + 3} = \frac{1}{6}$$

\textbf{Ejercicio 8.} $\lim_{x \to 1} \frac{\sqrt{x + 3} - 2}{x - 1}$

\textit{Solución:}

Multiplicamos por el conjugado:
$$\frac{\sqrt{x + 3} - 2}{x - 1} \cdot \frac{\sqrt{x + 3} + 2}{\sqrt{x + 3} + 2} = \frac{x + 3 - 4}{(x - 1)(\sqrt{x + 3} + 2)} = \frac{x - 1}{(x - 1)(\sqrt{x + 3} + 2)}$$

Simplificamos:
$$= \frac{1}{\sqrt{x + 3} + 2}$$

Evaluamos el límite:
$$\lim_{x \to 1} \frac{1}{\sqrt{4} + 2} = \frac{1}{2 + 2} = \frac{1}{4}$$

\textbf{Ejercicio 9.} $\lim_{x \to -2} \frac{x + 2}{\sqrt{x + 3} - 1}$

\textit{Solución:}

Multiplicamos por el conjugado del denominador:
$$\frac{x + 2}{\sqrt{x + 3} - 1} \cdot \frac{\sqrt{x + 3} + 1}{\sqrt{x + 3} + 1} = \frac{(x + 2)(\sqrt{x + 3} + 1)}{(x + 3) - 1} = \frac{(x + 2)(\sqrt{x + 3} + 1)}{x + 2}$$

Simplificamos:
$$= \sqrt{x + 3} + 1$$

Evaluamos el límite:
$$\lim_{x \to -2} (\sqrt{1} + 1) = 1 + 1 = 2$$

\textbf{Ejercicio 10.} $\lim_{x \to 0} \frac{x^2}{\sqrt{x + 3} - \sqrt{3}}$

\textit{Solución:}

Multiplicamos por el conjugado del denominador:
$$\frac{x^2}{\sqrt{x + 3} - \sqrt{3}} \cdot \frac{\sqrt{x + 3} + \sqrt{3}}{\sqrt{x + 3} + \sqrt{3}} = \frac{x^2(\sqrt{x + 3} + \sqrt{3})}{(x + 3) - 3} = \frac{x^2(\sqrt{x + 3} + \sqrt{3})}{x}$$

Simplificamos:
$$= x(\sqrt{x + 3} + \sqrt{3})$$

Evaluamos el límite:
$$\lim_{x \to 0} 0(\sqrt{3} + \sqrt{3}) = 0$$

\textbf{Ejercicio 11.} $\lim_{x \to 0} \frac{\sqrt{4 - 2x + x^2} - 2}{x}$

\textit{Solución:}

Multiplicamos por el conjugado:
\begin{align*}
\frac{\sqrt{4 - 2x + x^2} - 2}{x} \cdot \frac{\sqrt{4 - 2x + x^2} + 2}{\sqrt{4 - 2x + x^2} + 2} &= \frac{4 - 2x + x^2 - 4}{x(\sqrt{4 - 2x + x^2} + 2)}\\
&= \frac{x^2 - 2x}{x(\sqrt{4 - 2x + x^2} + 2)}\\
&= \frac{x - 2}{\sqrt{4 - 2x + x^2} + 2}
\end{align*}

Evaluamos el límite:
$$\lim_{x \to 0} \frac{-2}{\sqrt{4} + 2} = \frac{-2}{2 + 2} = -\frac{1}{2}$$

\begin{figure}[H]
    \centering
    \includegraphics[width=0.7\textwidth]{img/1_7_imagen_limites.jpg}
    \caption{Límites}
\end{figure}

La función $q(x) = \frac{2x^2 + 1}{x^2}$ se acerca asintóticamente a la recta horizontal $y = 2$ cuando $x$ tiende al infinito. Este comportamiento ilustra el concepto de límite: $\lim_{x \to \infty} q(x) = 2$, mostrando cómo una función puede aproximarse indefinidamente a un valor sin necesariamente alcanzarlo.

% ==================== UNIDAD II ====================
\chapter{Unidad II: Derivadas Parciales}

Las derivadas parciales constituyen una de las herramientas fundamentales del cálculo multivariable, permitiendo analizar de manera precisa cómo una función de varias variables cambia respecto a cada una de sus variables independientes. En el contexto de la ingeniería moderna, el dominio de estos conceptos es esencial para modelar fenómenos complejos que dependen de múltiples parámetros simultáneamente.

Desde el análisis de transferencia de calor en estructuras hasta la optimización de procesos de producción, las derivadas parciales proporcionan el marco matemático necesario para comprender y resolver problemas del mundo real. Este capítulo explora los conceptos fundamentales de las derivadas parciales, sus propiedades matemáticas, las reglas de derivación aplicables y sus múltiples aplicaciones en ciencia e ingeniería.

\section{Derivadas Parciales de Primer Orden}

\subsection{Definición y Concepto}

Cuando una función depende de dos o más variables independientes, por ejemplo $x$ e $y$, el estudio de cómo cambia la función al variar solo una de esas variables conduce al concepto de derivada parcial. A diferencia del cálculo de una variable, donde existe una única dirección de cambio, en funciones multivariables es necesario analizar el cambio en múltiples direcciones.

Para una función $f(x, y)$ de dos variables, la derivada parcial con respecto a $x$ se define como el límite:
\begin{equation}
\frac{\partial f}{\partial x} = \lim_{h \to 0} \frac{f(x + h, y) - f(x, y)}{h}
\label{eq:derivada_parcial_x}
\end{equation}

Esta expresión representa la tasa de cambio instantánea de la función $f$ cuando la variable $x$ se modifica mientras la variable $y$ permanece constante. De manera análoga, la derivada parcial con respecto a $y$ se define como:
\begin{equation}
\frac{\partial f}{\partial y} = \lim_{h \to 0} \frac{f(x, y + h) - f(x, y)}{h}
\label{eq:derivada_parcial_y}
\end{equation}

\subsection{Interpretación Geométrica}

Geométricamente, una función de dos variables $f(x, y)$ puede representarse como una superficie en el espacio tridimensional. La derivada parcial $\frac{\partial f}{\partial x}$ en un punto $(x_0, y_0)$ representa la pendiente de la curva que resulta de la intersección de la superficie con el plano $y = y_0$. Esta curva muestra cómo varía la función cuando nos movemos en la dirección del eje $x$ manteniendo $y$ fijo.

De forma similar, $\frac{\partial f}{\partial y}$ representa la pendiente de la curva resultante de intersectar la superficie con el plano $x = x_0$, mostrando la variación en la dirección del eje $y$.

Esta interpretación es particularmente útil en ingeniería cuando se analizan superficies topográficas, distribuciones de temperatura en placas, o cualquier fenómeno físico que pueda modelarse como una superficie en el espacio.

\subsection{Notación}

Existen varias notaciones equivalentes para las derivadas parciales, cada una con ventajas particulares según el contexto:

\begin{itemize}
    \item Notación de Leibniz: $\frac{\partial f}{\partial x}$, $\frac{\partial f}{\partial y}$
    \item Notación con subíndices: $f_x$, $f_y$
    \item Notación con índices: $D_1 f$, $D_2 f$ (donde el índice indica la variable)
    \item Notación de gradiente: $\nabla f = \left(\frac{\partial f}{\partial x}, \frac{\partial f}{\partial y}\right)$
\end{itemize}

La notación de Leibniz con el símbolo $\partial$ (d redondeada) es la más común y ayuda a distinguir visualmente las derivadas parciales de las derivadas totales u ordinarias.

\subsection{Ejemplo 1: Cálculo de Derivadas Parciales de Primer Orden}

\textbf{Problema:} Dada la función $f(x, y) = x^3 y^2 + 2xy - 5y^3$, calcular las derivadas parciales de primer orden $\frac{\partial f}{\partial x}$ y $\frac{\partial f}{\partial y}$.

\textbf{Solución:}

\textit{Paso 1: Calcular $\frac{\partial f}{\partial x}$}

Para calcular la derivada parcial con respecto a $x$, tratamos a $y$ como una constante y derivamos término por término:

\begin{align}
f(x, y) &= x^3 y^2 + 2xy - 5y^3 \notag\\
\frac{\partial f}{\partial x} &= \frac{\partial}{\partial x}(x^3 y^2) + \frac{\partial}{\partial x}(2xy) - \frac{\partial}{\partial x}(5y^3) \notag
\end{align}

Aplicando las reglas de derivación:
\begin{itemize}
    \item Para $x^3 y^2$: tratamos $y^2$ como constante, entonces $\frac{\partial}{\partial x}(x^3 y^2) = 3x^2 y^2$
    \item Para $2xy$: tratamos $y$ como constante, entonces $\frac{\partial}{\partial x}(2xy) = 2y$
    \item Para $5y^3$: es constante respecto a $x$, entonces $\frac{\partial}{\partial x}(5y^3) = 0$
\end{itemize}

Por lo tanto:
\begin{equation}
\frac{\partial f}{\partial x} = 3x^2 y^2 + 2y
\label{eq:ejemplo1_fx}
\end{equation}

\textit{Paso 2: Calcular $\frac{\partial f}{\partial y}$}

Ahora derivamos con respecto a $y$, tratando a $x$ como constante:

\begin{align}
\frac{\partial f}{\partial y} &= \frac{\partial}{\partial y}(x^3 y^2) + \frac{\partial}{\partial y}(2xy) - \frac{\partial}{\partial y}(5y^3) \notag
\end{align}

Aplicando las reglas de derivación:
\begin{itemize}
    \item Para $x^3 y^2$: tratamos $x^3$ como constante, entonces $\frac{\partial}{\partial y}(x^3 y^2) = x^3 \cdot 2y = 2x^3 y$
    \item Para $2xy$: tratamos $x$ como constante, entonces $\frac{\partial}{\partial y}(2xy) = 2x$
    \item Para $5y^3$: aplicamos la regla de la potencia, $\frac{\partial}{\partial y}(5y^3) = 15y^2$
\end{itemize}

Por lo tanto:
\begin{equation}
\frac{\partial f}{\partial y} = 2x^3 y + 2x - 15y^2
\label{eq:ejemplo1_fy}
\end{equation}

\textbf{Verificación en un punto:} Si evaluamos en $(x, y) = (1, 2)$:
\begin{align}
\frac{\partial f}{\partial x}\bigg|_{(1,2)} &= 3(1)^2(2)^2 + 2(2) = 12 + 4 = 16 \notag\\
\frac{\partial f}{\partial y}\bigg|_{(1,2)} &= 2(1)^3(2) + 2(1) - 15(2)^2 = 4 + 2 - 60 = -54 \notag
\end{align}

Esto indica que en el punto $(1, 2)$, la función aumenta a razón de 16 unidades por unidad de cambio en $x$, y disminuye a razón de 54 unidades por unidad de cambio en $y$.

\vspace{1cm}

\subsection*{Derivadas Parciales de Primer Orden}

% ============================================================
\subsection*{Ejercicio 1}
Sea:
\[
f(x,y)=x^2 + 3xy + y^2 + 5
\]

% -----------------------------------------
\subsubsection*{Derivada respecto a $x$}

\[
\frac{\partial}{\partial x}(x^2)=2x
\]

\[
\frac{\partial}{\partial x}(3xy)=3y
\]

\[
\frac{\partial}{\partial x}(y^2)=0
\]

\[
\frac{\partial}{\partial x}(5)=0
\]

\[
\boxed{\frac{\partial f}{\partial x}=2x+3y}
\]

% -----------------------------------------
\subsubsection*{Derivada respecto a $y$}

\[
\frac{\partial}{\partial y}(x^2)=0
\]

\[
\frac{\partial}{\partial y}(3xy)=3x
\]

\[
\frac{\partial}{\partial y}(y^2)=2y
\]

\[
\frac{\partial}{\partial y}(5)=0
\]

\[
\boxed{\frac{\partial f}{\partial y}=3x+2y}
\]

% ============================================================
\subsection*{Ejercicio 2}
Sea:
\[
f(x,y)=e^{\,2x^2 + xy + y^2}
\]

% -----------------------------------------
\subsubsection*{Derivada respecto a $x$}

\[
\frac{\partial}{\partial x}(2x^2)=4x
\]

\[
\frac{\partial}{\partial x}(xy)=y
\]

\[
\frac{\partial}{\partial x}(y^2)=0
\]

\[
\boxed{\frac{\partial f}{\partial x}=(4x+y)e^{\,2x^2 + xy + y^2}}
\]

% -----------------------------------------
\subsubsection*{Derivada respecto a $y$}

\[
\frac{\partial}{\partial y}(2x^2)=0
\]

\[
\frac{\partial}{\partial y}(xy)=x
\]

\[
\frac{\partial}{\partial y}(y^2)=2y
\]

\[
\boxed{\frac{\partial f}{\partial y}=(x+2y)e^{\,2x^2 + xy + y^2}}
\]

% ============================================================
\subsection*{Ejercicio 3}
Sea:
\[
f(x,y)=\sin(3x + 2y^2)
\]

% -----------------------------------------
\subsubsection*{Derivada respecto a $x$}

\[
\frac{\partial}{\partial x}(3x)=3
\]

\[
\frac{\partial}{\partial x}(2y^2)=0
\]

\[
\boxed{\frac{\partial f}{\partial x}=3\cos(3x+2y^2)}
\]

% -----------------------------------------
\subsubsection*{Derivada respecto a $y$}

\[
\frac{\partial}{\partial y}(3x)=0
\]

\[
\frac{\partial}{\partial y}(2y^2)=4y
\]

\[
\boxed{\frac{\partial f}{\partial y}=4y\cos(3x+2y^2)}
\]

% ============================================================
\subsection*{Ejercicio 4}
Sea:
\[
f(x,y) = x^3 + y^5 - 3x^2 y
\]

% -----------------------------------------
\subsubsection*{Derivada respecto a $x$}

\[
\frac{\partial}{\partial x}(x^3)=3x^2
\]

\[
\frac{\partial}{\partial x}(y^5)=0
\]

\[
\frac{\partial}{\partial x}(-3x^2 y)=-6xy
\]

\[
\boxed{\frac{\partial f}{\partial x}=3x^2 - 6xy}
\]

% -----------------------------------------
\subsubsection*{Derivada respecto a $y$}

\[
\frac{\partial}{\partial y}(x^3)=0
\]

\[
\frac{\partial}{\partial y}(y^5)=5y^4
\]

\[
\frac{\partial}{\partial y}(-3x^2 y)=-3x^2
\]

\[
\boxed{\frac{\partial f}{\partial y}=5y^4 - 3x^2}
\]

% ============================================================
\subsection*{Ejercicio 5}
Sea:
\[
f(x,y)=\frac{x^4 y^3}{6}-\frac{2xy^6}{3}
\]

% -----------------------------------------
\subsubsection*{Derivada respecto a $x$}

\[
\frac{\partial}{\partial x}\left(\frac{x^4 y^3}{6}\right)=\frac{4x^3 y^3}{6}
\]

\[
\frac{\partial}{\partial x}\left(\frac{2xy^6}{3}\right)=\frac{2y^6}{3}
\]

\[
\boxed{\frac{\partial f}{\partial x}=\frac{4x^3 y^3}{6}-\frac{2y^6}{3}}
\]

% -----------------------------------------
\subsubsection*{Derivada respecto a $y$}

\[
\frac{\partial}{\partial y}\left(\frac{x^4 y^3}{6}\right)=\frac{3x^4 y^2}{6}
= \frac{1}{2}x^4 y^2
\]

\[
\frac{\partial}{\partial y}\left(\frac{2xy^6}{3}\right)=4xy^5
\]

\[
\boxed{\frac{\partial f}{\partial y}=\frac{1}{2}x^4 y^2 - 4xy^5}
\]

% ============================================================
\subsection*{Ejercicio 6}
Sea:
\[
f(x,y)=6x^2\cos(y^3)
\]

% -----------------------------------------
\subsubsection*{Derivada respecto a $x$}

\[
\frac{\partial}{\partial x}(6x^2)=12x
\]

\[
\boxed{\frac{\partial f}{\partial x}=12x\cos(y^3)}
\]

% -----------------------------------------
\subsubsection*{Derivada respecto a $y$}

\[
\frac{\partial}{\partial y}(\cos(y^3))
= -\sin(y^3)\cdot 3y^2
\]

\[
\boxed{\frac{\partial f}{\partial y}
= -18x^2 y^2 \sin(y^3)}
\]

% ============================================================
\subsection*{Ejercicio 7}
Sea:
\[
f(x,y) = (5x^2 + 3xy - y^3)^5
\]

Definimos:
\[
u = 5x^2 + 3xy - y^3
\]

% -----------------------------------------
\subsubsection*{Derivada respecto a $x$}

\[
\frac{\partial f}{\partial x}
= 5u^{4}(10x + 3y)
\]

\[
\boxed{
\frac{\partial f}{\partial x}
= 5(5x^2 + 3xy - y^3)^4 (10x + 3y)
}
\]

% -----------------------------------------
\subsubsection*{Derivada respecto a $y$}

\[
\frac{\partial u}{\partial y}
= 3x - 3y^2
\]

\[
\frac{\partial f}{\partial y}
= 5u^{4}(3x - 3y^2)
\]

\[
\boxed{
\frac{\partial f}{\partial y}
= 5(5x^2 + 3xy - y^3)^4 (3x - 3y^2)
}
\]

\begin{figure}[H]
    \centering
    \includegraphics[width=0.7\textwidth]{img/2_1_imagen_derivadas_primer_orden.jpg}
    \caption{Derivadas Parciales de Primer Orden}
\end{figure}

La superficie $f(x,y) = x^2 - y^2$ forma una "silla de montar" (paraboloide hiperbólico) que muestra pendientes distintas según la dirección considerada. La derivada parcial $\frac{\partial f}{\partial x} = 2x$ mide la pendiente en la dirección $x$ (manteniendo $y$ constante), mientras que $\frac{\partial f}{\partial y} = -2y$ mide la pendiente en la dirección $y$ (manteniendo $x$ constante).

\section{Derivadas Parciales de Segundo Orden}

\subsection{Definición y Tipos}

Las derivadas parciales de segundo orden resultan de aplicar nuevamente la operación de derivación parcial a una derivada parcial de primer orden. Para una función de dos variables $f(x, y)$, existen cuatro posibles derivadas parciales de segundo orden:

\begin{itemize}
    \item \textbf{Derivada segunda respecto a $x$:} 
    \begin{equation}
    \frac{\partial^2 f}{\partial x^2} = \frac{\partial}{\partial x}\left(\frac{\partial f}{\partial x}\right) = f_{xx}
    \label{eq:derivada_segunda_xx}
    \end{equation}
    
    \item \textbf{Derivada segunda respecto a $y$:}
    \begin{equation}
    \frac{\partial^2 f}{\partial y^2} = \frac{\partial}{\partial y}\left(\frac{\partial f}{\partial y}\right) = f_{yy}
    \label{eq:derivada_segunda_yy}
    \end{equation}
    
    \item \textbf{Derivada mixta (primero $x$, luego $y$):}
    \begin{equation}
    \frac{\partial^2 f}{\partial y \partial x} = \frac{\partial}{\partial y}\left(\frac{\partial f}{\partial x}\right) = f_{xy}
    \label{eq:derivada_mixta_xy}
    \end{equation}
    
    \item \textbf{Derivada mixta (primero $y$, luego $x$):}
    \begin{equation}
    \frac{\partial^2 f}{\partial x \partial y} = \frac{\partial}{\partial x}\left(\frac{\partial f}{\partial y}\right) = f_{yx}
    \label{eq:derivada_mixta_yx}
    \end{equation}
\end{itemize}

\subsection{Teorema de Schwarz (Igualdad de las Derivadas Mixtas)}

Una propiedad fundamental de las derivadas parciales de segundo orden es que, bajo condiciones adecuadas de regularidad, las derivadas mixtas son iguales independientemente del orden de derivación. Este resultado se conoce como el Teorema de Schwarz o Teorema de Clairaut.

\textbf{Teorema:} Si $f$ es una función de dos variables y las derivadas parciales $f_{xy}$ y $f_{yx}$ son continuas en un conjunto abierto, entonces:
\begin{equation}
\frac{\partial^2 f}{\partial x \partial y} = \frac{\partial^2 f}{\partial y \partial x}
\label{eq:teorema_schwarz}
\end{equation}

Este teorema simplifica significativamente el cálculo de derivadas parciales de orden superior, ya que garantiza que para funciones suficientemente suaves, solo necesitamos calcular tres derivadas de segundo orden distintas en lugar de cuatro.

\subsection{Interpretación y Aplicaciones}

Las derivadas de segundo orden proporcionan información sobre la curvatura de la superficie representada por la función:

\begin{itemize}
    \item $\frac{\partial^2 f}{\partial x^2}$ mide la curvatura en la dirección $x$
    \item $\frac{\partial^2 f}{\partial y^2}$ mide la curvatura en la dirección $y$
    \item Las derivadas mixtas miden la torsión de la superficie
\end{itemize}

Esta información es crucial para el análisis de extremos (máximos, mínimos y puntos de silla) mediante el criterio de la segunda derivada, que involucra la matriz Hessiana:

\begin{equation}
H = \begin{pmatrix}
\frac{\partial^2 f}{\partial x^2} & \frac{\partial^2 f}{\partial x \partial y} \\
\frac{\partial^2 f}{\partial y \partial x} & \frac{\partial^2 f}{\partial y^2}
\end{pmatrix}
\label{eq:matriz_hessiana}
\end{equation}

\subsection{Ejemplo 2: Cálculo de Derivadas Parciales de Segundo Orden}

\textbf{Problema:} Para la función $f(x, y) = e^{xy} + x^2 \sin(y)$, calcular todas las derivadas parciales de segundo orden y verificar el Teorema de Schwarz.

\textbf{Solución:}

\textit{Paso 1: Calcular las derivadas parciales de primer orden}

\begin{align}
\frac{\partial f}{\partial x} &= \frac{\partial}{\partial x}(e^{xy}) + \frac{\partial}{\partial x}(x^2 \sin(y)) \notag\\
&= y e^{xy} + 2x \sin(y)
\label{eq:ejemplo2_fx}
\end{align}

\begin{align}
\frac{\partial f}{\partial y} &= \frac{\partial}{\partial y}(e^{xy}) + \frac{\partial}{\partial y}(x^2 \sin(y)) \notag\\
&= x e^{xy} + x^2 \cos(y)
\label{eq:ejemplo2_fy}
\end{align}

\textit{Paso 2: Calcular $\frac{\partial^2 f}{\partial x^2}$}

Derivamos $\frac{\partial f}{\partial x} = y e^{xy} + 2x \sin(y)$ con respecto a $x$:

\begin{align}
\frac{\partial^2 f}{\partial x^2} &= \frac{\partial}{\partial x}(y e^{xy}) + \frac{\partial}{\partial x}(2x \sin(y)) \notag\\
&= y \cdot y e^{xy} + 2\sin(y) \notag\\
&= y^2 e^{xy} + 2\sin(y)
\label{eq:ejemplo2_fxx}
\end{align}

\textit{Paso 3: Calcular $\frac{\partial^2 f}{\partial y^2}$}

Derivamos $\frac{\partial f}{\partial y} = x e^{xy} + x^2 \cos(y)$ con respecto a $y$:

\begin{align}
\frac{\partial^2 f}{\partial y^2} &= \frac{\partial}{\partial y}(x e^{xy}) + \frac{\partial}{\partial y}(x^2 \cos(y)) \notag\\
&= x \cdot x e^{xy} + x^2 \cdot (-\sin(y)) \notag\\
&= x^2 e^{xy} - x^2 \sin(y)
\label{eq:ejemplo2_fyy}
\end{align}

\textit{Paso 4: Calcular $\frac{\partial^2 f}{\partial y \partial x}$}

Derivamos $\frac{\partial f}{\partial x} = y e^{xy} + 2x \sin(y)$ con respecto a $y$:

\begin{align}
\frac{\partial^2 f}{\partial y \partial x} &= \frac{\partial}{\partial y}(y e^{xy}) + \frac{\partial}{\partial y}(2x \sin(y)) \notag\\
&= e^{xy} + y \cdot x e^{xy} + 2x \cos(y) \notag\\
&= e^{xy} + xy e^{xy} + 2x \cos(y) \notag\\
&= e^{xy}(1 + xy) + 2x \cos(y)
\label{eq:ejemplo2_fxy}
\end{align}

\textit{Paso 5: Calcular $\frac{\partial^2 f}{\partial x \partial y}$}

Derivamos $\frac{\partial f}{\partial y} = x e^{xy} + x^2 \cos(y)$ con respecto a $x$:

\begin{align}
\frac{\partial^2 f}{\partial x \partial y} &= \frac{\partial}{\partial x}(x e^{xy}) + \frac{\partial}{\partial x}(x^2 \cos(y)) \notag\\
&= e^{xy} + x \cdot y e^{xy} + 2x \cos(y) \notag\\
&= e^{xy} + xy e^{xy} + 2x \cos(y) \notag\\
&= e^{xy}(1 + xy) + 2x \cos(y)
\label{eq:ejemplo2_fyx}
\end{align}

\textit{Paso 6: Verificación del Teorema de Schwarz}

Observamos que:
\begin{equation}
\frac{\partial^2 f}{\partial y \partial x} = \frac{\partial^2 f}{\partial x \partial y} = e^{xy}(1 + xy) + 2x \cos(y)
\label{eq:ejemplo2_verificacion}
\end{equation}

Esto confirma el Teorema de Schwarz para esta función, ya que ambas derivadas mixtas son idénticas. Esta igualdad es válida porque las derivadas parciales de $f$ son continuas en todo $\mathbb{R}^2$.

\vspace{1cm}

\subsection*{Derivadas Parciales de Segundo Orden}
% =====================================================

% =====================================================
\subsection*{Ejercicio 1}
\( f(x,y)=\dfrac{x^4 y^3}{6}-\dfrac{2xy^6}{3} \)

% --- Parcial respecto a x ---
\[
\frac{\partial}{\partial x}\left(\frac{x^4 y^3}{6}\right)
= \frac{4x^3 y^3}{6}
\]

\[
\frac{\partial}{\partial x}\left(\frac{2xy^6}{3}\right)
= \frac{2y^6}{3}
\]

\[
\boxed{
\frac{\partial f}{\partial x}
= \frac{4x^3 y^3}{6} - \frac{2y^6}{3}
}
\]

% --- Parcial respecto a y ---
\[
\frac{\partial}{\partial y}\left(\frac{x^4 y^3}{6}\right)
= \frac{3x^4 y^2}{6}
= \frac{1}{2}x^4 y^2
\]

\[
\frac{\partial}{\partial y}\left(\frac{2xy^6}{3}\right)
= 4xy^5
\]

\[
\boxed{
\frac{\partial f}{\partial y}
= \frac{1}{2}x^4 y^2 - 4xy^5
}
\]

% --- Segunda derivada respecto a x ---
\[
\frac{\partial}{\partial x}\left(\frac{4x^3 y^3}{6}\right)
= \frac{12x^2 y^3}{6}
= 2x^2 y^3
\]

\[
\frac{\partial}{\partial x}\left(\frac{2y^6}{3}\right)=0
\]

\[
\boxed{
\frac{\partial^2 f}{\partial x^2} = 2x^2 y^3
}
\]

% --- Segunda derivada respecto a y ---
\[
\frac{\partial}{\partial y}\left(\frac{1}{2}x^4 y^2\right)
= x^4 y
\]

\[
\frac{\partial}{\partial y}(4xy^5)
= 20xy^4
\]

\[
\boxed{
\frac{\partial^2 f}{\partial y^2} = x^4 y - 20xy^4
}
\]

% --- Parcial mixta xy ---
\[
\frac{\partial}{\partial y}\left(\frac{4x^3 y^3}{6}\right)
= \frac{12x^3 y^2}{6}
= 2x^3 y^2
\]

\[
\frac{\partial}{\partial y}\left(\frac{-2y^6}{3}\right)
= -4y^5
\]

\[
\boxed{
\frac{\partial^2 f}{\partial y\partial x}
= 2x^3 y^2 - 4y^5
}
\]

% --- Parcial mixta yx ---
\[
\frac{\partial}{\partial x}\left(\frac{1}{2}x^4 y^2\right)
= 2x^3 y^2
\]

\[
\frac{\partial}{\partial x}(-4xy^5)
= -4y^5
\]

\[
\boxed{
\frac{\partial^2 f}{\partial x\partial y}
= 2x^3 y^2 - 4y^5
}
\]

% =====================================================
\subsection*{Ejercicio 2}
\( f(x,y)=6x^2\cos(y^3) \)

% --- df/dx ---
\[
\frac{\partial}{\partial x}(6x^2)=12x
\]

\[
\boxed{
\frac{\partial f}{\partial x}=12x\cos(y^3)
}
\]

% --- df/dy ---
\[
\frac{\partial}{\partial y}(\cos(y^3))
=-\sin(y^3)\cdot 3y^2
\]

\[
\boxed{
\frac{\partial f}{\partial y}
= -18x^2 y^2 \sin(y^3)
}
\]

% --- d2f/dx2 ---
\[
\frac{\partial}{\partial x}(12x\cos(y^3))
= 12\cos(y^3)
\]

\[
\boxed{
\frac{\partial^2 f}{\partial x^2}=12\cos(y^3)
}
\]

% --- d2f/dy2 ---
\[
\frac{\partial}{\partial y}\left(-18x^2 y^2\sin(y^3)\right)
= -18x^2\left(2y\sin(y^3)+y^2\cos(y^3)\cdot 3y^2\right)
\]

\[
\boxed{
\frac{\partial^2 f}{\partial y^2}
= -36x^2 y\sin(y^3) - 54x^2 y^4 \cos(y^3)
}
\]

% --- d2f/dxdy ---
\[
\frac{\partial}{\partial y}(12x\cos(y^3))
= 12x(-\sin(y^3)\cdot 3y^2)
\]

\[
\boxed{
\frac{\partial^2 f}{\partial y\partial x}
= -36xy^2 \sin(y^3)
}
\]

% --- d2f/dydx ---
\[
\frac{\partial}{\partial x}(-18x^2 y^2\sin(y^3))
= -36xy^2\sin(y^3)
\]

\[
\boxed{
\frac{\partial^2 f}{\partial x\partial y}
= -36xy^2\sin(y^3)
}
\]

% ============================================================
\subsection*{Ejercicio 3}
Sea la función:
\[
f(x,y)=\cos(x^4 + y^7)
\]

% ============================================================
\subsubsection*{Primera derivada respecto a $x$}

\[
\frac{\partial f}{\partial x}
= -\sin(x^4 + y^7)\cdot 4x^3
\]

\[
= -4x^3 \sin(x^4 + y^7)
\]

% ============================================================
\subsubsection*{Segunda derivada respecto a $x$}

Aplicamos regla del producto:

\[
\frac{\partial^2 f}{\partial x^2}
= -12x^2 \sin(x^4 + y^7)
- 6x^6 \cos(x^4 + y^7)
\]

\[
\boxed{
\frac{\partial^2 f}{\partial x^2}
= -12x^2 \sin(x^4 + y^7)
- 6x^6 \cos(x^4 + y^7)
}
\]

% ============================================================
\subsubsection*{Primera derivada respecto a $y$}

\[
\frac{\partial f}{\partial y}
= -\sin(x^4 + y^7)\cdot 7y^6
\]

\[
= -7y^6 \sin(x^4 + y^7)
\]

% ============================================================
\subsubsection*{Segunda derivada mixta $\frac{\partial^2 f}{\partial x\partial y}$}

\[
\frac{\partial^2 f}{\partial x\partial y}
= -4x^3 \cos(x^4 + y^7)\cdot 7y^6
\]

\[
= -28 x^3 y^6 \cos(x^4 + y^7)
\]

\[
\boxed{
\frac{\partial^2 f}{\partial x\partial y}
= -28 x^3 y^6 \cos(x^4 + y^7)
}
\]

% ============================================================
\subsubsection*{Segunda derivada mixta $\frac{\partial^2 f}{\partial y\partial x}$}

\[
\frac{\partial^2 f}{\partial y\partial x}
= -7y^6 \cos(x^4 + y^7)\cdot 4x^3
\]

\[
= -28 x^3 y^6 \cos(x^4 + y^7)
\]

\[
\boxed{
\frac{\partial^2 f}{\partial y\partial x}
= -28 x^3 y^6 \cos(x^4 + y^7)
}
\]

% ============================================================
\subsubsection*{Segunda derivada respecto a $y$}

Aplicando producto y regla de la cadena:

\[
\frac{\partial^2 f}{\partial y^2}
= -42y^5 \sin(x^4 + y^7)
+ 49y^{12} \cos(x^4 + y^7)
\]

\[
\boxed{
\frac{\partial^2 f}{\partial y^2}
= -42y^5 \sin(x^4 + y^7)
+ 49y^{12} \cos(x^4 + y^7)
}
\]

% =====================================================
\subsection*{Ejercicio 4}
\( f(x,y)=(30 - 6x + 2y)(4x - 5) \)

Expande:
\[
f(x,y)=120x -150 -24x^2 + 30x + 8xy -10y
\]

% df/dx
\[
\boxed{
\frac{\partial f}{\partial x}
= 48x + 150 + 8y
}
\]

% df/dy
\[
\boxed{
\frac{\partial f}{\partial y}
= 8x - 10
}
\]

% d2f/dx2
\[
\boxed{
\frac{\partial^2 f}{\partial x^2}=-48
}
\]

% d2f/dy2
\[
\boxed{
\frac{\partial^2 f}{\partial y^2}=0
}
\]

% d2f/dxdy
\[
\boxed{
\frac{\partial^2 f}{\partial y\partial x}
= 8
}
\]

% d2f/dydx
\[
\boxed{
\frac{\partial^2 f}{\partial x\partial y}
= 8
}
\]

% =====================================================
\subsection*{Ejercicio 5}
\( f(x,y)=x^3(4x+5y) \)

\[
f(x,y)=4x^4 + 5x^3 y
\]

% df/dx
\[
\frac{\partial f}{\partial x}
= 16x^3 + 15x^2 y
\]

% df/dy
\[
\frac{\partial f}{\partial y}
= 5x^3
\]

% d2f/dx2
\[
\frac{\partial^2 f}{\partial x^2}
= 48x^2 + 30xy
\]

% d2f/dy2
\[
\frac{\partial^2 f}{\partial y^2}=0
\]

% d2f/dxdy
\[
\frac{\partial^2 f}{\partial y\partial x}
= 15x^2
\]

% d2f/dydx
\[
\frac{\partial^2 f}{\partial x\partial y}
= 15x^2
\]

% =====================================================
\subsection*{Ejercicio 6}
\( f(x,y)=y^4(2x+3y) \)

\[
f(x,y)=2xy^4 + 3y^5
\]

% df/dx
\[
\frac{\partial f}{\partial x}=2y^4
\]

% df/dy
\[
\frac{\partial f}{\partial y}
= 8xy^3 + 15y^4
\]

% d2f/dx2
\[
\frac{\partial^2 f}{\partial x^2}=0
\]

% d2f/dy2
\[
\frac{\partial^2 f}{\partial y^2}
= 24xy^2 + 60y^3
\]

% d2f/dxdy
\[
\frac{\partial^2 f}{\partial y\partial x}
= 8y^3
\]

% d2f/dydx
\[
\frac{\partial^2 f}{\partial x\partial y}
= 8y^3
\]

% ============================================================
\subsection*{Ejercicio 7}
% ============================================================

Sea la función:
\[
f(x,y)=\frac{x^2}{x^2 + y^2}
\]

Definimos:
\[
u = x^2,
\qquad
v = x^2 + y^2
\]

% ============================================================
\subsubsection*{Primera derivada respecto a $x$}

Aplicamos regla del cociente:
\[
\frac{\partial f}{\partial x}
= \frac{2x(x^2+y^2) - x^2(2x)}{(x^2+y^2)^2}
\]

Desarrollamos:
\[
2x(x^2+y^2) = 2x^3 + 2xy^2
\]
\[
x^2(2x)=2x^3
\]

Sustituimos:
\[
\frac{\partial f}{\partial x}
= \frac{2x^3 + 2xy^2 - 2x^3}{(x^2+y^2)^2}
\]

Eliminamos términos:
\[
2x^3 - 2x^3 = 0
\]

Resultado:
\[
\boxed{
\frac{\partial f}{\partial x}
= \frac{2x y^2}{(x^2+y^2)^2}
}
\]

% ============================================================
\subsubsection*{Primera derivada respecto a $y$}

Aplicamos regla del cociente:
\[
\frac{\partial f}{\partial y}
= \frac{0(x^2+y^2) - x^2(2y)}{(x^2+y^2)^2}
\]

Simplificando:
\[
\frac{\partial f}{\partial y}
= -\frac{2x^2 y}{(x^2+y^2)^2}
\]

\[
\boxed{
\frac{\partial f}{\partial y}
= -\frac{2x^2 y}{(x^2+y^2)^2}
}
\]

% ============================================================
\subsubsection*{Segunda derivada respecto a $x$}

Partimos de:
\[
\frac{\partial f}{\partial x}
= \frac{2xy^2}{(x^2+y^2)^2}
\]

Sea:
\[
u_1 = 2xy^2,
\qquad
v_1 = (x^2+y^2)^2
\]

Aplicamos regla del cociente:
\[
\frac{\partial^2 f}{\partial x^2}
= \frac{(2y^2)(x^2+y^2)^2 - (2xy^2)(2)(x^2+y^2)(2x)}{(x^2+y^2)^4}
\]

Desarrollando:
\[
= \frac{2y^2(x^2+y^2)^2}{(x^2+y^2)^4}
-
\frac{8x^2 y^2 (x^2+y^2)}{(x^2+y^2)^4}
\]

Factorizamos:
\[
= \frac{2y^2(x^2+y^2) - 8x^2 y^2}{(x^2+y^2)^3}
\]

Resultado:
\[
\boxed{
\frac{\partial^2 f}{\partial x^2}
= -\frac{6x^2 y^2 + 2y^2}{(x^2+y^2)^3}
}
\]

% ============================================================
\subsubsection*{Derivada mixta $\frac{\partial^2 f}{\partial x \partial y}$}

Partimos de:
\[
\frac{\partial f}{\partial x}
= \frac{2xy^2}{(x^2+y^2)^2}
\]

Aplicamos regla del cociente respecto a $y$:
\[
\frac{\partial^2 f}{\partial y\partial x}
= \frac{
4xy(x^2+y^2)^2 - (2xy^2)(2)(x^2+y^2)(2y)
}{(x^2+y^2)^4}
\]

Desarrollando:
\[
= \frac{
4xy(x^2+y^2)^2 - 8xy^3(x^2+y^2)
}{(x^2+y^2)^4}
\]

Simplificando:
\[
= \frac{
4xy(x^2+y^2 - 2y^2)
}{(x^2+y^2)^3}
\]

Resultado:
\[
\boxed{
\frac{\partial^2 f}{\partial y\partial x}
= \frac{4xy(x^2 - y^2)}{(x^2+y^2)^3}
}
\]

% ============================================================
\subsubsection*{Derivada mixta $\frac{\partial^2 f}{\partial y \partial x}$}

Partimos de:
\[
\frac{\partial f}{\partial y}
= -\frac{2x^2 y}{(x^2+y^2)^2}
\]

Aplicamos regla del cociente respecto a $x$:
\[
\frac{\partial^2 f}{\partial x\partial y}
= \frac{
(x^2+y^2)^2(-4xy) + 8x^3y(x^2+y^2)
}{(x^2+y^2)^4}
\]

Después de factorizar:
\[
= 4xy\,\frac{x^2 - y^2}{(x^2+y^2)^3}
\]

\[
\boxed{
\frac{\partial^2 f}{\partial x\partial y}
= \frac{4xy(x^2 - y^2)}{(x^2+y^2)^3}
}
\]

% ============================================================
\subsubsection*{Segunda derivada respecto a $y$}

Partimos de:
\[
\frac{\partial f}{\partial y}
= -\frac{2x^2 y}{(x^2+y^2)^2}
\]

Aplicamos regla del cociente:
\[
\frac{\partial^2 f}{\partial y^2}
= \frac{
(x^2+y^2)^2(-2x^2) - (-2x^2y)(2)(x^2+y^2)(2y)
}{(x^2+y^2)^4}
\]

Desarrollando:
\[
= \frac{
-2x^2(x^2+y^2)^2 + 8x^2 y^2(x^2+y^2)
}{(x^2+y^2)^4}
\]

Simplificando:
\[
= \frac{
2x^2(3y^2 - x^2)
}{(x^2+y^2)^3}
\]

\[
\boxed{
\frac{\partial^2 f}{\partial y^2}
= \frac{2x^2(3y^2 - x^2)}{(x^2+y^2)^3}
}
\]

\begin{figure}[H]
    \centering
    \includegraphics[width=0.7\textwidth]{img/2_2_imagen_derivadas_segundo_orden.jpg}
    \caption{Derivadas Parciales de Segundo Orden}
\end{figure}

La función $g(x,y) = \sin(x) + \cos(y)$ genera una superficie tipo "caja de huevos" que permite visualizar concavidades y puntos de silla. Las derivadas parciales de segundo orden, como $\frac{\partial^2 f}{\partial x^2}$ y $\frac{\partial^2 f}{\partial y^2}$, determinan la curvatura de la superficie en diferentes direcciones, siendo fundamentales para identificar máximos, mínimos y puntos de silla.

\section{Reglas de Derivación}

Al calcular derivadas parciales, se aplican las mismas reglas fundamentales del cálculo de una variable, con la importante consideración de que todas las variables excepto aquella respecto a la cual se deriva deben tratarse como constantes.

\subsection{Regla de la Potencia}

Para una función que contiene una potencia de una variable, la regla de la potencia se aplica de la siguiente manera:

Si $f(x, y) = x^n g(y)$, donde $n$ es una constante y $g(y)$ es una función de $y$, entonces:
\begin{equation}
\frac{\partial f}{\partial x} = n x^{n-1} g(y)
\label{eq:regla_potencia}
\end{equation}

Al derivar respecto a $x$, $g(y)$ se trata como una constante multiplicativa. De manera análoga, al derivar respecto a $y$, $x^n$ actúa como constante.

\textbf{Ejemplo:} Para $f(x, y) = x^4 y^3$:
\begin{align}
\frac{\partial f}{\partial x} &= 4x^3 y^3 \quad \text{(tratando } y^3 \text{ como constante)} \notag\\
\frac{\partial f}{\partial y} &= x^4 \cdot 3y^2 = 3x^4 y^2 \quad \text{(tratando } x^4 \text{ como constante)} \notag
\end{align}

\subsection{Regla del Producto}

Cuando una función es el producto de dos funciones que dependen de la variable de derivación, se aplica la regla del producto:

Si $f(x, y) = u(x, y) \cdot v(x, y)$, entonces:
\begin{equation}
\frac{\partial f}{\partial x} = \frac{\partial u}{\partial x} \cdot v + u \cdot \frac{\partial v}{\partial x}
\label{eq:regla_producto}
\end{equation}

\textbf{Ejemplo:} Para $f(x, y) = x^2 e^y$:
\begin{align}
\frac{\partial f}{\partial x} &= 2x \cdot e^y + x^2 \cdot 0 = 2x e^y \notag\\
\frac{\partial f}{\partial y} &= 0 \cdot e^y + x^2 \cdot e^y = x^2 e^y \notag
\end{align}

Note que al derivar $e^y$ respecto a $x$, el resultado es cero porque $e^y$ es constante respecto a $x$.

\subsection{Regla de la Cadena}

La regla de la cadena es fundamental cuando se trabaja con composiciones de funciones. Para funciones de varias variables, la regla de la cadena toma varias formas según el contexto.

\textbf{Caso 1: Composición con una función de una variable}

Si $z = f(g(x, y))$, donde $g$ es una función de $x$ e $y$, y $f$ es una función de una variable, entonces:
\begin{equation}
\frac{\partial z}{\partial x} = f'(g(x, y)) \cdot \frac{\partial g}{\partial x}
\label{eq:regla_cadena_caso1}
\end{equation}

\textbf{Ejemplo:} Para $f(x, y) = (x^2 + y^2)^3$:

Aquí $g(x, y) = x^2 + y^2$ y la función exterior es $u^3$. Aplicando la regla de la cadena:
\begin{align}
\frac{\partial f}{\partial x} &= 3(x^2 + y^2)^2 \cdot \frac{\partial}{\partial x}(x^2 + y^2) \notag\\
&= 3(x^2 + y^2)^2 \cdot 2x = 6x(x^2 + y^2)^2 \notag
\end{align}

\textbf{Caso 2: Regla de la cadena general}

Si $z = f(u, v)$ donde $u = u(x, y)$ y $v = v(x, y)$, entonces:
\begin{equation}
\frac{\partial z}{\partial x} = \frac{\partial f}{\partial u} \cdot \frac{\partial u}{\partial x} + \frac{\partial f}{\partial v} \cdot \frac{\partial v}{\partial x}
\label{eq:regla_cadena_general}
\end{equation}

Esta forma de la regla de la cadena es especialmente importante en cambios de coordenadas y en el cálculo de derivadas direccionales.

\subsection{Regla del Cociente}

Para el cociente de dos funciones $f(x, y) = \frac{u(x, y)}{v(x, y)}$, donde $v(x, y) \neq 0$:
\begin{equation}
\frac{\partial f}{\partial x} = \frac{\frac{\partial u}{\partial x} \cdot v - u \cdot \frac{\partial v}{\partial x}}{v^2}
\label{eq:regla_cociente}
\end{equation}

\textbf{Ejemplo:} Para $f(x, y) = \frac{x^2}{x + y}$:
\begin{align}
\frac{\partial f}{\partial x} &= \frac{2x(x + y) - x^2 \cdot 1}{(x + y)^2} = \frac{2x^2 + 2xy - x^2}{(x + y)^2} = \frac{x^2 + 2xy}{(x + y)^2} \notag
\end{align}

\begin{figure}[H]
    \centering
    \includegraphics[width=0.7\textwidth]{img/2_3_imagen_reglas_derivacion.jpg}
    \caption{Reglas de Derivación}
\end{figure}

La superficie compleja $h(x,y) = x \cdot e^{-x^2 - y^2}$ presenta picos y valles que requieren la aplicación combinada de la regla del producto y la regla de la cadena. Al calcular $\frac{\partial h}{\partial x}$, primero se aplica la regla del producto: $\frac{\partial}{\partial x}[x \cdot e^{-x^2 - y^2}] = e^{-x^2 - y^2} + x \cdot \frac{\partial}{\partial x}[e^{-x^2 - y^2}]$, y luego la regla de la cadena para la función exponencial.

\section{Historia de las Derivadas Parciales}

\subsection{Orígenes del Cálculo Multivariable}

El desarrollo histórico de las derivadas parciales está íntimamente ligado a la evolución del cálculo y de la física matemática. Aunque el cálculo diferencial fue desarrollado en el siglo XVII por Isaac Newton y Gottfried Wilhelm Leibniz, sus trabajos se enfocaron principalmente en funciones de una sola variable. La necesidad de estudiar funciones de múltiples variables surgió naturalmente de problemas físicos y geométricos que no podían ser resueltos con las herramientas del cálculo univariable.

\subsection{Desarrollo en el Siglo XVIII}

Durante el siglo XVIII, varios matemáticos comenzaron a trabajar con funciones de múltiples variables sin tener aún una notación o teoría formal completamente desarrollada:

\textbf{Jean Le Rond d'Alembert (1746):} En su trabajo sobre la vibración de cuerdas, d'Alembert derivó lo que ahora conocemos como la ecuación de onda, una ecuación diferencial parcial que involucra derivadas respecto al tiempo y al espacio. Este fue uno de los primeros usos explícitos de derivadas parciales en física matemática.

\textbf{Leonhard Euler (1755):} Euler desarrolló métodos para trabajar con funciones de varias variables y fue uno de los primeros en usar notación específica para distinguir entre diferentes tipos de derivadas, aunque la notación moderna aún no existía.

\textbf{Joseph-Louis Lagrange (1788):} Introdujo el cálculo de variaciones de manera sistemática, trabajando extensivamente con funciones de múltiples variables y sus derivadas.

\subsection{Consolidación de la Notación}

La notación moderna para derivadas parciales tiene su propia historia fascinante:

\textbf{Marquis de Condorcet (1770):} Fue uno de los primeros en usar el símbolo $\partial$ para denotar diferencias parciales, aunque su uso no fue consistente ni ampliamente adoptado en ese momento.

\textbf{Adrien-Marie Legendre (1786):} Adoptó brevemente el símbolo $\partial$, pero posteriormente lo abandonó en favor de otras notaciones.

\textbf{Carl Gustav Jacob Jacobi (1841):} Reintrodujo y estandarizó el uso del símbolo $\partial$ para derivadas parciales. Jacobi también contribuyó significativamente a la teoría con su trabajo sobre determinantes jacobianos y transformaciones de coordenadas.

\subsection{Formalización en el Siglo XIX}

El siglo XIX vio la formalización rigurosa del análisis matemático:

\textbf{Augustin-Louis Cauchy (1820s):} Proporcionó las primeras definiciones rigurosas de límites y continuidad, estableciendo las bases para una teoría formal de derivadas parciales.

\textbf{Karl Weierstrass (1860s):} Desarrolló una teoría aún más rigurosa del análisis, incluyendo condiciones precisas para la existencia y continuidad de derivadas parciales.

\textbf{Bernhard Riemann (1850s):} Sus trabajos en geometría diferencial y análisis complejo ampliaron enormemente el alcance de las aplicaciones de derivadas parciales.

\subsection{Siglo XX y Aplicaciones Modernas}

En el siglo XX, la teoría de derivadas parciales se convirtió en una rama central de las matemáticas, con desarrollos en:

\begin{itemize}
    \item Teoría de ecuaciones en derivadas parciales (EDP)
    \item Análisis funcional y espacios de Hilbert
    \item Métodos numéricos y computacionales
    \item Aplicaciones en mecánica cuántica, relatividad y teoría de campos
    \item Optimización y teoría de control
\end{itemize}

Hoy en día, las derivadas parciales son fundamentales en prácticamente todas las áreas de la ciencia, ingeniería, economía y ciencia de datos, representando una de las herramientas matemáticas más versátiles y poderosas disponibles.

\section{Aplicaciones}

Las derivadas parciales encuentran aplicación en una amplia variedad de campos, desde las ciencias físicas hasta las ciencias sociales y la tecnología moderna. A continuación se presentan algunas de las aplicaciones más relevantes.

\subsection{Física e Ingeniería}

\subsubsection{Transferencia de Calor}

La ecuación del calor es una de las ecuaciones en derivadas parciales más importantes en física e ingeniería:
\begin{equation}
\frac{\partial T}{\partial t} = \alpha \left(\frac{\partial^2 T}{\partial x^2} + \frac{\partial^2 T}{\partial y^2} + \frac{\partial^2 T}{\partial z^2}\right)
\label{eq:ecuacion_calor}
\end{equation}

donde $T(x, y, z, t)$ es la temperatura, $t$ es el tiempo, y $\alpha$ es la difusividad térmica del material. Esta ecuación describe cómo se distribuye el calor en un cuerpo sólido a lo largo del tiempo. Las derivadas parciales espaciales ($\frac{\partial^2 T}{\partial x^2}$, etc.) describen la curvatura de la distribución de temperatura, mientras que $\frac{\partial T}{\partial t}$ representa la tasa de cambio temporal.

En ingeniería, esta ecuación se utiliza para diseñar sistemas de enfriamiento, analizar aislamiento térmico en edificios, y optimizar procesos industriales que involucran transferencia de calor.

\subsubsection{Mecánica de Fluidos}

Las ecuaciones de Navier-Stokes, que gobiernan el movimiento de fluidos, son sistemas complejos de ecuaciones en derivadas parciales:
\begin{equation}
\rho \left(\frac{\partial \mathbf{v}}{\partial t} + (\mathbf{v} \cdot \nabla)\mathbf{v}\right) = -\nabla p + \mu \nabla^2 \mathbf{v} + \mathbf{f}
\label{eq:navier_stokes}
\end{equation}

donde $\mathbf{v}$ es el campo de velocidad, $p$ es la presión, $\rho$ es la densidad, $\mu$ es la viscosidad dinámica, y $\mathbf{f}$ representa las fuerzas externas. Estas ecuaciones son fundamentales para el diseño de aeronaves, turbinas, sistemas de tuberías, y predicción meteorológica.

\subsubsection{Electromagnetismo}

Las ecuaciones de Maxwell, que describen todos los fenómenos electromagnéticos, están formuladas en términos de derivadas parciales:
\begin{align}
\nabla \cdot \mathbf{E} &= \frac{\rho}{\epsilon_0} \label{eq:maxwell1}\\
\nabla \cdot \mathbf{B} &= 0 \label{eq:maxwell2}\\
\nabla \times \mathbf{E} &= -\frac{\partial \mathbf{B}}{\partial t} \label{eq:maxwell3}\\
\nabla \times \mathbf{B} &= \mu_0 \mathbf{J} + \mu_0 \epsilon_0 \frac{\partial \mathbf{E}}{\partial t} \label{eq:maxwell4}
\end{align}

donde $\mathbf{E}$ es el campo eléctrico, $\mathbf{B}$ es el campo magnético, $\rho$ es la densidad de carga, y $\mathbf{J}$ es la densidad de corriente. Estas ecuaciones son la base del diseño de antenas, circuitos de alta frecuencia, dispositivos de comunicación inalámbrica y tecnología de radares.

\subsection{Geometría y Topografía}

En geometría diferencial, las derivadas parciales se utilizan para calcular propiedades de superficies como la curvatura y las normales. Para una superficie parametrizada por $\mathbf{r}(u, v)$, el vector normal se calcula mediante:
\begin{equation}
\mathbf{n} = \frac{\partial \mathbf{r}}{\partial u} \times \frac{\partial \mathbf{r}}{\partial v}
\label{eq:vector_normal}
\end{equation}

En topografía y cartografía, las curvas de nivel que estudiamos anteriormente se definen implícitamente mediante derivadas parciales. La pendiente del terreno en cualquier dirección se puede calcular usando el gradiente:
\begin{equation}
\nabla h = \left(\frac{\partial h}{\partial x}, \frac{\partial h}{\partial y}\right)
\label{eq:gradiente_altura}
\end{equation}

donde $h(x, y)$ representa la altitud en cada punto.

\subsection{Economía y Finanzas}

\subsubsection{Teoría del Consumidor}

En economía, la función de utilidad $U(x_1, x_2, \ldots, x_n)$ representa la satisfacción de un consumidor al consumir cantidades $x_i$ de diferentes bienes. Las derivadas parciales de la función de utilidad se conocen como utilidades marginales:
\begin{equation}
\frac{\partial U}{\partial x_i} = \text{Utilidad marginal del bien } i
\label{eq:utilidad_marginal}
\end{equation}

La tasa marginal de sustitución entre dos bienes se define como el cociente de sus utilidades marginales:
\begin{equation}
\text{TMS}_{12} = -\frac{\partial U / \partial x_1}{\partial U / \partial x_2}
\label{eq:tms}
\end{equation}

Este concepto es fundamental para entender las decisiones de consumo y la formación de precios en mercados competitivos.

\subsubsection{Ecuación de Black-Scholes}

En finanzas cuantitativas, la valoración de opciones se basa en la ecuación de Black-Scholes, una ecuación en derivadas parciales:
\begin{equation}
\frac{\partial V}{\partial t} + \frac{1}{2}\sigma^2 S^2 \frac{\partial^2 V}{\partial S^2} + rS\frac{\partial V}{\partial S} - rV = 0
\label{eq:black_scholes}
\end{equation}

donde $V(S, t)$ es el precio de la opción, $S$ es el precio del activo subyacente, $\sigma$ es la volatilidad, y $r$ es la tasa de interés libre de riesgo. Las derivadas parciales en esta ecuación tienen interpretaciones financieras específicas conocidas como "las griegas": Delta ($\frac{\partial V}{\partial S}$), Gamma ($\frac{\partial^2 V}{\partial S^2}$), Theta ($\frac{\partial V}{\partial t}$), entre otras.

\subsection{Inteligencia Artificial y Machine Learning}

\subsubsection{Redes Neuronales y Retropropagación}

En el entrenamiento de redes neuronales, el algoritmo de retropropagación (backpropagation) se basa fundamentalmente en el cálculo de derivadas parciales mediante la regla de la cadena. Para una función de pérdida $L(\mathbf{w})$ que depende de los pesos $\mathbf{w}$ de la red, el gradiente:
\begin{equation}
\nabla L = \left(\frac{\partial L}{\partial w_1}, \frac{\partial L}{\partial w_2}, \ldots, \frac{\partial L}{\partial w_n}\right)
\label{eq:gradiente_perdida}
\end{equation}

indica la dirección de mayor crecimiento de la función de pérdida. El algoritmo de descenso del gradiente utiliza esta información para actualizar los pesos:
\begin{equation}
w_i^{(k+1)} = w_i^{(k)} - \eta \frac{\partial L}{\partial w_i}
\label{eq:descenso_gradiente}
\end{equation}

donde $\eta$ es la tasa de aprendizaje. Este proceso iterativo minimiza la función de pérdida, mejorando el rendimiento del modelo.

\subsubsection{Optimización}

En problemas de optimización multivariable, las derivadas parciales son esenciales para encontrar puntos críticos. Para encontrar los extremos de una función $f(x, y)$, se resuelve el sistema:
\begin{align}
\frac{\partial f}{\partial x} &= 0 \label{eq:extremo_x}\\
\frac{\partial f}{\partial y} &= 0 \label{eq:extremo_y}
\end{align}

La naturaleza del punto crítico (máximo, mínimo o punto de silla) se determina mediante el análisis de la matriz Hessiana y el cálculo del determinante:
\begin{equation}
D = \frac{\partial^2 f}{\partial x^2} \cdot \frac{\partial^2 f}{\partial y^2} - \left(\frac{\partial^2 f}{\partial x \partial y}\right)^2
\label{eq:determinante_hessiano}
\end{equation}

Este criterio se utiliza ampliamente en algoritmos de optimización para problemas de regresión, clasificación, y aprendizaje por refuerzo.

\subsection{Visión por Computadora}

En procesamiento de imágenes, las derivadas parciales se utilizan para detectar bordes y características. El operador de Sobel, por ejemplo, aproxima el gradiente de la intensidad de la imagen:
\begin{equation}
\nabla I = \left(\frac{\partial I}{\partial x}, \frac{\partial I}{\partial y}\right)
\label{eq:gradiente_imagen}
\end{equation}

La magnitud del gradiente $|\nabla I|$ es alta en los bordes de los objetos, lo que permite su detección automática. Esta técnica es fundamental en reconocimiento de patrones, seguimiento de objetos y reconstrucción 3D.

\subsection{Biología y Medicina}

En biología matemática, las derivadas parciales modelan fenómenos como la difusión de sustancias químicas, el crecimiento de poblaciones en medios heterogéneos, y la propagación de enfermedades. La ecuación de reacción-difusión de Fisher-KPP:
\begin{equation}
\frac{\partial u}{\partial t} = D\frac{\partial^2 u}{\partial x^2} + ru(1 - u)
\label{eq:fisher_kpp}
\end{equation}

describe la dispersión espacial de una población con crecimiento logístico, donde $u(x, t)$ es la densidad de población, $D$ es el coeficiente de difusión, y $r$ es la tasa de crecimiento.

En imagenología médica, las derivadas parciales se utilizan en algoritmos de reconstrucción tomográfica, análisis de resonancias magnéticas, y modelado de flujo sanguíneo.

\subsection{Síntesis}

\begin{figure}[H]
    \centering
    \includegraphics[width=0.7\textwidth]{img/2_5_imagen_aplicaciones_derivadas.jpg}
    \caption{Aplicaciones de las Derivadas Parciales}
\end{figure}

Las aplicaciones de las derivadas parciales son prácticamente ilimitadas en la ciencia y la ingeniería modernas. Desde la predicción del clima hasta el diseño de chips de computadora, desde la optimización de carteras de inversión hasta el desarrollo de fármacos, las derivadas parciales proporcionan el lenguaje matemático necesario para describir, analizar y predecir el comportamiento de sistemas complejos que dependen de múltiples variables. Su dominio es esencial para cualquier profesional que trabaje con modelos matemáticos del mundo real.

% ==================== UNIDAD III ====================
\chapter{Unidad III: Integrales}

\section{Integrales Definidas}

La integral definida es uno de los conceptos fundamentales del cálculo integral y representa la acumulación de cantidades continuas en un intervalo específico. Este concepto permite calcular áreas bajo curvas, volúmenes, trabajo, distancias y muchas otras magnitudes físicas y geométricas.

\subsection{Definición Formal de la Integral Definida}

La integral definida de una función $f(x)$ en un intervalo $[a, b]$ se define como el límite de una suma de Riemann cuando el número de subdivisiones tiende a infinito. Matemáticamente, se expresa como:

\begin{equation}
\int_a^b f(x)\, dx = \lim_{n \to \infty} \sum f(x_i^*) \Delta x
\label{eq:integral_definida}
\end{equation}

Donde:
\begin{itemize}
    \item $a$ y $b$ son los límites de integración ($a$ es el límite inferior y $b$ es el límite superior)
    \item $f(x)$ es la función a integrar, llamada integrando
    \item $dx$ representa el diferencial de $x$
    \item El resultado es un número real que representa el área neta entre la curva y el eje $x$
\end{itemize}

La definición rigurosa mediante sumas de Riemann consiste en dividir el intervalo $[a, b]$ en $n$ subintervalos de ancho $\Delta x = (b - a)/n$, evaluar la función en puntos representativos de cada subintervalo, multiplicar por el ancho y sumar todos los productos. Cuando el número de subdivisiones tiende a infinito, esta suma se aproxima al valor exacto de la integral.

\subsection{Historia del Desarrollo de las Integrales}

El concepto de integral tiene raíces profundas en la historia de las matemáticas, con contribuciones que se remontan a la antigua Grecia. Arquímedes empleó el método de exhausción para calcular áreas y volúmenes, anticipando las ideas del cálculo integral. Sin embargo, el desarrollo formal del cálculo integral ocurrió durante el siglo XVII.

\textbf{Isaac Newton (1660s-1670s):} Desarrolló el cálculo de fluxiones, que incluía técnicas para calcular áreas bajo curvas. Newton entendió la relación inversa entre diferenciación e integración, aunque su notación era diferente a la moderna.

\textbf{Gottfried Wilhelm Leibniz (1670s-1680s):} De manera independiente, creó el cálculo diferencial e integral con una notación más intuitiva y práctica. Leibniz introdujo el símbolo $\int$ (una ``S'' estilizada que representa ``suma'') para denotar la integración, notación que se mantiene hasta hoy.

\textbf{Teorema Fundamental del Cálculo:} Tanto Newton como Leibniz establecieron la conexión fundamental entre derivadas e integrales, demostrando que la integración y la diferenciación son operaciones inversas.

\textbf{Augustin-Louis Cauchy (Siglo XIX):} Formalizó el concepto de integral mediante sumas, proporcionando una definición más rigurosa basada en límites.

\textbf{Bernhard Riemann (1854):} Desarrolló la integral de Riemann, que es la definición estándar utilizada en cálculo básico. Riemann introdujo las sumas que llevan su nombre y estableció condiciones precisas para la integrabilidad de funciones.

\textbf{Henri Lebesgue (Siglo XX):} Generalizó el concepto de integral para funciones más complejas, creando la teoría de la medida y la integral de Lebesgue, fundamental en el análisis moderno.

\subsection{Propiedades Fundamentales de las Integrales Definidas}

Las integrales definidas poseen propiedades algebraicas que facilitan su cálculo y aplicación:

\begin{itemize}
    \item \textbf{Linealidad:} La integral de una suma es la suma de las integrales, y se puede sacar una constante fuera de la integral.
    \begin{equation}
    \int_a^b [cf(x) + g(x)]\, dx = c\int_a^b f(x)\, dx + \int_a^b g(x)\, dx
    \label{eq:linealidad_integral}
    \end{equation}

    \item \textbf{Aditividad respecto al intervalo:} Si $c$ está entre $a$ y $b$, entonces:
    \begin{equation}
    \int_a^b f(x)\, dx = \int_a^c f(x)\, dx + \int_c^b f(x)\, dx
    \label{eq:aditividad_intervalo}
    \end{equation}

    \item \textbf{Inversión de límites:} Intercambiar los límites de integración cambia el signo de la integral.
    \begin{equation}
    \int_a^b f(x)\, dx = -\int_b^a f(x)\, dx
    \label{eq:inversion_limites}
    \end{equation}

    \item \textbf{Integral de valor cero:} Si los límites son iguales, la integral es cero.
    \begin{equation}
    \int_a^a f(x)\, dx = 0
    \label{eq:integral_cero}
    \end{equation}

    \item \textbf{Comparación:} Si $f(x) \leq g(x)$ para todo $x$ en $[a, b]$, entonces:
    \begin{equation}
    \int_a^b f(x)\, dx \leq \int_a^b g(x)\, dx
    \label{eq:comparacion_integrales}
    \end{equation}
\end{itemize}

\subsection{Teorema Fundamental del Cálculo}

El Teorema Fundamental del Cálculo establece la conexión profunda entre diferenciación e integración, y consta de dos partes:

\textbf{Primera Parte:} Si $f$ es continua en $[a, b]$ y $F$ es una función definida por:
\begin{equation}
F(x) = \int_a^x f(t)\, dt
\label{eq:tfc_parte1}
\end{equation}

Entonces $F$ es diferenciable en $(a, b)$ y $F'(x) = f(x)$. Esto significa que la derivada de una integral con límite superior variable es la función original.

\textbf{Segunda Parte:} Si $f$ es continua en $[a, b]$ y $F$ es cualquier antiderivada de $f$ (es decir, $F'(x) = f(x)$), entonces:
\begin{equation}
\int_a^b f(x)\, dx = F(b) - F(a)
\label{eq:tfc_parte2}
\end{equation}

Esta segunda parte proporciona el método práctico para evaluar integrales definidas: encontrar una antiderivada y evaluar la diferencia en los límites de integración.

\subsection{Aplicaciones de las Integrales Definidas}

Las integrales definidas tienen aplicaciones extensas en diversos campos:

\begin{itemize}
    \item \textbf{Cálculo de áreas:} Permiten determinar el área entre una curva y el eje $x$, o entre dos curvas.
    \item \textbf{Física:} Se utilizan para calcular trabajo, energía, centro de masa, momento de inercia, flujo de fluidos y campo eléctrico.
    \item \textbf{Ingeniería:} Son fundamentales para el análisis de estructuras, circuitos eléctricos, transferencia de calor y análisis de señales.
    \item \textbf{Economía:} Calculan excedente del consumidor y del productor, valor presente de flujos de ingresos continuos.
    \item \textbf{Probabilidad y estadística:} La integral definida permite calcular probabilidades en distribuciones continuas.
    \item \textbf{Biología:} Modelan crecimiento poblacional acumulado, dosis total de medicamentos y tasas de reacción bioquímica.
\end{itemize}

\vspace{1cm}

\subsection{Ejercicios: Integrales Definidas}

\textbf{Ejercicio 1.} $\int_{-3}^{2} x^4\, dx$

\textit{Solución:}

Primero encontramos la antiderivada:
$$\frac{x^5}{5} + C$$

Evaluamos en los límites:
$$\frac{(2)^5}{5} = \frac{32}{5}$$
$$\frac{(-3)^5}{5} = \frac{-243}{5}$$

Restamos:
$$\frac{32}{5} - \left(\frac{-243}{5}\right) = \frac{32}{5} + \frac{243}{5} = \frac{32 + 243}{5} = \frac{275}{5} = 55$$

\textbf{Ejercicio 2.} $\int_0^{\pi} \cos(5x)\, dx$

\textit{Solución:}

La antiderivada es:
$$\frac{\sin(5x)}{5} + C$$

Evaluamos en los límites:
$$\frac{\sin(5\pi)}{5} = \frac{0}{5} = 0$$
$$\frac{\sin(5 \cdot 0)}{5} = \frac{\sin(0)}{5} = \frac{0}{5} = 0$$

Por lo tanto:
$$0 - 0 = 0$$

\begin{figure}[H]
    \centering
    \includegraphics[width=0.7\textwidth]{img/3_1_imagen_integrales_definidas.jpg}
    \caption{Integrales Definidas}
\end{figure}

La siguiente figura muestra $\int_{0}^{\pi} (\sin(x) + 2)\, dx$, representando el área sombreada bajo la curva seno desplazada entre $0$ y $\pi$. Este ejemplo ilustra cómo la integral definida calcula la acumulación neta de la función en un intervalo específico.

\section{Integrales Indefinidas}

La integral indefinida, también conocida como antiderivada o primitiva, representa el conjunto de todas las funciones cuya derivada es una función dada. A diferencia de la integral definida, que produce un número específico, la integral indefinida produce una familia de funciones.

\subsection{Definición Formal de la Integral Indefinida}

Si $F'(x) = f(x)$ para todo $x$ en un intervalo, entonces $F(x)$ se denomina antiderivada o primitiva de $f(x)$. La integral indefinida de $f(x)$ se denota como:

\begin{equation}
\int f(x)\, dx = F(x) + C
\label{eq:integral_indefinida}
\end{equation}

Donde:
\begin{itemize}
    \item $f(x)$ es el integrando
    \item $F(x)$ es una antiderivada particular de $f(x)$
    \item $C$ es la constante de integración, que representa todas las posibles traslaciones verticales de $F(x)$
    \item $dx$ indica la variable de integración
\end{itemize}

La constante de integración $C$ es fundamental porque si $F(x)$ es una antiderivada de $f(x)$, entonces $F(x) + C$ también lo es para cualquier constante $C$, ya que la derivada de una constante es cero.

\subsection{Existencia y Unicidad de la Antiderivada}

Si una función $f$ es continua en un intervalo, entonces existe al menos una antiderivada $F$ de $f$ en ese intervalo. Además, si $F_1$ y $F_2$ son dos antiderivadas de $f$ en el mismo intervalo, entonces $F_1 - F_2 = C$, donde $C$ es una constante. Esto significa que dos antiderivadas de la misma función difieren únicamente por una constante.

\subsection{Propiedades Fundamentales de las Integrales Indefinidas}

Las integrales indefinidas comparten propiedades similares a las derivadas, pero en sentido inverso:

\begin{itemize}
    \item \textbf{Linealidad:} La integral de una suma es la suma de las integrales, y las constantes pueden salir fuera de la integral.
    \begin{equation}
    \int [cf(x) + g(x)]\, dx = c\int f(x)\, dx + \int g(x)\, dx
    \label{eq:linealidad_indefinida}
    \end{equation}

    \item \textbf{Relación con la derivada:} La derivada y la integral indefinida son operaciones inversas.
    \begin{align}
    \frac{d}{dx}\left[\int f(x)\, dx\right] &= f(x) \label{eq:derivada_integral}\\
    \int f'(x)\, dx &= f(x) + C \label{eq:integral_derivada}
    \end{align}
\end{itemize}

\subsection{Integrales Inmediatas y Fórmulas Básicas}

Existen fórmulas estándar de integración que se derivan directamente de las reglas de derivación:

\begin{itemize}
    \item \textbf{Potencias:} $\int x^n\, dx = \frac{x^{n+1}}{n+1} + C$, para $n \neq -1$
    \item \textbf{Recíproca:} $\int \frac{1}{x}\, dx = \ln|x| + C$
    \item \textbf{Exponencial:} $\int e^x\, dx = e^x + C$
    \item \textbf{Exponencial general:} $\int a^x\, dx = \frac{a^x}{\ln(a)} + C$
    \item \textbf{Trigonométricas:}
    \begin{align}
    \int \sin(x)\, dx &= -\cos(x) + C \notag\\
    \int \cos(x)\, dx &= \sin(x) + C \notag\\
    \int \sec^2(x)\, dx &= \tan(x) + C \notag\\
    \int \csc^2(x)\, dx &= -\cot(x) + C \notag\\
    \int \sec(x)\tan(x)\, dx &= \sec(x) + C \notag\\
    \int \csc(x)\cot(x)\, dx &= -\csc(x) + C \notag
    \end{align}
\end{itemize}

\subsection{Métodos de Integración}

Cuando las fórmulas directas no son aplicables, existen diversos métodos para calcular integrales indefinidas:

\begin{itemize}
    \item \textbf{Sustitución o cambio de variable:} Consiste en realizar un cambio de variable $u = g(x)$ para transformar la integral en una forma más simple. Este método es la contraparte de la regla de la cadena en derivación.
    \item \textbf{Integración por partes:} Se utiliza para productos de funciones y se explica en detalle en la siguiente sección.
    \item \textbf{Fracciones parciales:} Descompone funciones racionales en fracciones más simples que pueden integrarse individualmente.
    \item \textbf{Sustituciones trigonométricas:} Se emplean para integrar funciones que contienen expresiones como $\sqrt{a^2 - x^2}$, $\sqrt{a^2 + x^2}$ o $\sqrt{x^2 - a^2}$.
\end{itemize}

\subsection{Aplicaciones de las Integrales Indefinidas}

Las integrales indefinidas son fundamentales para:

\begin{itemize}
    \item \textbf{Resolver ecuaciones diferenciales:} Muchas ecuaciones diferenciales se resuelven mediante integración.
    \item \textbf{Encontrar funciones originales:} Cuando se conoce la tasa de cambio de una cantidad, la integral indefinida permite recuperar la función original.
    \item \textbf{Física:} Determinar la velocidad cuando se conoce la aceleración, o la posición cuando se conoce la velocidad.
    \item \textbf{Economía:} Calcular funciones de costo total a partir de funciones de costo marginal.
    \item \textbf{Ingeniería:} Analizar sistemas dinámicos y encontrar soluciones generales a problemas de modelado.
\end{itemize}

\vspace{1cm}

\subsection{Ejercicios: Integrales Indefinidas}

\textbf{Ejercicio 1.} $\int x^4\, dx$

\textit{Solución:}
$$\int x^4\, dx = \frac{x^{4+1}}{4+1} + C = \frac{x^5}{5} + C$$

\begin{figure}[H]
    \centering
    \includegraphics[width=0.7\textwidth]{img/3_2_imagen_integrales_indefinidas.jpg}
    \caption{Integrales Indefinidas}
\end{figure}

La figura muestra la familia de funciones $f(x) = x^2 + k$ para $k \in \{-3, -2, -1, 0, 1, 2, 3\}$, representando parábolas desplazadas verticalmente. Cada curva corresponde a una antiderivada diferente de $f'(x) = 2x$, ilustrando la constante de integración $C$ en las integrales indefinidas.

\section{Integrales por Partes}

La integración por partes es una técnica fundamental del cálculo integral que permite calcular la integral de un producto de funciones. Este método se deriva de la regla del producto para derivadas y es especialmente útil cuando otras técnicas de integración no son aplicables.

\subsection{Fórmula de Integración por Partes}

La fórmula de integración por partes se expresa como:

\begin{equation}
\int u\, dv = uv - \int v\, du
\label{eq:integracion_partes}
\end{equation}

O en su forma equivalente para integrales definidas:

\begin{equation}
\int_a^b u\, dv = [uv]_a^b - \int_a^b v\, du
\label{eq:integracion_partes_definida}
\end{equation}

Donde:
\begin{itemize}
    \item $u$ y $v$ son funciones diferenciables de $x$
    \item $du = u'(x)\, dx$
    \item $dv = v'(x)\, dx$
\end{itemize}

\subsection{Derivación de la Fórmula}

La fórmula de integración por partes se deriva directamente de la regla del producto para derivadas. Si $u$ y $v$ son funciones de $x$, entonces:

$$\frac{d}{dx}(uv) = u\frac{dv}{dx} + v\frac{du}{dx}$$

Integrando ambos lados:

$$uv = \int u\frac{dv}{dx}\, dx + \int v\frac{du}{dx}\, dx$$

Reordenando:

$$\int u\frac{dv}{dx}\, dx = uv - \int v\frac{du}{dx}\, dx$$

Lo cual se expresa de manera más compacta como $\int u\, dv = uv - \int v\, du$.

\subsection{Estrategia para Elegir $u$ y $dv$}

La clave para aplicar con éxito la integración por partes es elegir correctamente qué parte del integrando será $u$ y cuál será $dv$. Una guía útil es la regla mnemotécnica ILATE:

\begin{itemize}
    \item \textbf{I}nversa trigonométrica (arcsen, arctan, etc.)
    \item \textbf{L}ogarítmica (ln $x$, log $x$)
    \item \textbf{A}lgebraica (polinomios: $x$, $x^2$, $x^3$, etc.)
    \item \textbf{T}rigonométrica (sen $x$, cos $x$, tan $x$, etc.)
    \item \textbf{E}xponencial ($e^x$, $a^x$)
\end{itemize}

La función que aparezca primero en esta lista debería elegirse como $u$, y el resto como $dv$. Esta regla funciona porque:

\begin{itemize}
    \item Las funciones inversas trigonométricas y logarítmicas se simplifican al derivarlas
    \item Los polinomios eventualmente se reducen a cero con derivaciones sucesivas
    \item Las funciones trigonométricas y exponenciales se mantienen similares al derivar o integrar
\end{itemize}

\subsection{Ejemplos Típicos de Aplicación}

Algunos casos comunes donde se aplica integración por partes incluyen:

\begin{itemize}
    \item \textbf{Producto de polinomio y exponencial:} $\int x e^x\, dx$, donde $u = x$ y $dv = e^x\, dx$
    \item \textbf{Producto de polinomio y función trigonométrica:} $\int x \cos(x)\, dx$, donde $u = x$ y $dv = \cos(x)\, dx$
    \item \textbf{Funciones logarítmicas:} $\int \ln(x)\, dx$, donde $u = \ln(x)$ y $dv = dx$
    \item \textbf{Producto de exponencial y función trigonométrica:} Casos que requieren aplicar integración por partes dos veces y resolver algebraicamente
\end{itemize}

\subsection{Integración por Partes Repetida}

En algunos casos, es necesario aplicar la integración por partes múltiples veces. Por ejemplo, para calcular $\int x^2 e^x\, dx$, se debe aplicar el método dos veces, reduciendo gradualmente el exponente del polinomio.

En casos especiales, como $\int e^x \cos(x)\, dx$ o $\int e^x \sin(x)\, dx$, la integración por partes debe aplicarse dos veces, y luego la integral original aparece nuevamente en el resultado, permitiendo resolver algebraicamente para encontrar la solución.

\subsection{Integración por Partes Tabular}

Para integrales que requieren múltiples aplicaciones de integración por partes con polinomios, el método tabular (también llamado método de Tico) proporciona un procedimiento sistemático. Se crea una tabla con derivadas sucesivas de $u$ en una columna e integrales sucesivas de $dv$ en otra, alternando signos, hasta que la derivada de $u$ sea cero.

\subsection{Aplicaciones de la Integración por Partes}

La integración por partes tiene aplicaciones importantes en:

\begin{itemize}
    \item \textbf{Cálculo avanzado:} Resolución de integrales complejas que no pueden evaluarse con métodos básicos.
    \item \textbf{Ecuaciones diferenciales:} Muchas soluciones de ecuaciones diferenciales requieren integración por partes.
    \item \textbf{Transformadas integrales:} La transformada de Laplace y la transformada de Fourier emplean integración por partes en su desarrollo teórico.
    \item \textbf{Física matemática:} Aparece en el cálculo de funciones de Green, teoría de potencial y mecánica cuántica.
    \item \textbf{Ingeniería eléctrica:} Análisis de circuitos y procesamiento de señales.
\end{itemize}

\vspace{1cm}

\subsection{Ejercicios: Integrales por Partes}

\textbf{Ejercicio 1.} $\int x e^{2x}\, dx$ (por partes)

\textit{Solución:}

Aplicamos integración por partes:
$$u = x, \quad dv = e^{2x}\, dx$$
$$du = dx, \quad v = \frac{1}{2} e^{2x}$$

$$\int x e^{2x}\, dx = x \cdot \frac{1}{2} e^{2x} - \int \frac{1}{2} e^{2x}\, dx$$

Evaluamos la integral restante:
$$\int \frac{1}{2} e^{2x}\, dx = \frac{1}{2} \int e^{2x}\, dx = \frac{1}{2} \cdot \frac{1}{2} e^{2x} = \frac{1}{4} e^{2x}$$

Por lo tanto:
$$\int x e^{2x}\, dx = \frac{1}{2} x e^{2x} - \frac{1}{4} e^{2x} + C = \frac{e^{2x}}{4} (2x - 1) + C$$

\textbf{Ejercicio 2.} $\int 5x(3x-1)^4\, dx$ (por partes)

\textit{Solución:}

Aplicamos integración por partes:
$$u = 5x, \quad dv = (3x-1)^4\, dx$$
$$du = 5\, dx$$

Para encontrar $v$, hacemos sustitución: $w = 3x - 1$, $dw = 3\, dx$, $dx = \frac{dw}{3}$

$$v = \int w^4 \frac{dw}{3} = \frac{1}{3} \int w^4\, dw = \frac{1}{3} \cdot \frac{w^5}{5} = \frac{w^5}{15} = \frac{(3x-1)^5}{15}$$

Aplicamos la fórmula:
$$\int 5x(3x-1)^4\, dx = 5x \cdot \frac{(3x-1)^5}{15} - \int \frac{(3x-1)^5}{15} \cdot 5\, dx = \frac{x(3x-1)^5}{3} - \frac{1}{3} \int (3x-1)^5\, dx$$

Calculamos la integral restante con la misma sustitución:
$$\int (3x-1)^5\, dx = \int w^5 \frac{dw}{3} = \frac{1}{3} \int w^5\, dw = \frac{1}{3} \cdot \frac{w^6}{6} = \frac{(3x-1)^6}{18}$$

Por lo tanto:
$$\int 5x(3x-1)^4\, dx = \frac{x(3x-1)^5}{3} - \frac{1}{3} \cdot \frac{(3x-1)^6}{18} + C = \frac{(3x-1)^5(15x+1)}{54} + C$$

\textbf{Ejercicio 3.} $\int e^{4x} \cos(3x + 2)\, dx$ (por partes repetida)

\textit{Solución:}

Sea $I = \int e^{4x} \cos(3x + 2)\, dx$

Primera aplicación:
$$u = \cos(3x + 2), \quad du = -3 \sin(3x + 2)\, dx$$
$$dv = e^{4x}\, dx, \quad v = \frac{e^{4x}}{4}$$

$$I = \frac{e^{4x}}{4} \cos(3x + 2) + \frac{3}{4} \int e^{4x} \sin(3x + 2)\, dx$$

Segunda aplicación:
$$u = \sin(3x + 2), \quad du = 3 \cos(3x + 2)\, dx$$
$$dv = e^{4x}\, dx, \quad v = \frac{e^{4x}}{4}$$

$$\int e^{4x} \sin(3x + 2)\, dx = \frac{e^{4x}}{4} \sin(3x + 2) - \frac{3}{4} \int e^{4x} \cos(3x + 2)\, dx$$

Sustituyendo:
$$I = \frac{e^{4x}}{4} \cos(3x + 2) + \frac{3}{4} \left[ \frac{e^{4x}}{4} \sin(3x + 2) - \frac{3}{4} I \right]$$

$$I = \frac{e^{4x}}{4} \cos(3x + 2) + \frac{3e^{4x}}{16} \sin(3x + 2) - \frac{9}{16} I$$

$$I + \frac{9}{16} I = \frac{e^{4x}}{4} \cos(3x + 2) + \frac{3e^{4x}}{16} \sin(3x + 2)$$

$$\frac{25}{16} I = \frac{e^{4x}}{4} \cos(3x + 2) + \frac{3e^{4x}}{16} \sin(3x + 2)$$

$$I = \frac{4}{25} e^{4x} \cos(3x + 2) + \frac{3}{25} e^{4x} \sin(3x + 2) + C$$

\textbf{Ejercicio 4.} $\int e^{3x} \sin(4x + 2)\, dx$ (por partes repetida)

\textit{Solución:}

Sea $I = \int e^{3x} \sin(4x + 2)\, dx$

Primera aplicación:
$$u = \sin(4x + 2), \quad du = 4 \cos(4x + 2)\, dx$$
$$dv = e^{3x}\, dx, \quad v = \frac{1}{3} e^{3x}$$

$$I = \frac{1}{3} e^{3x} \sin(4x + 2) - \frac{4}{3} \int e^{3x} \cos(4x + 2)\, dx$$

Segunda aplicación:
$$u = \cos(4x + 2), \quad du = -4 \sin(4x + 2)\, dx$$
$$dv = e^{3x}\, dx, \quad v = \frac{1}{3} e^{3x}$$

$$\int e^{3x} \cos(4x + 2)\, dx = \frac{1}{3} e^{3x} \cos(4x + 2) + \frac{4}{3} I$$

Sustituyendo:
$$I = \frac{1}{3} e^{3x} \sin(4x + 2) - \frac{4}{3} \left[ \frac{1}{3} e^{3x} \cos(4x + 2) + \frac{4}{3} I \right]$$

$$I = \frac{1}{3} e^{3x} \sin(4x + 2) - \frac{4}{9} e^{3x} \cos(4x + 2) - \frac{16}{9} I$$

$$\frac{25}{9} I = \frac{1}{3} e^{3x} \sin(4x + 2) - \frac{4}{9} e^{3x} \cos(4x + 2)$$

$$I = \frac{3}{25} e^{3x} \sin(4x + 2) - \frac{4}{25} e^{3x} \cos(4x + 2) + C$$

$$I = \frac{e^{3x}}{25} \left[ 3 \sin(4x + 2) - 4 \cos(4x + 2) \right] + C$$

\textbf{Ejercicio 5.} $\int e^{-x} \cos(2x)\, dx$ (por partes repetida)

\textit{Solución:}

Sea $I = \int e^{-x} \cos(2x)\, dx$

Primera aplicación:
$$u = \cos(2x), \quad du = -2\sin(2x)\, dx$$
$$dv = e^{-x}\, dx, \quad v = -e^{-x}$$

$$I = -e^{-x} \cos(2x) - 2\int e^{-x} \sin(2x)\, dx$$

Segunda aplicación:
$$u = \sin(2x), \quad du = 2\cos(2x)\, dx$$
$$dv = e^{-x}\, dx, \quad v = -e^{-x}$$

$$\int e^{-x} \sin(2x)\, dx = -e^{-x} \sin(2x) + 2\int e^{-x} \cos(2x)\, dx$$

Sustituyendo:
$$I = -e^{-x} \cos(2x) + 2e^{-x} \sin(2x) - 4I$$

$$5I = -e^{-x} \cos(2x) + 2e^{-x} \sin(2x)$$

$$I = \frac{e^{-x}}{5} \left[2\sin(2x) - \cos(2x)\right] + C$$

\begin{figure}[H]
    \centering
    \includegraphics[width=0.7\textwidth]{img/3_3_imagen_integrales_por_partes.jpg}
    \caption{Integrales por Partes}
\end{figure}

La gráfica muestra la función $f(x) = x \ln(x)$, un ejemplo clásico que requiere el método de integración por partes. Esta técnica es esencial cuando el integrando es un producto de funciones donde ni la sustitución directa ni las fórmulas básicas son suficientes.

\section{Integrales Dobles}

Las integrales dobles extienden el concepto de integración a funciones de dos variables, permitiendo calcular volúmenes bajo superficies, áreas de regiones planas, masas de láminas con densidad variable y otros problemas en dos dimensiones.

\subsection{Definición Formal de la Integral Doble}

Una integral doble de una función $f(x, y)$ sobre una región $R$ en el plano $xy$ se define como el límite de una suma doble de Riemann:

\begin{equation}
\iint_R f(x, y)\, dA = \lim_{n,m \to \infty} \sum\sum f(x_i^*, y_j^*) \Delta A
\label{eq:integral_doble}
\end{equation}

Donde:
\begin{itemize}
    \item $R$ es la región de integración en el plano $xy$
    \item $f(x, y)$ es la función a integrar
    \item $dA = dx\, dy$ o $dA = dy\, dx$ representa el elemento de área infinitesimal
    \item El resultado representa el volumen bajo la superficie $z = f(x, y)$ sobre la región $R$
\end{itemize}

La región $R$ se subdivide en pequeños rectángulos de área $\Delta A = \Delta x \Delta y$, se evalúa la función en puntos representativos de cada rectángulo, se multiplica por el área y se suma. El límite cuando las subdivisiones tienden a infinito da el valor exacto de la integral doble.

\subsection{Teorema de Fubini}

El Teorema de Fubini es fundamental para evaluar integrales dobles, ya que permite convertirlas en integrales iteradas. Si $f(x, y)$ es continua en la región rectangular $R = [a, b] \times [c, d]$, entonces:

\begin{equation}
\iint_R f(x, y)\, dA = \int_a^b \int_c^d f(x, y)\, dy\, dx = \int_c^d \int_a^b f(x, y)\, dx\, dy
\label{eq:teorema_fubini}
\end{equation}

Esto significa que la integral doble puede calcularse mediante dos integrales simples sucesivas, y el orden de integración puede intercambiarse. Para regiones más generales, los límites de integración pueden ser funciones en lugar de constantes.

\subsection{Regiones de Integración}

Las regiones de integración para integrales dobles se clasifican en:

\begin{itemize}
    \item \textbf{Regiones de tipo I (verticalmente simples):} La región está acotada por dos funciones de $x$ en la dirección vertical:
    $$R = \{(x, y) : a \leq x \leq b, g_1(x) \leq y \leq g_2(x)\}$$
    La integral se expresa como:
    $$\iint_R f(x, y)\, dA = \int_a^b \int_{g_1(x)}^{g_2(x)} f(x, y)\, dy\, dx$$

    \item \textbf{Regiones de tipo II (horizontalmente simples):} La región está acotada por dos funciones de $y$ en la dirección horizontal:
    $$R = \{(x, y) : c \leq y \leq d, h_1(y) \leq x \leq h_2(y)\}$$
    La integral se expresa como:
    $$\iint_R f(x, y)\, dA = \int_c^d \int_{h_1(y)}^{h_2(y)} f(x, y)\, dx\, dy$$
\end{itemize}

La elección del orden de integración depende de la forma de la región y puede simplificar significativamente los cálculos.

\subsection{Cambio de Orden de Integración}

En algunos casos, cambiar el orden de integración puede simplificar el cálculo o hacer posible evaluar una integral que de otro modo sería muy difícil. Para cambiar el orden, es necesario reexpresar la región de integración según el nuevo orden y ajustar los límites de integración apropiadamente.

\subsection{Coordenadas Polares}

Cuando la región de integración o la función tienen simetría circular, es conveniente utilizar coordenadas polares. La transformación de coordenadas cartesianas a polares es:

$$x = r \cos(\theta), \quad y = r \sin(\theta)$$

El elemento de área se transforma como:
$$dA = dx\, dy = r\, dr\, d\theta$$

Por lo tanto, la integral doble en coordenadas polares se expresa como:

\begin{equation}
\iint_R f(x, y)\, dA = \int\int f(r \cos\theta, r \sin\theta)\, r\, dr\, d\theta
\label{eq:coordenadas_polares}
\end{equation}

Este cambio de variables es especialmente útil para regiones circulares, anulares o sectores circulares.

\subsection{Propiedades de las Integrales Dobles}

Las integrales dobles poseen propiedades similares a las integrales simples:

\begin{itemize}
    \item \textbf{Linealidad:}
    $$\iint_R [cf(x,y) + g(x,y)]\, dA = c\iint_R f(x,y)\, dA + \iint_R g(x,y)\, dA$$

    \item \textbf{Aditividad sobre regiones:} Si $R = R_1 \cup R_2$ y $R_1$ y $R_2$ no se solapan:
    $$\iint_R f(x,y)\, dA = \iint_{R_1} f(x,y)\, dA + \iint_{R_2} f(x,y)\, dA$$

    \item \textbf{Comparación:} Si $f(x,y) \leq g(x,y)$ en $R$, entonces:
    $$\iint_R f(x,y)\, dA \leq \iint_R g(x,y)\, dA$$
\end{itemize}

\subsection{Aplicaciones de las Integrales Dobles}

Las integrales dobles tienen aplicaciones extensas:

\begin{itemize}
    \item \textbf{Cálculo de volúmenes:} El volumen bajo una superficie $z = f(x, y)$ sobre una región $R$ es $\iint_R f(x, y)\, dA$.
    \item \textbf{Cálculo de áreas:} El área de una región plana $R$ es $\iint_R 1\, dA$.
    \item \textbf{Física:} Cálculo de masa, centro de masa y momentos de inercia de láminas planas con densidad variable $\rho(x, y)$.
    \item \textbf{Probabilidad:} Cálculo de probabilidades para variables aleatorias bidimensionales continuas.
    \item \textbf{Ingeniería:} Análisis de distribuciones de carga, flujo de calor bidimensional y campos electromagnéticos.
\end{itemize}

\vspace{1cm}

\subsection{Ejercicios: Integrales Dobles}

\textbf{Ejercicio 1.} $\int_{0}^{1} \int_{\sqrt{x}}^{1+x} (2xy)\, dy\, dx$

\textit{Solución:}

Integramos primero respecto a $y$:
$$\int_{\sqrt{x}}^{1+x} 2xy\, dy = x [y^2]_{\sqrt{x}}^{1+x} = x[(1+x)^2 - x]$$

$$= x[1 + 2x + x^2 - x] = x[1 + x + x^2] = x + x^2 + x^3$$

Ahora integramos respecto a $x$:
$$\int_{0}^{1} (x + x^2 + x^3)\, dx = \left[ \frac{x^2}{2} + \frac{x^3}{3} + \frac{x^4}{4} \right]_{0}^{1} = \frac{1}{2} + \frac{1}{3} + \frac{1}{4}$$

$$= \frac{6 + 4 + 3}{12} = \frac{13}{12}$$

\textbf{Ejercicio 2.} $\int_{1}^{4} \int_{1}^{2} (6x^2y)\, dy\, dx$

\textit{Solución:}

Integramos primero respecto a $y$:
$$\int_{1}^{2} 6x^2y\, dy = 6x^2 \left[ \frac{y^2}{2} \right]_{1}^{2} = 6x^2 \left( \frac{4}{2} - \frac{1}{2} \right) = 6x^2 \left( \frac{3}{2} \right) = 9x^2$$

Ahora integramos respecto a $x$:
$$\int_{1}^{4} 9x^2\, dx = 9 \left[ \frac{x^3}{3} \right]_{1}^{4} = 9 \left( \frac{64}{3} - \frac{1}{3} \right) = 9 \left( \frac{63}{3} \right) = 9 \cdot 21 = 189$$

\textbf{Ejercicio 3.} $\int_{1}^{4} \int_{0}^{2} (6x^2y - 2x)\, dy\, dx$

\textit{Solución:}

Integramos primero respecto a $y$:
$$\int_{0}^{2} (6x^2y - 2x)\, dy = \left[ 3x^2y^2 - 2xy \right]_{0}^{2} = 12x^2 - 4x$$

Ahora integramos respecto a $x$:
$$\int_{1}^{4} (12x^2 - 4x)\, dx = \left[ 4x^3 - 2x^2 \right]_{1}^{4} = (256 - 32) - (4 - 2) = 224 - 2 = 222$$

\textbf{Ejercicio 4.} $\int_{0}^{\pi/2} \int_{0}^{\pi/2} \sin(x-y)\, dy\, dx$

\textit{Solución:}

Integramos primero respecto a $y$:
$$\int_{0}^{\pi/2} \sin(x-y)\, dy = \left[ \cos(x-y) \right]_{0}^{\pi/2} = \cos(x - \frac{\pi}{2}) - \cos(x)$$

Como $\cos(x - \frac{\pi}{2}) = \sin(x)$:
$$\int_{0}^{\pi/2} \sin(x-y)\, dy = \sin(x) - \cos(x)$$

Ahora integramos respecto a $x$:
$$\int_{0}^{\pi/2} (\sin(x) - \cos(x))\, dx = \left[ -\cos(x) - \sin(x) \right]_{0}^{\pi/2} = (0 - 1) - (-1 - 0) = -1 + 1 = 0$$

\textbf{Ejercicio 5.} $\int_{0}^{\pi/2} \int_{0}^{2x} \sin(x+2y)\, dy\, dx$

\textit{Solución:}

Integramos primero respecto a $y$:
$$\int_{0}^{2x} \sin(x+2y)\, dy = \left[ -\frac{1}{2}\cos(x+2y) \right]_{0}^{2x} = -\frac{1}{2} \cos(5x) + \frac{1}{2} \cos(x)$$

$$= \frac{1}{2} (\cos(x) - \cos(5x))$$

Ahora integramos respecto a $x$:
$$\int_{0}^{\pi/2} \frac{1}{2} (\cos(x) - \cos(5x))\, dx = \frac{1}{2} \left[ \sin(x) - \frac{1}{5}\sin(5x) \right]_{0}^{\pi/2}$$

$$= \frac{1}{2} \left[ \left(1 - \frac{1}{5}\right) - 0 \right] = \frac{1}{2} \left(\frac{4}{5}\right) = \frac{2}{5}$$

\textbf{Ejercicio 6.} $\int_{0}^{1} \int_{x^2}^{1} e^{2x} \sin(2y)\, dy\, dx$ (cambio de orden)

\textit{Solución:}

Cambiamos el orden de integración. La región original es $0 \leq x \leq 1$, $x^2 \leq y \leq 1$. En el nuevo orden: $0 \leq y \leq 1$, $0 \leq x \leq \sqrt{y}$.

$$\int_{0}^{1} \int_{0}^{\sqrt{y}} e^{2x} \sin(2y)\, dx\, dy$$

Integramos primero respecto a $x$:
$$\int_{0}^{\sqrt{y}} e^{2x} \sin(2y)\, dx = \sin(2y) \left[ \frac{1}{2} e^{2x} \right]_{0}^{\sqrt{y}} = \frac{1}{2} \sin(2y) (e^{2\sqrt{y}} - 1)$$

Ahora integramos respecto a $y$:
$$\int_{0}^{1} \frac{1}{2} \sin(2y) (e^{2\sqrt{y}} - 1)\, dy = \frac{1}{2} \int_{0}^{1} \sin(2y) e^{2\sqrt{y}}\, dy - \frac{1}{2} \int_{0}^{1} \sin(2y)\, dy$$

La segunda integral:
$$\frac{1}{2} \int_{0}^{1} \sin(2y)\, dy = \frac{1}{2} \left[ -\frac{1}{2} \cos(2y) \right]_{0}^{1} = \frac{1}{4} (1 - \cos(2)) \approx 0.25$$

\begin{figure}[H]
    \centering
    \includegraphics[width=0.7\textwidth]{img/3_4_imagen_integrales_dobles.jpg}
    \caption{Integrales Dobles}
\end{figure}

La superficie tridimensional muestra $z = 4 - x^2$, formando un túnel parabólico. Las integrales dobles permiten calcular el volumen bajo esta superficie sobre una región específica del plano $xy$, extendiendo el concepto de área bajo la curva a dimensiones superiores.

\section{Integrales Triples}

Las integrales triples extienden el concepto de integración a funciones de tres variables, permitiendo calcular volúmenes de regiones en el espacio tridimensional, masas de sólidos con densidad variable y otras cantidades físicas en tres dimensiones.

\subsection{Definición Formal de la Integral Triple}

Una integral triple de una función $f(x, y, z)$ sobre una región sólida $E$ en el espacio tridimensional se define como el límite de una suma triple de Riemann:

\begin{equation}
\iiint_E f(x, y, z)\, dV = \lim_{n,m,p \to \infty} \sum\sum\sum f(x_i^*, y_j^*, z_k^*) \Delta V
\label{eq:integral_triple}
\end{equation}

Donde:
\begin{itemize}
    \item $E$ es la región de integración en el espacio tridimensional
    \item $f(x, y, z)$ es la función a integrar
    \item $dV = dx\, dy\, dz$ representa el elemento de volumen infinitesimal
    \item El resultado puede interpretarse como una masa total si $f$ representa densidad, o como un valor acumulado en el espacio
\end{itemize}

La región $E$ se subdivide en pequeñas cajas rectangulares de volumen $\Delta V = \Delta x \Delta y \Delta z$, se evalúa la función en puntos representativos, se multiplica por el volumen y se suma.

\subsection{Evaluación de Integrales Triples mediante Integrales Iteradas}

Al igual que las integrales dobles, las integrales triples se evalúan mediante integrales iteradas. El orden de integración puede variar, y la elección depende de la geometría de la región:

$$\iiint_E f(x, y, z)\, dV = \int\int\int f(x, y, z)\, dz\, dy\, dx$$

Para evaluar una integral triple, se integra primero respecto a una variable manteniendo las otras dos constantes, luego respecto a la segunda variable, y finalmente respecto a la tercera.

\subsection{Tipos de Regiones de Integración}

Las regiones tridimensionales se pueden clasificar según cómo están acotadas:

\begin{itemize}
    \item \textbf{Regiones de tipo I:} La región $E$ está acotada superiormente e inferiormente por funciones de $x$ e $y$:
    $$E = \{(x, y, z) : (x, y) \in D, u_1(x,y) \leq z \leq u_2(x,y)\}$$

    Donde $D$ es una región en el plano $xy$. La integral se expresa como:
    $$\iiint_E f(x,y,z)\, dV = \iint_D \left[\int_{u_1(x,y)}^{u_2(x,y)} f(x,y,z)\, dz\right] dA$$

    \item \textbf{Regiones de tipo II y III:} De manera similar, las regiones pueden estar acotadas en otras direcciones.
\end{itemize}

La elección del tipo de región y del orden de integración puede simplificar considerablemente los cálculos.

\subsection{Coordenadas Cilíndricas}

Cuando la región tiene simetría cilíndrica (alrededor del eje $z$), es conveniente utilizar coordenadas cilíndricas:

$$x = r \cos(\theta), \quad y = r \sin(\theta), \quad z = z$$

El elemento de volumen se transforma como:
$$dV = dx\, dy\, dz = r\, dr\, d\theta\, dz$$

La integral triple en coordenadas cilíndricas se expresa como:

\begin{equation}
\iiint_E f(x, y, z)\, dV = \int\int\int f(r \cos\theta, r \sin\theta, z)\, r\, dr\, d\theta\, dz
\label{eq:coordenadas_cilindricas}
\end{equation}

Este sistema es útil para cilindros, conos y otras regiones con simetría rotacional alrededor de un eje.

\subsection{Coordenadas Esféricas}

Cuando la región tiene simetría esférica (alrededor del origen), es conveniente utilizar coordenadas esféricas:

$$x = \rho \sin(\phi) \cos(\theta), \quad y = \rho \sin(\phi) \sin(\theta), \quad z = \rho \cos(\phi)$$

Donde:
\begin{itemize}
    \item $\rho$ es la distancia al origen ($0 \leq \rho < \infty$)
    \item $\phi$ es el ángulo con el eje $z$ positivo ($0 \leq \phi \leq \pi$)
    \item $\theta$ es el ángulo en el plano $xy$ ($0 \leq \theta \leq 2\pi$)
\end{itemize}

El elemento de volumen se transforma como:
$$dV = \rho^2 \sin(\phi)\, d\rho\, d\phi\, d\theta$$

La integral triple en coordenadas esféricas se expresa como:

\begin{equation}
\iiint_E f(x, y, z)\, dV = \int\int\int f(\rho \sin\phi \cos\theta, \rho \sin\phi \sin\theta, \rho \cos\phi)\, \rho^2 \sin\phi\, d\rho\, d\phi\, d\theta
\label{eq:coordenadas_esfericas}
\end{equation}

Este sistema es especialmente útil para esferas, conos esféricos y regiones con simetría radial.

\subsection{Jacobiano de Transformaciones}

Cuando se realiza un cambio de variables general de $(x, y, z)$ a $(u, v, w)$, el elemento de volumen se transforma mediante el Jacobiano de la transformación:

$$dV = |J|\, du\, dv\, dw$$

Donde $J$ es el determinante Jacobiano:

$$J = \frac{\partial(x,y,z)}{\partial(u,v,w)} = \begin{vmatrix}
\frac{\partial x}{\partial u} & \frac{\partial x}{\partial v} & \frac{\partial x}{\partial w}\\
\frac{\partial y}{\partial u} & \frac{\partial y}{\partial v} & \frac{\partial y}{\partial w}\\
\frac{\partial z}{\partial u} & \frac{\partial z}{\partial v} & \frac{\partial z}{\partial w}
\end{vmatrix}$$

El valor absoluto del Jacobiano representa el factor de escala local del volumen bajo la transformación.

\subsection{Propiedades de las Integrales Triples}

Las integrales triples comparten propiedades con las integrales simples y dobles:

\begin{itemize}
    \item \textbf{Linealidad:}
    $$\iiint_E [cf(x,y,z) + g(x,y,z)]\, dV = c\iiint_E f(x,y,z)\, dV + \iiint_E g(x,y,z)\, dV$$

    \item \textbf{Aditividad sobre regiones:} Si $E = E_1 \cup E_2$ y las regiones no se solapan:
    $$\iiint_E f(x,y,z)\, dV = \iiint_{E_1} f(x,y,z)\, dV + \iiint_{E_2} f(x,y,z)\, dV$$
\end{itemize}

\subsection{Aplicaciones de las Integrales Triples}

Las integrales triples tienen aplicaciones fundamentales:

\begin{itemize}
    \item \textbf{Cálculo de volúmenes:} El volumen de una región sólida $E$ es $\iiint_E 1\, dV$.
    \item \textbf{Masa de sólidos:} Si $\rho(x, y, z)$ es la densidad en cada punto, la masa total es:
    $$M = \iiint_E \rho(x, y, z)\, dV$$

    \item \textbf{Centro de masa:} Las coordenadas del centro de masa de un sólido son:
    $$\bar{x} = \frac{1}{M}\iiint_E x\rho(x,y,z)\, dV, \quad \bar{y} = \frac{1}{M}\iiint_E y\rho(x,y,z)\, dV, \quad \bar{z} = \frac{1}{M}\iiint_E z\rho(x,y,z)\, dV$$

    \item \textbf{Momento de inercia:} Se calcula mediante integrales triples ponderadas por distancias al cuadrado.
    \item \textbf{Física:} Cálculo de carga eléctrica total, flujo de calor tridimensional, potencial gravitatorio.
    \item \textbf{Ingeniería:} Análisis de esfuerzos y deformaciones en estructuras tridimensionales, distribuciones de temperatura.
    \item \textbf{Probabilidad:} Funciones de densidad de probabilidad para tres variables aleatorias continuas.
\end{itemize}

\begin{figure}[H]
    \centering
    \includegraphics[width=0.7\textwidth]{img/3_5_imagen_integrales_triples.jpg}
    \caption{Integrales Triples}
\end{figure}

La esfera sólida definida por $x^2 + y^2 + z^2 = 25$ representa una región común de integración triple. Este tipo de integrales se utiliza para calcular volúmenes, masas y otras propiedades físicas de sólidos tridimensionales con densidades variables.

% ==================== CAPÍTULO 4 ====================
\chapter{Unidad IV: Funciones Vectoriales y Cálculo Vectorial}

Las funciones vectoriales constituyen una extensión fundamental del cálculo hacia el análisis de curvas y trayectorias en el espacio. A diferencia de las funciones escalares tradicionales, las funciones vectoriales asignan un vector a cada valor del parámetro, permitiendo describir movimientos, flujos y campos de manera precisa y elegante.

En el contexto de la ingeniería, estas funciones son esenciales para modelar trayectorias de partículas, diseñar curvas en sistemas CAD, analizar campos electromagnéticos y resolver problemas de mecánica de fluidos. Este marco teórico es fundamental para la descripción del movimiento en el espacio y el análisis de campos de fuerza. Este capítulo explora los conceptos fundamentales de las funciones vectoriales, sus propiedades matemáticas, el cálculo diferencial e integral aplicado a ellas, y sus múltiples aplicaciones en ciencia e ingeniería.

\section{Funciones Vectoriales}

Una función vectorial es aquella cuyo dominio es un conjunto de números reales y cuyo rango es un conjunto de vectores. Estas funciones permiten describir curvas en el plano y en el espacio tridimensional mediante un único parámetro.

\subsection{Definición, Dominio y Representación}

Una \textbf{función vectorial} es una función cuyo dominio es un conjunto de números reales y cuyo rango es un conjunto de vectores. Se denota comúnmente como $\mathbf{r}(t)$. En el espacio tridimensional, una función vectorial se expresa mediante sus funciones componentes $f(t)$, $g(t)$ y $h(t)$, que son funciones escalares de valor real:

\begin{equation}
    \mathbf{r}(t) = f(t)\mathbf{i} + g(t)\mathbf{j} + h(t)\mathbf{k} = \langle f(t), g(t), h(t) \rangle
    \label{eq:funcion_vectorial}
\end{equation}

El \textbf{dominio} de $\mathbf{r}$ consiste en todos los valores de $t$ para los cuales todas las funciones componentes $f$, $g$ y $h$ están definidas simultáneamente:

\begin{equation}
    \text{Dom}(\mathbf{r}) = \text{Dom}(f) \cap \text{Dom}(g) \cap \text{Dom}(h)
    \label{eq:dominio_vectorial}
\end{equation}

Geométricamente, el conjunto de todos los puntos $(x, y, z)$ en el espacio definidos por $x=f(t)$, $y=g(t)$ y $z=h(t)$ forma una \textbf{curva espacial} $C$. La variable $t$ se denomina \textit{parámetro} y determina la \textit{orientación} de la curva (el sentido en que se traza a medida que $t$ aumenta).

\subsection{Ejercicios: Determinación de Dominio}

\textbf{Ejercicio 1.} Determinar el dominio de la función vectorial $\mathbf{r}(t) = \sqrt{4-t^2}\,\mathbf{i} + e^{-3t}\,\mathbf{j} + \ln(t+1)\,\mathbf{k}$.

\textit{Solución:}

Analizamos el dominio de cada componente por separado:
\begin{enumerate}
    \item \textbf{Componente $\mathbf{i}$:} $f(t) = \sqrt{4-t^2}$. Para que la raíz sea real:
    \[ 4 - t^2 \ge 0 \implies t^2 \le 4 \implies -2 \le t \le 2 \]
    \[ \text{Dom}(f) = [-2, 2] \]
    
    \item \textbf{Componente $\mathbf{j}$:} $g(t) = e^{-3t}$. La exponencial existe en todos los reales:
    \[ \text{Dom}(g) = (-\infty, \infty) \]
    
    \item \textbf{Componente $\mathbf{k}$:} $h(t) = \ln(t+1)$. El argumento debe ser positivo:
    \[ t + 1 > 0 \implies t > -1 \]
    \[ \text{Dom}(h) = (-1, \infty) \]
\end{enumerate}

La intersección de los tres intervalos es:
\[ \text{Dom}(\mathbf{r}) = (-1, 2] \]

\subsection{Límites y Continuidad}

El límite de una función vectorial se define tomando los límites de sus funciones componentes. Si $\mathbf{r}(t) = \langle f(t), g(t), h(t) \rangle$, entonces:

\begin{equation}
    \lim_{t \to a} \mathbf{r}(t) = \left\langle \lim_{t \to a} f(t), \lim_{t \to a} g(t), \lim_{t \to a} h(t) \right\rangle
    \label{eq:limite_vectorial}
\end{equation}

siempre que existan los límites de las funciones componentes.

Una función vectorial $\mathbf{r}$ es \textbf{continua} en un punto $a$ si se cumple que:
\[ \lim_{t \to a} \mathbf{r}(t) = \mathbf{r}(a) \]

Intuitivamente, esto significa que la curva no tiene roturas ni saltos en ese punto.

\subsection{Ejercicios: Cálculo de Límites}

\textbf{Ejercicio 2.} Calcular el límite $\lim_{t \to 0} \left( \frac{\sin t}{t}\,\mathbf{i} + (t^2-3)\,\mathbf{j} + \frac{e^t - 1}{t}\,\mathbf{k} \right)$.

\textit{Solución:}

Calculamos el límite componente a componente:
\begin{itemize}
    \item Componente $\mathbf{i}$: $\lim_{t \to 0} \frac{\sin t}{t} = 1$ (Límite notable).
    \item Componente $\mathbf{j}$: $\lim_{t \to 0} (t^2 - 3) = -3$.
    \item Componente $\mathbf{k}$: Aplicando la Regla de L'Hôpital:
    \[ \lim_{t \to 0} \frac{e^t - 1}{t} = \lim_{t \to 0} \frac{e^t}{1} = 1 \]
\end{itemize}

Por lo tanto:
\[ \lim_{t \to 0} \mathbf{r}(t) = \langle 1, -3, 1 \rangle \]

\begin{figure}[H]
    \centering
    \includegraphics[width=0.7\textwidth]{img/4_1_imagen_funciones_vectoriales.jpg}
    \caption{Funciones Vectoriales}
\end{figure}

La hélice tridimensional muestra la curva paramétrica $\mathbf{r}(t) = (3\cos(t), 3\sin(t), t)$ para $t \in [0, 4\pi]$. Esta función vectorial describe un resorte que asciende uniformemente mientras gira alrededor del eje $z$, combinando movimiento circular en el plano $xy$ con desplazamiento vertical.

\section{Ecuaciones Paramétricas}

Las ecuaciones paramétricas constituyen una herramienta fundamental para representar curvas en el plano y en el espacio. En lugar de expresar una variable en función de otra directamente, se expresan todas las coordenadas en función de un parámetro auxiliar.

\subsection{Parametrización y Curvas Planas}

Las ecuaciones paramétricas permiten representar curvas mediante una variable auxiliar $t$. En el plano, una curva $C$ se define por $x = f(t)$ y $y = g(t)$, donde $t$ varía en un intervalo $I$:

\begin{equation}
    x = f(t), \quad y = g(t), \quad t \in [a, b]
    \label{eq:parametricas_plano}
\end{equation}

Esta representación tiene ventajas sobre la cartesiana $y=f(x)$ ya que permite describir curvas que no son funciones (como circunferencias o espirales) y define inherentemente una dirección de movimiento.

\subsection{Eliminación del Parámetro}

Una técnica común para identificar la gráfica es la \textbf{eliminación del parámetro} para obtener la ecuación rectangular. Esto consiste en despejar el parámetro de una ecuación y sustituirlo en la otra, o usar identidades matemáticas apropiadas. Sin embargo, esto a veces pierde la información sobre la orientación o restringe el dominio original.

\subsection{Ejercicios: Eliminación del Parámetro}

\textbf{Ejercicio 3.} Dadas $x = 2\cos \theta$ y $y = 3\sin \theta$ ($0 \le \theta \le 2\pi$), hallar la ecuación cartesiana.

\textit{Solución:}

\begin{enumerate}
    \item Despejamos las funciones trigonométricas: $\frac{x}{2} = \cos \theta$, $\frac{y}{3} = \sin \theta$.
    \item Usamos la identidad $\sin^2 \theta + \cos^2 \theta = 1$:
    \[ \left(\frac{x}{2}\right)^2 + \left(\frac{y}{3}\right)^2 = 1 \implies \frac{x^2}{4} + \frac{y^2}{9} = 1 \]
\end{enumerate}

Es una elipse con centro en el origen, eje mayor vertical de longitud 6 y eje menor horizontal de longitud 4.

\subsection{Cálculo en Curvas Paramétricas}

Dada una curva definida paramétricamente por $x=f(t)$ y $y=g(t)$, podemos aplicar el cálculo sin necesidad de eliminar el parámetro:

\begin{itemize}
    \item \textbf{Pendiente de la tangente:} Se aplica la regla de la cadena.
    \begin{equation}
        \frac{dy}{dx} = \frac{dy/dt}{dx/dt}, \quad \text{si } \frac{dx}{dt} \neq 0
        \label{eq:derivada_parametrica}
    \end{equation}
    
    \item \textbf{Concavidad (Segunda derivada):}
    \begin{equation}
        \frac{d^2y}{dx^2} = \frac{\frac{d}{dt}\left(\frac{dy}{dx}\right)}{\frac{dx}{dt}}
        \label{eq:segunda_derivada_parametrica}
    \end{equation}
    
    \item \textbf{Longitud de Arco:} Si $f'$ y $g'$ son continuas en $[a,b]$ y la curva no se interseca a sí misma, la longitud $L$ es:
    \begin{equation}
        L = \int_a^b \sqrt{\left(\frac{dx}{dt}\right)^2 + \left(\frac{dy}{dt}\right)^2} \, dt
        \label{eq:longitud_arco}
    \end{equation}
\end{itemize}

\subsection{Ejercicios: Recta Tangente}

\textbf{Ejercicio 4.} Hallar la ecuación de la recta tangente a la curva $x = t^2$, $y = t^3 - 3t$ en $t = 2$.

\textit{Solución:}

\begin{enumerate}
    \item Punto de tangencia en $t=2$: $x_1 = (2)^2 = 4$, $y_1 = (2)^3 - 3(2) = 8 - 6 = 2$. Punto $(4, 2)$.
    \item Derivadas: $\frac{dx}{dt} = 2t$, $\frac{dy}{dt} = 3t^2 - 3$.
    \item Pendiente $m$ en $t = 2$:
    \[ m = \frac{3(2)^2 - 3}{2(2)} = \frac{12 - 3}{4} = \frac{9}{4} \]
    \item Ecuación punto-pendiente: $y - 2 = \frac{9}{4}(x - 4)$.
\end{enumerate}

La ecuación de la recta tangente es: $y = \frac{9}{4}x - 7$.

\begin{figure}[H]
    \centering
    \includegraphics[width=0.7\textwidth]{img/4_2_imagen_ecuaciones_parametricas.jpg}
    \caption{Ecuaciones Paramétricas}
\end{figure}

La trayectoria cerrada muestra la curva paramétrica $x = \sin(t) - \sin(2t)$, $y = \cos(t)$ para $t \in [0, 2\pi]$. Las ecuaciones paramétricas permiten describir curvas complejas que serían difíciles o imposibles de expresar como funciones explícitas $y = f(x)$.

\section{Derivación e Integración de Funciones Vectoriales}

El cálculo diferencial e integral se extiende naturalmente a las funciones vectoriales, operando componente a componente. Estas operaciones tienen importantes interpretaciones geométricas y físicas.

\subsection{Derivada de una Función Vectorial}

La derivada de una función vectorial se define como el límite del cociente incremental:

\begin{equation}
    \mathbf{r}'(t) = \lim_{\Delta t \to 0} \frac{\mathbf{r}(t+\Delta t) - \mathbf{r}(t)}{\Delta t}
    \label{eq:definicion_derivada_vectorial}
\end{equation}

Si $\mathbf{r}(t) = \langle f(t), g(t), h(t) \rangle$ y las funciones componentes son diferenciables, entonces:

\begin{equation}
    \mathbf{r}'(t) = \langle f'(t), g'(t), h'(t) \rangle
    \label{eq:derivada_vectorial}
\end{equation}

Si $\mathbf{r}'(t)$ existe y es distinta de cero, representa el \textbf{vector tangente} a la curva en el punto $P(t)$ y apunta en la dirección de la orientación de la curva. Si $\mathbf{r}(t)$ representa la posición de una partícula, $\mathbf{r}'(t)$ es el vector velocidad $\mathbf{v}(t)$.

\subsection{Propiedades de la Derivada Vectorial}

Sean $\mathbf{u}(t)$ y $\mathbf{v}(t)$ funciones vectoriales diferenciables, $c$ un escalar y $f(t)$ una función real:

\begin{enumerate}
    \item $\frac{d}{dt}[c\mathbf{u}(t)] = c\mathbf{u}'(t)$
    \item $\frac{d}{dt}[\mathbf{u}(t) \pm \mathbf{v}(t)] = \mathbf{u}'(t) \pm \mathbf{v}'(t)$
    \item $\frac{d}{dt}[f(t)\mathbf{u}(t)] = f'(t)\mathbf{u}(t) + f(t)\mathbf{u}'(t)$
    \item $\frac{d}{dt}[\mathbf{u}(t) \cdot \mathbf{v}(t)] = \mathbf{u}'(t) \cdot \mathbf{v}(t) + \mathbf{u}(t) \cdot \mathbf{v}'(t)$ (Regla del producto punto)
    \item $\frac{d}{dt}[\mathbf{u}(t) \times \mathbf{v}(t)] = \mathbf{u}'(t) \times \mathbf{v}(t) + \mathbf{u}(t) \times \mathbf{v}'(t)$ (Regla del producto cruz)
\end{enumerate}

\subsection{Vector Tangente Unitario}

El \textbf{vector tangente unitario} $\mathbf{T}(t)$ normaliza la dirección del movimiento:

\begin{equation}
    \mathbf{T}(t) = \frac{\mathbf{r}'(t)}{||\mathbf{r}'(t)||}, \quad \text{si } ||\mathbf{r}'(t)|| \neq 0
    \label{eq:tangente_unitario}
\end{equation}

\subsection{Integral de una Función Vectorial}

La integral de una función vectorial se calcula integrando cada componente:

\begin{equation}
    \int \mathbf{r}(t)\, dt = \left\langle \int f(t)\, dt, \int g(t)\, dt, \int h(t)\, dt \right\rangle
    \label{eq:integral_vectorial}
\end{equation}

\subsection{Ejercicios: Cálculo Vectorial}

\textbf{Ejercicio 5.} Hallar el vector tangente unitario $\mathbf{T}(t)$ para $\mathbf{r}(t) = \cos t \,\mathbf{i} + \sin t \,\mathbf{j} + t \,\mathbf{k}$ en $t = \pi/2$.

\textit{Solución:}

\begin{enumerate}
    \item Derivada: $\mathbf{r}'(t) = \langle -\sin t, \cos t, 1 \rangle$.
    \item En $t = \pi/2$: $\mathbf{r}'(\pi/2) = \langle -\sin(\pi/2), \cos(\pi/2), 1 \rangle = \langle -1, 0, 1 \rangle$.
    \item Magnitud: $||\mathbf{r}'(\pi/2)|| = \sqrt{(-1)^2 + 0^2 + 1^2} = \sqrt{2}$.
    \item Vector tangente unitario:
    \[ \mathbf{T}(\pi/2) = \frac{\langle -1, 0, 1 \rangle}{\sqrt{2}} = \left\langle -\frac{1}{\sqrt{2}}, 0, \frac{1}{\sqrt{2}} \right\rangle \]
\end{enumerate}

\vspace{0.5cm}

\textbf{Ejercicio 6.} Calcular $\int_{0}^{1} (2t \,\mathbf{i} + 4t^3 \,\mathbf{j} + e^t \,\mathbf{k}) \, dt$.

\textit{Solución:}

Integramos componente a componente:
\begin{itemize}
    \item Componente $\mathbf{i}$: $\int_{0}^{1} 2t \, dt = [t^2]_0^1 = 1$
    \item Componente $\mathbf{j}$: $\int_{0}^{1} 4t^3 \, dt = [t^4]_0^1 = 1$
    \item Componente $\mathbf{k}$: $\int_{0}^{1} e^t \, dt = [e^t]_0^1 = e - 1$
\end{itemize}

Por lo tanto:
\[ \int_{0}^{1} \mathbf{r}(t) \, dt = 1\mathbf{i} + 1\mathbf{j} + (e-1)\mathbf{k} = \langle 1, 1, e-1 \rangle \]

\begin{figure}[H]
    \centering
    \includegraphics[width=0.7\textwidth]{img/4_3_imagen_derivacion_vectorial.jpg}
    \caption{Derivación e Integración de Funciones Vectoriales}
\end{figure}

La parábola muestra la curva $\mathbf{c}(t) = (t, t^2, 0)$ para $t \in [-2, 2]$ junto con un vector tangente. La derivada de una función vectorial $\mathbf{r}'(t)$ proporciona el vector velocidad, indicando la dirección y magnitud del movimiento instantáneo a lo largo de la trayectoria.

\section{Integrales de Línea}

La \textbf{integral de línea} generaliza la integral definida simple $\int_a^b f(x)dx$. En lugar de integrar sobre un intervalo en el eje $x$, integramos sobre una curva $C$ en el plano o el espacio. Las integrales de línea son fundamentales en física e ingeniería para calcular trabajo, circulación y flujo.

\subsection{Integral de Línea de un Campo Escalar}

Si $f(x,y,z)$ es una función continua definida en una curva suave $C$ dada por $\mathbf{r}(t)$, la integral de línea es:

\begin{equation}
    \int_C f(x,y,z) \, ds = \int_a^b f(x(t), y(t), z(t)) \, ||\mathbf{r}'(t)|| \, dt
    \label{eq:integral_linea_escalar}
\end{equation}

donde $ds$ representa el elemento de longitud de arco. Físicamente, si $\rho(x,y,z)$ representa la densidad lineal de un alambre con la forma de la curva $C$, esta integral calcula la \textbf{masa total} del alambre.

\subsection{Integral de Línea de un Campo Vectorial (Trabajo)}

Sea $\mathbf{F}$ un campo vectorial de fuerza continuo definido sobre una curva suave $C$. El trabajo realizado por $\mathbf{F}$ para mover una partícula a lo largo de $C$ es:

\begin{equation}
    W = \int_C \mathbf{F} \cdot d\mathbf{r} = \int_a^b \mathbf{F}(\mathbf{r}(t)) \cdot \mathbf{r}'(t) \, dt
    \label{eq:integral_linea_vectorial}
\end{equation}

\subsection{Independencia de la Trayectoria}

Una propiedad importante es la \textbf{independencia de la trayectoria}. Si $\mathbf{F}$ es un \textit{campo conservativo} (es decir, $\mathbf{F} = \nabla f$ para alguna función potencial $f$), entonces la integral de línea depende solo de los puntos inicial y final, y es cero para cualquier trayectoria cerrada simple.

\subsection{Ejercicios: Integrales de Línea}

\textbf{Ejercicio 7.} Evaluar $\int_C 2xy \, ds$ sobre $C: x = \cos t, y = \sin t$ para $t \in [0, \pi/2]$ (primer cuadrante).

\textit{Solución:}

\begin{enumerate}
    \item Calculamos el diferencial de arco:
    \[ ds = \sqrt{\left(\frac{dx}{dt}\right)^2 + \left(\frac{dy}{dt}\right)^2} \, dt = \sqrt{(-\sin t)^2 + (\cos t)^2} \, dt = \sqrt{\sin^2 t + \cos^2 t} \, dt = 1 \, dt \]
    
    \item Sustituimos en la integral:
    \[ \int_C 2xy \, ds = \int_{0}^{\pi/2} 2(\cos t)(\sin t) \cdot 1 \, dt = \int_{0}^{\pi/2} \sin(2t) \, dt \]
    
    \item Evaluamos:
    \[ \left[-\frac{1}{2}\cos(2t)\right]_0^{\pi/2} = -\frac{1}{2}\cos(\pi) + \frac{1}{2}\cos(0) = -\frac{1}{2}(-1) + \frac{1}{2}(1) = \frac{1}{2} + \frac{1}{2} = 1 \]
\end{enumerate}

El valor de la integral es $1$.

\vspace{0.5cm}

\textbf{Ejercicio 8.} Calcular el trabajo realizado por el campo de fuerza $\mathbf{F} = \langle y, -x \rangle$ sobre una partícula que se mueve a lo largo de la curva $\mathbf{r}(t) = \langle t, t^2 \rangle$ para $t \in [0, 1]$.

\textit{Solución:}

\begin{enumerate}
    \item Evaluamos el campo en la curva: $\mathbf{F}(\mathbf{r}(t)) = \langle t^2, -t \rangle$.
    \item Calculamos la derivada de la curva: $\mathbf{r}'(t) = \langle 1, 2t \rangle$.
    \item Calculamos el producto punto:
    \[ \mathbf{F}(\mathbf{r}(t)) \cdot \mathbf{r}'(t) = (t^2)(1) + (-t)(2t) = t^2 - 2t^2 = -t^2 \]
    \item Evaluamos la integral:
    \[ W = \int_0^1 -t^2 \, dt = \left[-\frac{t^3}{3}\right]_0^1 = -\frac{1}{3} - 0 = -\frac{1}{3} \]
\end{enumerate}

El trabajo realizado es $W = -\frac{1}{3}$ unidades. El signo negativo indica que el campo de fuerza realiza trabajo negativo (opuesto al movimiento) sobre la partícula.

\begin{figure}[H]
    \centering
    \includegraphics[width=0.7\textwidth]{img/4_4_imagen_integrales_linea.jpg}
    \caption{Integrales de Línea}
\end{figure}

El campo vectorial muestra $\mathbf{F}(x,y) = (-y/3, x/3)$ representado por flechas, atravesado por una trayectoria curva. Las integrales de línea calculan el trabajo realizado por un campo de fuerzas a lo largo de una curva, siendo fundamentales en física e ingeniería para modelar trabajo, flujo y circulación.

% ==================== REFERENCIAS ====================
\begin{thebibliography}{99}

\bibitem{wolfram2025}
Wolfram MathWorld. (2025). \textit{Function}. Recuperado el 10 de octubre de 2025, de \url{https://mathworld.wolfram.com/Function.html}

\bibitem{stewart2016}
Stewart, J. (2016). \textit{Cálculo de una variable: Trascendentes tempranas} (7a ed.). Cengage Learning.

\bibitem{niven2006}
Niven, I., \& Zuckerman, H. S. (2006). \textit{Mathematics: A practical introduction} (2nd ed.). Wiley.

\bibitem{wikipedia_dominio}
Wikipedia. (2025). Dominio de una función. Recuperado el 10 de octubre de 2025, de \url{https://es.wikipedia.org/wiki/Dominio_de_una_función}

\bibitem{wikipedia_rango}
Wikipedia. (2025). Rango (matemáticas). Recuperado el 10 de octubre de 2025, de \url{https://es.wikipedia.org/wiki/Rango_(matemáticas)}

\bibitem{disfruta2025}
Disfruta las matemáticas. (2025). Dominio, codominio y rango. Recuperado el 10 de octubre de 2025, de \url{https://www.disfrutalasmatematicas.com/conjuntos/dominio-rango-codominio.html}

\bibitem{libretexts2022}
LibreTexts. (2022). 4.7: Dominio y rango de una función. Recuperado el 10 de octubre de 2025, de \url{https://espanol.libretexts.org/Bookshelves/Matematicas/Matematicas_Aplicadas/Calculo_para_Negocios_y_Ciencias_Sociales_Corequisite_Workbook_(Dominguez_Martinez_y_Saykali)/04:_Funciones/4.07:_Dominio_y_rango_de_una_función}

\bibitem{larson2017}
Larson, R., \& Edwards, B. H. (2017). \textit{Cálculo: Trascendentes tempranas} (9ª ed.). Cengage Learning.

\bibitem{wikipedia_implicita}
Wikipedia. (2025). Función implícita. En \textit{Wikipedia: La enciclopedia libre}. Recuperado el 15 de octubre de 2025, de \url{https://es.wikipedia.org/wiki/Función_implícita}

\bibitem{libretexts_implicita}
LibreTexts. (2023). Funciones explícitas e implícitas en cálculo. Recuperado el 15 de octubre de 2025, de \url{https://espanol.libretexts.org/Bookshelves/Matematicas/Calculo}

\bibitem{khan_academy}
Khan Academy. (2025). Funciones explícitas e implícitas. Recuperado el 15 de octubre de 2025, de \url{https://es.khanacademy.org/math/differential-calculus}

\bibitem{wikipedia_curvas}
Wikipedia. (2025). Curva de nivel. En \textit{Wikipedia: La enciclopedia libre}. Recuperado el 15 de octubre de 2025, de \url{https://es.wikipedia.org/wiki/Curva_de_nivel}

\bibitem{wikipedia_limite}
Wikipedia. (2025). Límite (matemáticas). En \textit{Wikipedia: La enciclopedia libre}. Recuperado el 15 de octubre de 2025, de \url{https://es.wikipedia.org/wiki/Límite_(matemáticas)}

\bibitem{thomas2018}
Thomas, G. B., Weir, M. D., Hass, J., \& Giordano, F. R. (2018). \textit{Thomas' Calculus} (14ª ed.). Pearson Education.

\bibitem{larson2018}
Larson, R., \& Edwards, B. H. (2018). \textit{Calculus} (11th ed.). Cengage Learning.

\bibitem{stewart2020}
Stewart, J. (2020). \textit{Calculus: Early Transcendentals} (9th ed.). Cengage Learning.

\bibitem{zill2011}
Zill, D. G., \& Wright, W. S. (2011). \textit{Calculus: Early Transcendentals} (4th ed.). Jones \& Bartlett Learning.

\end{thebibliography}

\end{document}
